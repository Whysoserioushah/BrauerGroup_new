\section{Wedderburn-Artin Theorem for Simple Rings}\label{sec:wed-artin}

\subsection{Classification of Simple Rings}\label{sec:wed-artin-proof}


\begin{lemma}[minimal ideal of simple rings]\label{lemma:min-ideal-simple-ring}
  Let $A$ be a ring and $I$ a non-trivial minimal left ideal of $A$, then $I$ is
  a simple $A$-module.
  \leanok
  \lean{minimal_ideal_isSimpleModule}
\end{lemma}
\begin{proof}
  Let $J\le I$ be an $A$-submodule of $I$, suppose $J$ is non-trivial, we prove
  that $J=I$. Then the image $J'$ of $J$ under $I \hookrightarrow A$ is a
  non-trivial left ideal of $A$. Since $I \hookrightarrow A$ is injective, it is
  sufficient to prove that $J' = I$. This is because $J'\le I$ and $J' \not< J$.
\end{proof}

\begin{lemma}
  Let $A$ be a simple ring and $I$ a non-trivial left ideal. One can write
  $1 \in A$ as $\sum_{i=0}^{n}x_{i}y_{i}$ for some $x_{i}\in I$ and
  $y_{i} \in A$.
\end{lemma}

\begin{proof}
  Let $I'$ be the two-sided ideal spanned by $I$. Then since $A$ is a simple
  ring, $I' = A$. Thus $1 \in I'$. One can write $1 \in A$ as
  $\sum_{i}a_{i}x_{i}b_{i}$ for some $x_{i}\in I$ and $a_{i},b_{i}\in A$, since
  $I$ is a left ideal $a_{i}x_{i}\in I$ as well.
\end{proof}

Now, we can find the smallest $n$ such that $1\in A$ can be written as
$\sum_{i=0}^{n}x_{i}y_{i}$ for some $x_{i}\in I$ and $y_{i}\in A$. Let us fix
the notations $n$, $x_{i}$ and $y_{i}$

\begin{lemma} The $n$, $x_{i}$ and $y_{i}$ are all non-zero. \leanok
  \lean{Wedderburn_Artin.aux.nxi_ne_zero}
\end{lemma}

\begin{proof}
  If $n$ is $0$, then $1 = 0$ in $A$, but all simple rings are non-trivial. We
  argue by contradiction to prove that all $x_{i}$ and $y_{i}$ are non-zero.
  Assume there exists a $j$ such that $y_{j}\ne0$ implies $x_{j} = 0$. Without
  loss of generality, we assume $j = 0$. Then
  $1 = \sum_{i=0}^{n}x_{i}y_{i}=\sum_{i=1}^{n}x_{i}y_{i}$. This contradicts the
  minimality of $n$.
\end{proof}

\begin{theorem}[Wedderburn]\label{thm:wed}
  Let $A$ be a simple ring and $I$ a non-trivial minimal left ideal. Then there
  exists a non-zero $n \in \mathbb{N}$ such that $A \cong I^{n}$ as $A$-modules.
  \leanok
  \lean{Wedderburn_Artin.aux.equivIdeal}
  \uses{lemma:min-ideal-simple-ring}
  \uses{lemma:min-ideal-simple-ring}
\end{theorem}

\begin{proof}
  We continue to write $1 = \sum_{i=0}^{n}x_{i}y_{i}$ in the shortest possible
  manner. Then we can define an $A$-linear map $g : I^{n}\to A$ by
  $(v_{i})\mapsto \sum v_{i}y_{i}$. Then $g$ is surjective: if $a \in A$, then
  $(ax_{i})$ is mapped to $a$ under $g$. $g$ is injective as well: support
  $g(v_{i})=0=\sum_{i}v_{i}y_{i}$ with $(v_{i})$ not all zero. Without loss of
  generality, we assume $v_{0} \ne 0$, then the ideal $\langle v_{0}\rangle$ is
  equal to $I$ (since $I$ is simple\cref{lemma:min-ideal-simple-ring}). Thus
  $x_{0}\in I = \langle v_{0}\rangle$; implying that $x_{0}=r\cdot v_{0}$ for
  some $r\in A$. Thus
  $1=1 - r\cdot 0 = \sum_{i=0}^{n}x_{i}y_{i}-\sum_{i=0}^{n}r\cdot v_{i}y_{i}$.
  In this way, we cancelled the term at $i=0$, contradicting the minimality of
  $n$. Hence $g$ is an isomorphism.
\end{proof}

\begin{theorem}[Wedderburn-Artin (Ideal)]
  Let $A$ be an Artinian simple ring. There exists a non-zero $n$ and an ideal
  $I \subseteq A$ such that $I$ is simple as an $A$-module and $A \cong I^{n}$
  as $A$-module.
  \label{thm:wed-artin-ideal}
  \leanok
  \lean{Wedderburn_Artin_ideal_version}
  \uses{thm:wed}
\end{theorem}

\begin{proof}
  By~\cref{thm:wed}, we only need a minimal left ideal. Since $A$ is Artinian,
  such ideal exists.
\end{proof}

\begin{theorem}[Wedderburn-Artin (Algebra)] \label{thm:wed-artin-algebra}
 Let $K$ be a field and $B$ an finite dimensional simple algebra over $K$.
  There exists a non-zero $n\in\mathbb{N}$ and a division $K$-algebra $S$ such
  that $B \cong \Mat_{n}(S)$.
  \leanok
  \lean{Wedderburn_Artin_algebra_version}
  \uses{thm:wed-artin-ideal, con:end-vec-iso-matrix}
\end{theorem}

\begin{proof}
  By~\cref{thm:wed-artin-ideal}, we can find a $n$ and a minimal left ideal $I$
  such $A \cong I^{n}$ as $A$-modules. Note that
  ${\left(\End_{B}I\right)}^{\opp}$ is a division ring. Then since
  $B^{\opp}\cong\End_{B}B\cong{\End_{B}(I^{n})}\cong\Mat_{n}\left(\End_{B}I\right)$
  as rings where the final isomorphism is from~\cref{con:end-vec-iso-matrix}, we
  have $e : B\cong \Mat_{n}\left(\End_{B}I\right)^{\opp}$ as rings. We also have
  a $K$-algebra structure on ${\left(\End_{B} I\right)}^{\opp}$ given by
  $(a \cdot f)(x)=f(a\cdot x)$, and this algebra structure promotes the ring
  isomorphism $e$ to a $k$-algebra isomorphism.
\end{proof}

\subsection{Uniqueness of the Classification}\label{sec:wed-artin-unique}
In the previous section, we know that finite dimensional simple $K$-algebra $B$
over is in fact a matrix algebras of a division $K$-algebra $S$. In this
section, we prove that the division algebra $S$ is essentially unique.

\begin{theorem}[Uniqueness of Wedderburn-Artin theorem]\label{thm:wed-artin-uniq}
  Let $B$ be a
  finite-dimensional simple $K$-algebra. Suppose $B$ is isomorphic as
  $k$-algebras to both $\Mat_{n}(D)$ and $\Mat_{n'}(D')$ where $n, n'$ are
  non-zero natural numbers and $D, D'$ are $k$-division algebra, then $n = n'$
  and $D \cong D'$ as $k$-algebras.\label{thm:wed-artin-unique} \leanok
  \lean{Wedderburn_Artin_uniqueness₀}
  \uses{lem:end-simple-iso}
\end{theorem}

\begin{proof}
  Since $D^{n}$ is a simple $B$-module, by~\cref{lem:end-simple-iso}, we see
  that $\End_{A}D^{n}\cong D^{\opp}$ and $\End_{A}D^{n}\cong D'^{\opp}$ as
  $k$-algebras. Thus $D^{\opp}\cong D'^{\opp}$ as $k$-algebras, consequently
  $D\cong D'$ as $k$-algebras as well. Since
  $A \cong \Mat_{n}(D)\cong \Mat_{n'}(D')\cong \Mat_{n'}(D)$ as $k$-algebras and
  $A$ is finite $k$-dimensional, a dimension argument shows that $n=n'$.
\end{proof}

%%% Local Variables:
%%% mode: LaTeX
%%% TeX-master: "../print"
%%% End:
