\subsection{$\HH^{2} \circ \cross$ and $\cross \circ \HH^{2}$}

For a finite dimensional Galois extension of field $K/F$, we have constructed two functions $\HH^{2}$ and $\cross$ between the second cohomology group $\HH^{2}\left(\gal(K/F), K^{\star}\right)$ and the relative Brauer group $\br(K/F)$. In this section, we prove that they are mutual inverse to one another,

\begin{lemma}\label{lem:relative-br-snd-inverse-1}
  The composition of $\cross$ and $\HH^{2}$ is the identity:
  \begin{center}
    \begin{tikzcd}
      \displaystyle \HH^2\left(\gal(K/F), K^\star\right) \arrow{r}{\cross} \arrow[bend right = 15, swap]{rr}{\id} &
      \br(K/F) \arrow{r}{\HH^2} & \displaystyle \HH^2\left(\gal(K/F), K^\star\right).
    \end{tikzcd}
  \end{center}
  \leanok
  \lean{RelativeBrGroup.toSnd_fromSnd}
\end{lemma}
\begin{proof}
  Let $\mfa$ be any 2-cocycle, by~\cref{lem:cross-product-basis-conj}, we notice that $x : \sigma \mapsto \Delta_{\sigma, 1}$ is a conjugation sequence for $\cross_{\mfa}$. Hence by~\cref{con:good-rep-to-2-cocycles}, ~\cref{cor:to-2-cocylce-wd} and~\cref{thm:snd-coh-to-relative-br}, we evaluate the composition at $\mfa$ as:
  \begin{center}
    \begin{tikzcd}
      {[\mfa]} \arrow[mapsto]{r} & {\left[\cross_\mfa\right]_{\sim_{\br}}} \arrow[mapsto]{r}& {\left[(\sigma,\tau)\mapsto\comp^{x}_{\Delta_{\sigma, 1}, \Delta_{\tau, 1}, \Delta_{\sigma\tau, 1}}\right]}
    \end{tikzcd}.
  \end{center}
  That is, we need to show that $\mfa$ and $(\sigma, \tau) \mapsto \comp_{\Delta_{\sigma, 1}, \Delta_{\tau, 1}, \Delta_{\sigma\tau, 1}}$ are 2-cohomologous. In fact, they are equal.
  By\cref{con:compare-conj-factors}~, we have that $\iota_{\cross_{\mfa}}\left(\comp^{x}_{\Delta_{\sigma, 1}, \Delta_{\tau, 1}, \Delta_{\sigma\tau, 1}}\right) = \Delta_{\sigma, 1}\Delta_{\tau, 1}\Delta_{\sigma\tau, 1}^{-1} = \mfa(\sigma,\tau)\cdot \Delta_{\sigma\tau, 1}\Delta_{\sigma\tau, 1}^{-1} = \mfa(\sigma,\tau)\cdot 1 = \Delta_{\id,\mfa(\sigma,\tau)}$ which is precisely $\iota_{\cross_{\mfa}}(\mfa(\sigma,\tau))$.
\end{proof}

\begin{lemma}\label{lem:relative-br-snd-inverse-2}
  The composition of $\HH^{2}$ and $\cross$ is the identity:
  \begin{center}
    \begin{tikzcd}
      \br(K/F) \arrow{r}{\HH^2} \arrow[bend right = 15, swap]{rr}{\id} &
      \HH^2\left(\gal(K/F), K^\star\right) \arrow{r}{\cross} &
      \br(K/F)
    \end{tikzcd}.
  \end{center}
  \leanok
  \lean{RelativeBrGroup.fromSnd_toSnd}
\end{lemma}
\begin{proof}
  Let $X \in \br(K/F)$, $A$ be an arbitrary good representation of $X$ and $x$ be an arbitrary $A$-conjugation sequence which exists by~\cref{cor:mem-relative-br-iff-good-rep} and~\cref{con:exists-conj-seq}. By~\cref{def:good-rep}, $X = \left[A\right]_{\sim_{\br}}$.
  Hence by~\cref{cor:to-2-cocylce-wd} and~\cref{thm:snd-coh-to-relative-br}, we evaluate the composition at $X$ as:
  \begin{center}
    \begin{tikzcd}
      {[A]_{\sim_{\br}}} \arrow[mapsto]{r} &
      {\left[\mathcal{B}^2_x\right]} \arrow[mapsto]{r} &
      {\left[\cross_{\mathcal{B}^2_x}\right]}
    \end{tikzcd}.
  \end{center}
  Hence we need to prove that $A$ and $\cross_{\mathcal{B}^{2}_{x}}$ are Brauer equivalent. We will show that they are isomorphic as $F$-algebras. since $\{x_{\sigma}|\sigma\in\gal(K/F)\}$ is a $K$-basis for $A$ and $\{\Delta_{\sigma, 1}|\sigma \in \gal(K/F)\}$ is a $K$-basis for $\cross_{\mathcal{B}^{2}_{x}}$, they are certainly isomorphic as $K$-modules. Let $\phi : \cross_{\mathcal{B}^{2}_{x}} \cong A$ be the $K$-linear isomorphism defined by $\Delta_{\sigma, 1} \mapsto x_{\sigma}$, since the $K$-action on $A$ and the $F$-action on $A$ are compatible (\cref{con:good-rep-mod}), $\phi$ is also an $F$-linear isomorphism. Like in~\cref{thm:snd-coh-to-relative-br}, we check that $\phi(1) = 1$ and $\phi(xy)=\phi(x)\phi(y)$ for all $x,y \in A$:
  \begin{enumerate}
    \item preservation of one: by~\cref{con:compare-conj-factors}, we have
          \[
          \begin{aligned}
            \phi(1)
            &= \phi\left(\Delta_{\id,\mathcal{B}^{2}_{x}(\id,\id)^{-1}}\right) \\
            &= \mathcal{B}^{2}_{x}(\id,\id)^{-1} \phi\left(\Delta_{\id, 1}\right) \\
            &= \mathcal{B}^{2}_{x}(\id,\id)^{-1} x_{\id} \\
            &= \comp_{x_{\id},x_{\id},x_{\id}}^{-1} x_{\id} \\
            &= \comp_{x_{\id},x_{\id},x_{\id}} x_{\id} x_{\id} x_{\id}^{-1} \\
            &= x_{\id} x_{\id}^{-1} \\
            &= 1.
          \end{aligned}
          \]
    \item preservation of multiplication: let $\sigma,\tau\in\gal(K/F)$ and $c,d \in K$, by~\cref{con:compare-conj-factors} and~\cref{def:conj-factor}, we have
          \[
          \begin{aligned}
            \phi\left(\Delta_{\sigma,c}\Delta_{\tau,d}\right)
            &= \phi\left(\Delta_{\sigma\tau, c\sigma(d)\mathcal{B}^{2}_{x}(\sigma,\tau)}\right) \\
            &= c\sigma(d)\mathcal{B}^{2}_{x}(\sigma,\tau) \,\cdot\, \phi\left(\Delta_{\sigma\tau, 1}\right) \\
            &= c\sigma(d)\mathcal{B}^{2}_{x}(\sigma,\tau) \,\cdot\, x_{\sigma\tau} \\
            &= c\sigma(d)\comp_{x_{\sigma},x_{\tau},x_{\sigma\tau}} \,\cdot\, x_{\sigma\tau} \\
            &= c\sigma(d)\,\cdot\, \iota_{A}\left(\comp_{x_{\sigma},x_{\tau},x_{\sigma\tau}}\right)x_{\sigma\tau} \\
            &= c\sigma(d) \,\cdot\, x_{\sigma}x_{\tau}\\
            %%%%
            %%%%
            \phi\left(\Delta_{\sigma,c}\right)\phi\left(\Delta_{\tau,d}\right)
            &=\left(c \cdot \phi\left(\Delta_{\sigma, 1}\right)\right)
              \left(d \cdot \phi\left(\Delta_{\tau, 1}\right)\right) \\
            &= \left(c \cdot x_{\sigma}\right) \left(d \cdot x_{\tau}\right)\\
            &= c \,\cdot\, x_{\sigma}\iota_{A}(d) x_{\tau} \\
            &= c\sigma_{d} \,\cdot\, x_{\sigma}x_{\tau}.
          \end{aligned}
          \]
  \end{enumerate}
\end{proof}

\begin{corollary}
  For a finite dimensional and Galois extension of field $K/F$, the relative Brauer group $K/F$ bijects to the second cohomology group $\HH^{2}\left(\gal(K/F), K^{\star}\right)$ by the following commutative diagram
  \begin{center}
  \begin{tikzcd}
    {\br(K/F)} \arrow{r}{\HH^2} \arrow[equal]{d} & {\HH^2\left(\gal(K/F), K^\star\right)} \arrow[equal]{d} \\
    {\br(K/ F)} & {\HH^2\left(\gal(K/F), K^{\star}\right)} \arrow{l}{\cross}
  \end{tikzcd}.
\end{center}
\leanok
\lean{RelativeBrGroup.equivSnd}
\end{corollary}

\begin{proof}
  Exactly~\cref{lem:relative-br-snd-inverse-1} and~\cref{lem:relative-br-snd-inverse-2}.
\end{proof}

\subsection{Group Homomorphism}

In previous sections, when $K/F$ is a finite dimensional Galois extension, we have set up a bijection between the relative Brauer group $\br(K/F)$ and the second cohomology group $\HH^{2}\left(\gal(K/F), K^{\star}\right)$. But both functions $\HH^{2}$ and $\cross$ are only set-theoretical function. In this section, we aim to upgrade them to group homomorphisms. Technically, we only need to prove either one of them preserves multiplication; we provide a proof that $\HH^{2}$ preserves one anyway because we found the proof to be entertaining.

\subsubsection{$\cross_{1} = 1$ and $\HH^{2}(1) = 1$}

\begin{theorem}
  \label{thm:map_one}
  The function $\cross:\HH^{2}\left(\gal(K/F), K^{\star}\right) \to \br(K/F)$ preserves one, that is $\cross_{}$
 \leanok
  \lean{map_one_proof.fromSnd_zero}
\end{theorem}

\begin{proof}
  Since $\{\Delta_{\sigma, 1}|\sigma\in\gal(K/F)\}$ is a $K$-basis for $\cross_{1}$ where $1 \in \mathcal{B}^{2}\left(\gal(K/F),K^{\star}\right)$ is the constant function $1$ (\cref{lem:cross-product-basis}), we construct a $K$-linear map $\phi : \cross_{1} \to \End_{F} K$ by $\Delta_{\sigma, 1} \mapsto \sigma$; note that $\phi$ is $F$-linear as well. In fact, $\phi$ is also an $F$-algebra homomorphism:
  \begin{enumerate}
    \item $\phi(1) = 1$: indeed $\phi\left(\Delta_{\id,1}\right) = \id$.
    \item $\phi(xy) = \phi(x)\phi(y)$: indeed, let $\sigma,\tau\in\gal(K/F)$ and $c,d\in K$, we need to check that $\phi\left(\Delta_{\sigma,c}\Delta_{\tau,d}\right) = \phi\left(\Delta_{\sigma,c}\right)\phi\left(\Delta_{\tau,d}\right)$. The left hand side is equal to
          \[\phi\left(\Delta_{\sigma\tau,c\sigma(d)}\right) = \phi\left(c\sigma(d)\,\cdot\, \Delta_{\sigma\tau,1}\right) = c\sigma(d)\,\cdot\, \sigma\tau;\]
          and the right hand side is equal to
          \[
        \phi\left(c\cdot \Delta_{\sigma,1}\right)\phi\left(d\cdot \Delta_{\tau, 1}\right) = (c \cdot \sigma) (d \cdot \tau).
          \]
          For any $x \in K$, applying left hand side to $x$ will result in $c\sigma(d)\sigma(\tau(x))$ while right hand side will result in $c\sigma(d\tau(x))$, hence both sides are equal.
  \end{enumerate}
  Hence, $\phi$ is an $F$-algebra isomorphism by~\cref{cor:alghom-bijective-of-dim-eq}; that is we have $\cross_{1} \cong \End_{F}K \cong\mat_{\dim_{F}K}(F)$. We conclude that $\cross_{1}$ is Brauer equivalent to $F$ and consequently $\HH^{2}(1) = 1$.
\end{proof}


\begin{corollary}
  The function $\HH^{2}:\br(K/F) \to \HH^{2}\left(\gal(K/F), K^{\star}\right)$ preserves one, that is $\HH^{2}(1) = 1$.
  \leanok
  \lean{RelativeBrGroup.toSndAddMonoidHom}
\end{corollary}
\begin{proof}
  Apply $\cross$ then use~\cref{lem:relative-br-snd-inverse-2} and~\cref{thm:map_one}.
\end{proof}

\subsubsection{$\cross_{\mfa\mfb}\sim_{\br}\cross_{\mfa}\otimes_{F}\cross_{\mfb}$}
The argument in this section is more complicated, because, unlike before, the left hand side and the right hand side are not isomorphic as $F$-algebras --- left hand side has $F$-dimension $\left(\dim_{F}K\right)$ while the right hand side has $F$-dimension $\left(\dim_{F}K\right)^{4}$. Let $\mfa$ and $\mfb$ be two 2-cocycles in $\mathcal{B}^{2}\left(\gal(K/F), K^{\star}\right)$, we denote $\mfc$ to be the 2-cocycle $\mfa \mfb$. Intuitively, $\cross_{\mfa} \ox_{F} \cross_{\mfb}$ is too ``big'', to address this issue we introduce a quotient module.

\begin{construction}[$M$]
  Consider the quotient module
  \[M := {}^{\cross_{\mfa}\ox_{F}\cross_{\mfb}}/_{\left\langle (k\cdot a)\ox b - a \ox (k\cdot b) | k \in K, a \in \cross_{\mfa}, b \in \cross_{\mfb} \right\rangle}.\]
  For any $a' \in \cross_{\mfa}$ and $b' \in \cross_{\mfb}$, we can define an $F$-linear map $M \to M$ by descending the $F$-linear map
$\cross_{\mfa}\ox_{F}\cross_{\mfb} \to \cross_{\mfa}\ox \cross_{\mfb}$
  \[
      a\ox b \mapsto aa' \ox bb';
    \]
    we need to check that for all $k \in K, a \in \cross_{\mfa}, b \in \cross_{\mfb}$, the image of $(k\cdot a)\ox b - a\ox(k\cdot b)$ is in $\left\langle (k\cdot a)\ox b - a \ox (k\cdot b) | k \in K, a \in \cross_{\mfa}, b \in \cross_{\mfb} \right\rangle$:
    the image is $(k\,\cdot\, aa')\ox b - a\ox(k\,\cdot\, bb')$ which is in the generating set with $k \in K, aa' \in\cross_{\mfa}, $ and $bb'\in\cross_{\mfb}$.
  This map is in fact $F$-linear in both $a'$ and $b'$, hence we have an $F$-bilinear map $\cross_{\mfa}\ox\cross_{\mfb} \to M \to M$.
  This gives $M$ a $\left(\cross_{\mfa}\ox_{F}\cross_{\mfb}\right)^{\opp}$-module structure given by
  \[
    (a' \ox b') \cdot [a\ox b] = aa' \ox bb'
  \]
  for any $a,a' \in \cross_{\mfa}$ and $b, b' \in\cross_{\mfb}$. All of the module axioms in this case follows from $F$-bilinearity.


  For any $c \in \cross_{\mfc}$, we can define another $F$-linear map $M \to M$ by descending the $F$-linear map $\cross_{\mfa}\ox_{F}\cross_{\mfb} \to \cross_{\mfa}\ox \cross_{\mfb}$
  \[
      a \ox b \mapsto \sum_{\sigma\in\gal(K/F)} \Delta^{\mfa}_{\sigma,c(\sigma)}a \, \ox \, \Delta^{\mfb}_{\sigma, 1}b;
    \]
    we need check that for all $k \in K, a \in\cross_{\mfa}, b \in \cross_{\mfb}$, the image of $(k\cdot a)\ox b - a \ox (k\cdot b)$ is in
    $\left\langle (k\cdot a)\ox b - a \ox (k\cdot b) | k \in K, a \in \cross_{\mfa}, b \in \cross_{\mfb} \right\rangle$: by~\cref{lem:cross-product-basis-conj} the image is
    \[
\begin{aligned}
      & \sum_{\sigma\in\gal(K/F)}
      \Delta^{\mfa}_{\sigma, c(\sigma)}(k \cdot a) \,\ox\, \Delta^{\mfb}_{\sigma, 1}b
      -
        \Delta^{\mfa}_{\sigma, c(\sigma)}a \,\ox\, \Delta^{\mfb}_{\sigma, 1}(k \cdot b) \\
  = & \sum_{\sigma\in\gal(K/F)}
      \Delta^{\mfa}_{\sigma,c(\sigma)}\iota_{\mfa}(k)a \,\ox\, \Delta^{\mfb}_{\sigma,1}b
      -
      \Delta^{\mfa}_{\sigma,c(\sigma)} \,\ox\, \Delta^{\mfb}_{\sigma, 1}\iota_{\mfb}(k)b \\
  =& \sum_{\sigma\in\gal(K/F)}
     \sigma(k)\cdot\Delta^{\mfa}_{\sigma, c(\sigma)}a \,\ox\, \Delta^{\mfb}_{\sigma, 1}b
     - \Delta^{\mfa}_{\sigma, c(\sigma)} \,\ox\, \sigma(k)\cdot\Delta^{\mfb}_{\sigma,1}b,
  \end{aligned}
\]
which is in $\left\langle (k\cdot a)\ox b - a \ox (k\cdot b) | k \in K, a \in \cross_{\mfa}, b \in \cross_{\mfb} \right\rangle$ because for each $\sigma \in \gal(K/F)$, the summand is in the generating set with $\sigma(k) \in K, \Delta^{\mfa}_{\sigma, c(\sigma)}a \in \cross_{\mfa}$ and $\Delta^{\mfb}_{\sigma, 1}b \in \cross_{\mfb}$. This map is in fact $F$-linear in $c$, therefore we have an $F$-bilinear map $\cross_{\mfc}\to M \to M$. (In the above calculation ``$\ox$'' symbol has low precedence.) This gives $M$ a $\cross_{\mfc}$-module structure given by
\[
  c \cdot [a \ox b] = \left[\sum_{\sigma\in\gal(K/F)}\Delta^{\mfa}_{\sigma, c(\sigma)}a \ox \Delta^{\mfb}_{\sigma, 1}b\right].
\]
In particular, if $c$ is of the form $k \cdot \Delta^{\mfc}_{\tau, 1}$, then $\left(k\cdot \Delta^{\mfc}_{\tau, 1}\right)\cdot [a \ox b]$ is equal to $\left[\left(k\cdot \Delta^{\mfa}_{\tau,1}\right)a \ox \Delta^{\mfb}_{\tau, 1} b\right]$ because $\Delta_{\tau, 1}^{\mfc}(\sigma) = 0$ for all $\sigma \ne \tau$. Two of the module axioms need more than $F$-bilinearity:
\begin{itemize}
  \item $c = 1$: note that $c = 1 = \mfb(\id,\id)^{-1}\mfa(\id,\id)^{-1}\,\cdot\,\Delta^{\mfc}_{\id, 1}$, hence
        \[
        \begin{aligned}
          1 \cdot [a\ox b]
          &= [\mfb(\id,\id)^{-1}\mfa(\id,\id)^{-1}\Delta^{\mfa}_{\id,1}a \ox \Delta^{\mfb}_{\id, 1}b] \\
          &= [\Delta^{\mfa}_{\id,\mfa(\id,\id)^{-1}}a \ox \Delta^{\mfb}_{\id,\mfb(\id,\id)^{-1}}b] \\
          &= [a\ox b].
        \end{aligned}
        \]
  \item $c_{1}c_{2}\cdot [a \ox b] = c_{1}\cdot c_{2}\cdot [a\ox b]$: assume $c_{1} = k_{1}\cdot\Delta^{\mfc}_{\tau_{1}, 1}$ and $c_{2}=k_{2}\cdot\Delta^{\mfc}_{\tau_{2},1}$. Then $c_{1}c_{2} = k_{1}\tau_{1}\left(k_{2}\right)\cdot \Delta^{\mfc}_{\tau_{1}\tau_{2},\mfc(\tau_{1},\tau_{2})} = k_{1}\tau_{1}\left(k_{2}\right)\mfa(\tau_{1},\tau_{2})\mfb(\tau_{1},\tau_{2})\cdot \Delta^{\mfc}_{\tau_{1}\tau_{2},1}$.
        Therefore, the left hand side is equal to
        \[
        \begin{aligned}
          & \left[\left(k_{1}\tau_{1}(k_{2})\mfa(\tau_{1},\tau_{2})\mfb(\tau_{1},\tau_{2})\cdot \Delta^{\mfa}_{\tau_{1}\tau_{2}, 1}\right)a \ox \Delta^{\mfb}_{\tau_{1}\tau_{2},1}b\right] \\
          =& \left[
             k_{1}\tau_{1}(k_{2})\mfa(\tau_{1},\tau_{2})\cdot \Delta^{\mfa}_{\tau_{1}\tau_{2},1} a \ox \mfb(\tau_{1},\tau_{2})\Delta^{\mfb}_{\tau_{1}\tau_{2}, 1}b
             \right]\\
          =& \left[
             \Delta^{\mfa}_{\tau_{1},k_{1}}\Delta^{\mfa}_{\tau_{2},k_{2}} a \ox
             \Delta^{\mfb}_{\tau_{1},1}\Delta^{\mfb}_{\tau_{2},1}b
             \right];
        \end{aligned}
        \]
        and the right hand side is also equal to
        \[
        \begin{aligned}
        &\left(k_{1}\cdot \Delta^{\mfc}_{\tau_{1}, 1}\right)
        \left[
        k_{2}\cdot\Delta^{\mfa}_{\tau_{2},1}a \ox \Delta^{\mfb}_{\tau_{2}, 1} b
          \right] \\
          =& \left[
             \left(k_{1}\cdot \Delta^{\mfa}_{\tau_{1}, 1}\right)\left(k_{2}\cdot\Delta^{\mfa}_{\tau_{2}, 1}\right)a \ox
             \Delta^{\mfb}_{\tau_{1},1}\Delta^{\mfb}_{\tau_{2},1}b
             \right]\\
          =&\left[
           k_{1}\tau_{1}(k_{2})\cdot \Delta^{\mfa}_{\tau_{1}\tau_{2},\mfa(\tau_{1},\tau_{2})}a \ox \Delta^{\mfb}_{\tau_{1}, 1}\Delta^{\mfb}_{\tau_{2},1}b
             \right]\\
          =&\left[
             \left(k_{1}\cdot \Delta^{\mfa}_{\tau_{1}, 1}\right)\left(k_{2}\cdot\Delta^{\mfa}_{\tau_{2}, 1}\right)a \ox
             \Delta^{\mfb}_{\tau_{1},1}\Delta^{\mfb}_{\tau_{2},1}b
             \right].
          \end{aligned}
        \]
\end{itemize}

Expanding everything out and checking on the basic elements, we see that for any $x \in \left(\cross_{\mfa}\ox_{F}\cross_{\mfb}\right)^{\opp}$, $y \in \cross_{\mfc}$ and $z \in M$, $x \cdot y \cdot z = y \cdot x \cdot z$. In another word, we gave $M$ a $\left(\cross_{\mfc}, \cross_{\mfa} \ox_{F} \cross_{\mfb}\right)$-bimodule structure.
  \leanok
  \lean{map_mul_proof.M, map_mul_proof.Aox_FB_op_tmul_smul_mk_tmul, map_mul_proof.C_smul_calc}
\end{construction}

%%% Local Variables:
%%% mode: LaTeX
%%% TeX-master: "../print"
%%% End:
