% \setcounter{chapter}{-1}
\section{A Collection of Useful Lemmas}\label{sec:unclassified}

In section, we collect some lemmas that does not really belong to anywhere.

\subsection{Tensor Product}

\begin{lemma}\label{lem:expand-tensor-in-basis}
  Let $M$ and $N$ be $R$-modules such that $\mathcal{C}_{i\in\iota}$ is a basis for $N$, then every elements of $x \in M \otimes_{R} N$ can be uniquely written as $\sum_{i\in\iota}m_{i}\otimes \mathcal{C}_{i}$ where only finitely many $m_{i}$'s are non-zero
\end{lemma}

\begin{proof}
  Given the basis $\mathcal{C}$, we have $R$-linear isomorphism $N \cong\bigoplus_{i\in\iota}R$, hence $M\otimes_{R}N \cong \bigoplus_{i\in\iota}(M\otimes_{R}R)\cong\bigoplus_{i\in\iota}M$ as $R$-modules.
\end{proof}
By switching $M$ and $N$, the symmetric statement goes without saying.

% \noindent\rule{\textwidth}{0.2pt}

\begin{lemma}

  Let $K$ be a field, $M$ and $N$ be flat $K$-modules. Suppose $p \subseteq M$ and $q \subseteq N$ are $K$-submodules, then $(p \otimes_{K} N) \sqcap (M \otimes_{K} q) = p \otimes_{K} q$ as $K$-submodules.
\end{lemma}

\begin{proof}
  The hard direction is to show $(p \otimes_{R} N) \sqcap (M \otimes_{R} q) \le p \otimes_{R} q$. Consider the following diagram:
  \begin{center}
    \begin{tikzcd}
      p \otimes_{K} q \ar[r, "u"] \ar[d, "\alpha"] & M \otimes_{K} q \ar[r, "v"] \ar[d, "\beta"] & {}^{M}/_{p} \otimes_{K} q \ar[d, "\gamma"] \\
      p \otimes_{K} N \ar[r, "u'"] & M \otimes_{K} N \ar[r, "v'"] & ^{M}/_{p} \otimes_{K} N
    \end{tikzcd}
  \end{center}
  Since $^{M}/_{p}$ is flat, $\gamma$ is injective.
  Let $z \in (p \otimes_{R} N) \sqcap (M \otimes_{R} q) = \im \beta \sqcap \im u'$. By abusing notation, replace $z$ with some elements of $M \otimes_{K} q$ and continue with $\beta(z)\in\im \beta \sqcap \im u'$. Since $v'(\beta(z))=\gamma(v(z))$ and that $\beta(z)\in \im u'$, we conclude that $\gamma(v(z))=0$, that is $z\in\ker v=\im u$. We abuse notation again, let $z \in p \otimes_{K} q$, we need to show $\beta(u(z))\in\im \beta \sqcap \im u'$, but $\beta\circ u=u'\circ\alpha$, we finish the proof.
\end{proof}

\subsection{Centralizer and Center}
Let $R$ be a commutative ring and $A$, $B$ be two $R$-algebras. We denote centralizer of $S\subseteq A$ by $C_{A}S$ and centre of $A$ by $Z(A)$.

\begin{lemma}
  Let $S, T$ be two subalgebras of $A$, then $C_{A}(S\sqcup T)=C_{A}(S)\sqcap C_{A}(T)$.
  \leanok
  \lean{Subalgebra.centralizer_sup}
\end{lemma}
This lemma can be generalized to centralizers of arbitrary supremum of subalgebras.

\begin{lemma}
  If we assume $B$ is free as $R$-module, then $C_{A\otimes_{R}B}\left(\im\left(A\to A\otimes_{R}B\right)\right)$ is $Z(A) \otimes_{R} B$
\end{lemma}
A symmetric statement goes without saying.
\begin{proof}
  Let $w\in C_{A\otimes_{R}B}\left(\im\left(A\to A\otimes_{R}B\right)\right)$. Since $B$ is free, we choose an arbitrary basis $\mathcal{B}$; by~\cref{lem:expand-tensor-in-basis}, we write $w = \sum_{i}m_{i}\otimes_{K}\mathcal{B}_{i}$. It is sufficient to show that $m_{i}\in Z(A)$ for all $i$. Let $a \in A$, we need to show that $m_{i}\cdot a = a \cdot m_{i}$. Since $w$ is in the centralizer, $w \cdot (a\otimes 1) = (a\otimes 1)\cdot w$. Hence we have $\sum_{i}(a\cdot m_{i})\otimes\mathcal{B}_{i}=\sum_{i}(m_{i}\cdot a)\otimes\mathcal{B}_{i}$. By the uniqueness of~\cref{lem:expand-tensor-in-basis}, we conclude $a\cdot m_{i}=m_{i}\cdot a$.
\end{proof}

\begin{remark}
  Since $\im\left(R\otimes_{R}B\to A\otimes_{R}B\right)=\im\left(A\to A\otimes_{R}B\right)$, we conclude its centralizer in $A\otimes_{R}B$ is $Z(A)\otimes_{R}B$ as well.
\end{remark}

\begin{lemma}
  \label{lem:center-tensor}
  Assume $R$ is a field. The centre of $A\otimes_{R} B$ is $Z\left(A\right)\otimes_{R}Z\left(B\right)$.
  \leanok
  \lean{IsCentralSimple.center_tensorProduct}
\end{lemma}
\begin{proof}
 From previous lemmas, we know that $Z\left(A\otimes_{R}B\right)$ is equal to $Z\left(A\right)\otimes_{R}B \sqcap A\otimes_{R}Z\left(B\right)$ which is $Z\left(A\right)\otimes_{R}Z\left(B\right)$
\end{proof}

%%% Local Variables:
%%% mode: LaTeX
%%% TeX-master: "../print"
%%% End:
