% \setcounter{chapter}{-1}
\section{A Collection of Useful Lemmas}\label{sec:unclassified}

In section, we collect some lemmas that does not really belong to anywhere.

\subsection{Tensor Product}

\begin{lemma}\label{lem:expand-tensor-in-basis}
  Let $M$ and $N$ be $R$-modules such that $\mathcal{C}_{i\in\iota}$ is a basis
  for $N$, then every elements of $x \in M \otimes_{R} N$ can be uniquely
  written as $\sum_{i\in\iota}m_{i}\otimes \mathcal{C}_{i}$ where only finitely
  many $m_{i}$'s are non-zero
\end{lemma}

\begin{proof}
  Given the basis $\mathcal{C}$, we have $R$-linear isomorphism
  $N \cong\bigoplus_{i\in\iota}R$, hence
  $M\otimes_{R}N \cong \bigoplus_{i\in\iota}(M\otimes_{R}R)\cong\bigoplus_{i\in\iota}M$
  as $R$-modules.
\end{proof}
By switching $M$ and $N$, the symmetric statement goes without saying.

% \noindent\rule{\textwidth}{0.2pt}

\begin{lemma}
  \label{lem:submodule-tensor-submodule}
  Let $K$ be a field, $M$ and $N$ be flat $K$-modules. Suppose $p \subseteq M$
  and $q \subseteq N$ are $K$-submodules, then
  $(p \otimes_{K} N) \sqcap (M \otimes_{K} q) = p \otimes_{K} q$ as
  $K$-submodules.
\end{lemma}

\begin{proof}
  The hard direction is to show
  $(p \otimes_{R} N) \sqcap (M \otimes_{R} q) \le p \otimes_{R} q$. Consider the
  following diagram:
  \begin{center}
    \begin{tikzcd}
      p \otimes_{K} q \ar[r, "u"] \ar[d, "\alpha"] & M \otimes_{K} q \ar[r, "v"] \ar[d, "\beta"] & {}^{M}/_{p} \otimes_{K} q \ar[d, "\gamma"] \\
      p \otimes_{K} N \ar[r, "u'"] & M \otimes_{K} N \ar[r, "v'"] & ^{M}/_{p}
      \otimes_{K} N
    \end{tikzcd}
  \end{center}
  Since $^{M}/_{p}$ is flat, $\gamma$ is injective. Let
  $z \in (p \otimes_{R} N) \sqcap (M \otimes_{R} q) = \im \beta \sqcap \im u'$.
  By abusing notation, replace $z$ with some elements of $M \otimes_{K} q$ and
  continue with $\beta(z)\in\im \beta \sqcap \im u'$. Since
  $v'(\beta(z))=\gamma(v(z))$ and that $\beta(z)\in \im u'$, we conclude that
  $\gamma(v(z))=0$, that is $z\in\ker v=\im u$. We abuse notation again, let
  $z \in p \otimes_{K} q$, we need to show
  $\beta(u(z))\in\im \beta \sqcap \im u'$, but $\beta\circ u=u'\circ\alpha$, we
  finish the proof.
\end{proof}

\subsection{Centralizer and Center}
Let $R$ be a commutative ring and $A$, $B$ be two $R$-algebras. We denote
centralizer of $S\subseteq A$ by $C_{A}S$ and centre of $A$ by $Z(A)$.

\begin{lemma}\label{lem:sup-centralizer}
  Let $S, T$ be two subalgebras of $A$, then
  $C_{A}(S\sqcup T)=C_{A}(S)\sqcap C_{A}(T)$. \leanok
  \lean{Subalgebra.centralizer_sup}
\end{lemma}
This lemma can be generalized to centralizers of arbitrary supremum of
subalgebras.

\begin{lemma}\label{lem:centralizer-inclusion-tensor}
  If we assume $B$ is free as $R$-module, then for any $R$-subalgebra $S$, we
  have that $C_{A\otimes_{R}B}\left(\im\left(S\to A\otimes_{R}B\right)\right)$
  is $C_{A}(S) \otimes_{R} B$
  \leanok
  \lean{centralizer_inclusionRight}
  % \uses{lem:expand-tensor-in-basis}
\end{lemma}
A symmetric statement goes without saying.
\begin{proof}
  Let $w\in C_{A\otimes_{R}B}\left(\im\left(S\to A\otimes_{R}B\right)\right)$.
  Since $B$ is free, we choose an arbitrary basis $\mathcal{B}$;
  by~\cref{lem:expand-tensor-in-basis}, we write
  $w = \sum_{i}m_{i}\otimes_{K}\mathcal{B}_{i}$. It is sufficient to show that
  $m_{i}\in C_{A}(S)$ for all $i$. Let $a \in S$, we need to show that
  $m_{i}\cdot a = a \cdot m_{i}$. Since $w$ is in the centralizer,
  $w \cdot (a\otimes 1) = (a\otimes 1)\cdot w$. Hence we have
  $\sum_{i}(a\cdot m_{i})\otimes\mathcal{B}_{i}=\sum_{i}(m_{i}\cdot a)\otimes\mathcal{B}_{i}$.
  By the uniqueness of~\cref{lem:expand-tensor-in-basis}, we conclude
  $a\cdot m_{i}=m_{i}\cdot a$.
\end{proof}

\begin{remark}
  A useful special case is when $S=A$, then since $C_{A}(A)=Z(A)$, we have
  $C_{A\otimes_{R}B}\left(\im\left(A\to A\otimes_{R} B\right)\right)$ is equal
  to $Z(A)\otimes_{R} B$. Since
  $\im\left(R\otimes_{R}B\to A\otimes_{R}B\right)=\im\left(A\to A\otimes_{R}B\right)$,
  we conclude its centralizer in $A\otimes_{R}B$ is $Z(A)\otimes_{R}B$.
\end{remark}

\begin{corollary}\label{cor:centralizer-tensor-centralizer}
  Assume $R$ is a field. Let $S$ and $T$ be $R$-subalgebras of $A$ and $B$
  respectively. Then $C_{A\otimes_{R}B}\left(S\otimes_{R}T\right)$ is equal to
  $C_{A}(S)\otimes_{R}C_{B}(T)$
  \leanok
  \lean{centralizer_tensor_centralizer}
  % \uses{lem:submodule-tensor-submodule}
\end{corollary}
\begin{proof}
  From~\cref{lem:submodule-tensor-submodule}, $C_{A}(S)\otimes_{R}C_{B}(T)$ is
  equal to
  $\left(C_{A}(S)\otimes_{R} B\right)\sqcap \left(A\otimes_{R}C_{B}(T)\right)$.
  The left hand side $C_{A}(S)\otimes_{R} B$ is equal to
  $C_{A\otimes_{R}B}\left(\im\left(S\to A\otimes_{R}B\right)\right)$ and the
  right hand side is equal to
  $C_{A\otimes_{R}B}\left(\im\left(T\to A\otimes_{R}B\right)\right)$. Hence
  by~\cref{lem:sup-centralizer}, their intersection is equal to
  \[C_{A\otimes_{R}B}\left(\im\left(S\to A\otimes_{R} B\right)\sqcup \im\left(T\to A\otimes_{R}B\right)\right)\]
  This is precisely $C_{A\otimes_{R}B}\left(S\otimes_{R}T\right)$.
\end{proof}

\begin{corollary}
  \label{cor:center-tensor-center}
  Assume $R$ is a field. The centre of $A\otimes_{R} B$ is
  $Z\left(A\right)\otimes_{R}Z\left(B\right)$. 
  \leanok
  \lean{IsCentralSimple.center_tensorProduct}
  % \uses{cor:centralizer-tensor-centralizer}
\end{corollary}
\begin{proof}
  Special case of~\cref{cor:centralizer-tensor-centralizer}.
\end{proof}

\subsection{Some Isomorphisms}

\begin{construction}
  \label{con:self-tensor-opp-iso-end}
  Let $R$ be a commutative ring and $A$ an $R$-algebra. Then we
  have an $R$-algebra homomorphism $A\otimes_{R}A^{\opp} \cong \End_{R}A$ given
  by $a\otimes 1 \mapsto (a\cdot\bullet)$ and
  $1\otimes a \mapsto (\bullet\cdot a)$. When $R$ is a field and $A$ is a finite dimensional central simple algebra, this morphism is an isomorphism by~\cref{cor:alghom-bijective-of-dim-eq}.
\end{construction}

\begin{construction}
  \label{con:end-vec-iso-matrix}
  Let $A$ be an $R$-algebra and $M$ an $A$-module. We have isomorphism
  $\End_{A}\left(M^{n}\right)\cong \Mat_{n}\left(\End_{A}M\right)$ as
  $R$-algebras. For any $f \in \End_{A}\left(M^{n}\right)$, we define a matrix
  $M$ whose $(i,j)$-th entry is
  \[
    x \mapsto f\left(0,\dots,x,\dots,0\right)_{i},
  \]
  where $x$ is at the $j$-th position. On the other hand, if
  $M \in \Mat_{n}\left(\End_{A}M\right)$, we define an $A$-linear map
  $f : M^{n}\to M^{n}$ by
  \[
    v \mapsto \left(\sum_{j}M_{ij}v_{j}\right)_{i}.
  \]
  \leanok
  \lean{endVecAlgEquivMatrixEnd}
\end{construction}

\begin{construction}
  \label{con:matrix-matrix}
  Let $A$ be an $R$-algebra. Then $\Mat_{m}\left(\Mat_{n}(A)\right) \cong \Mat_{mn}(A)$.
  The trick is to think $\Mat_{m}A$ as $\{0,\dots,m-1\} \times \{0,\dots,m-1\} \to A$. Since the indexing set $\{0,\dots,mn-1\}$ bijects with $\left(\{0,\dots,m-1\}\times\{0,\dots,n-1\}\right)$, the isomorphism is just function currying, function uncurrying, precomposing and postcomposing bijections.
  \leanok
  \lean{Matrix.comp}
\end{construction}

\begin{construction}
  \label{con:matrix-tensor-matrix}
  Let $A, B$ be $R$-algebras. Then $\Mat_{mn}\left(A\otimes_{R}B\right)\cong\Mat_{m}(A)\otimes_{R}\Mat_{n}B$ as $K$-algebras. We first construct $R$-algebra isomorphism $A \otimes_{R}\Mat_{n}(R) \cong \Mat_{n}(A)$:
  \[
    \begin{aligned}
      & a \otimes 1 \mapsto \diag{a}~\text{and}~ 1\otimes (m_{ij}) \mapsto (m_{ij})\\
      & \sum_{i,j} m_{ij}\otimes \delta_{ij}\mapsfrom (m_{ij}),
    \end{aligned}
  \]
  where $\diag$ is the diagonal matrix and $\delta_{ij}$ the matrix whose only non-zero entry is at $(i,j)$-th and is equal to $1$.
  Thus $\Mat_{m}(A)\otimes_{R}\Mat_{n}(B)\cong \left(A\otimes_{R}B\right)\otimes_{R} \left(\Mat_{m}(R)\otimes_{R} \Mat_{n}(R)\right)$ as $R$-algebra.
  The Kronecker product gives us an $R$-algebra map $\Mat_{m}(R)\otimes_{R}\Mat_{n}(R) \to \Mat_{mn}(R)$. We want this map to be an isomorphism. By~\cref{lem:simple-ring-iff}, we only need to prove it to be surjective: for all $\delta_{ij} \in \Mat_{mn}(R)$, we interpret $\Mat_{mn}(R)$ as a function $\{0,\dots,m-1\}\times\{0,\dots,n-1\}\to R$, then $\delta_{ij}$ is the image of $\delta_{ab}\otimes \delta_{cd} \in \Mat_{m}(R)\otimes_{R}\Mat_{n}(R)$ where $i = (a, c)$ and $j = (b,d)$. Combine everything together, we see $\Mat_{mn}\left(A\otimes_{R}B\right)$ is isomorphic to $\Mat_{mn}\left(A\otimes_{R} B\right)$ as $R$-algebras.
  \leanok
  \lean{BrauerGroup.matrix_eqv, BrauerGroup.kroneckerMatrixTensor'}
  % \uses{lem:simple-ring-iff}
\end{construction}

%%% Local Variables:
%%% mode: LaTeX
%%% TeX-master: "../print"
%%% End:
