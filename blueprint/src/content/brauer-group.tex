\chapter{Brauer Group}\label{cha:brauer-group}

\section{Construction of Brauer Group}
Let $K$ be a field. We denote the class of finite dimensional central simple $K$-algebras as $\CSA_{K}$. When $K$ is clear, we drop the subscript.

\begin{remark}
  By~\cref{lem:tensor-central} and~\cref{thm:tensor-csa}, $\CSA$ is closed under tensor product, that is if $A, B\in\CSA$, we have $A\otimes_{K} B\in\CSA$ as well.
\end{remark}

\begin{definition}[Brauer Equivalence]
  For any two $A, B\in\CSA$, we say $A$ and $B$ are Brauer equivalent, when there exists $m, n \in \mathbb{N}_{\ge0}$ such that $\Mat_{m}(A)\cong \Mat_{n}B$ as $K$-algebras. We denote this relation as $A\sim_{\Br_{K}} B$, when $K$ is clear, we drop the subscript.
\end{definition}

\begin{remark}
  Isomorphic $K$-algebras are Brauer equivalent.
\end{remark}

\begin{lemma}
  $\sim_{\Br}$ is reflexive.
  \leanok
  \lean{IsBrauerEquivalent.refl}
\end{lemma}
\begin{proof}
  Indeed, $A \cong \Mat_{1}(A)$ as $K$-algerbas.
\end{proof}

\begin{lemma}
  $\sim_{\Br}$ is symmetric.
  \leanok
  \lean{IsBrauerEquivalent.symm}
\end{lemma}
\begin{proof}
  Indeed, just exchange $m$ and $n$.
\end{proof}

\begin{lemma}
  $\sim_{\Br}$ is transitive.
  \leanok
  \lean{IsBrauerEquivalent.trans}
\end{lemma}
\begin{proof}
  Let $A\sim_{\Br}B$ and $B\sim_{\Br}C$; that is for some $m,n,p, q\in\mathbb{N}_{\ge0}$ we have $\Mat_{n}(A)\cong\Mat_{m}(B)$ and $\Mat_{p}(B)\cong \Mat_{q}(C)$ as $K$-algebras. Hence, from~\cref{con:matrix-matrix}, we have the following:
  \[
    \begin{aligned}
      \Mat_{np}(A)&\cong \Mat_{p}\left(\Mat_{n}(A)\right)\cong\Mat_{p}\left(\Mat_{m}(B)\right)\\
                  &\cong\Mat_{mp}(B)\cong\Mat_{m}\left(\Mat_{p}(B)\right)\\
      &\cong\Mat_{m}\left(\Mat_{q}(C)\right)\cong\Mat_{mq}(C).
    \end{aligned}
  \]
  In another word, $A\sim_{\Br}C$.
\end{proof}

Hence $\sim_{\Br}$ is really an equivalence relation, we denote the quotient ${}^{\CSA}/_{\sim_{\Br}}$ as $\Br(K)$.

\begin{lemma}\label{lem:br-mul-wd}
  $(\bullet\otimes_{K}\bullet):\CSA\times\CSA \to \CSA$ descends to a function on $\Br(K)$.
  \leanok
  \lean{BrauerGroup.eqv_tensor_eqv}
\end{lemma}
\begin{proof}
  We need to prove that for all $A, B, C, D \in \CSA$ such that $A\sim_{\Br} B$ and $C\sim_{\Br}D$, $A\otimes_{R}C \sim_{\Br} D\otimes_{R}D$ as well.
  Suppose $\Mat_{m}(A)\cong \Mat_{n}(B)$ as $K$-algebras and $\Mat_{p}(C)\cong\Mat_{q}(D)$, by~\cref{con:matrix-tensor-matrix}, we have
  \[
    \begin{aligned}
      \Mat_{mp}\left(A\otimes_{R} C\right)&\cong \Mat_{m}(A)\otimes_{R}\Mat_{p}(C) \\
                                          &\cong \Mat_{n}(B)\otimes_{R}\Mat_{q}(D)\\
      &\cong \Mat_{nq}\left(B\otimes_{R}D\right).
    \end{aligned}
  \]

\end{proof}

\begin{construction}[Brauer Group]\label{con:br}
  $\Br(K)$ forms a group under $[A]_{\sim_{\Br}}\cdot[B]_{\sim_{\Br}}=[A\otimes_{K}B]_{\sim_{\Br}}$ with neutral element $[K]_{\sim_{\Br}}$ where $A, B\in\CSA$ and $[A]_{\sim_{Br}}^{-1}=[A^{\opp}]_{\sim_{\Br}}$. We need to prove the following properties:
  \begin{enumerate}
    \item associativity: for all $A, B, C\in\CSA$, $[A]_{\sim_{\Br}}\cdot\left([B]_{\sim_{\Br}}\cdot[C]_{\sim_{\Br}}\right)=\left([A]_{\sim_{\Br}}\cdot[B]_{\sim_{\Br}}\right)\cdot [C]_{\sim_{\Br}}$ because $A\ox_{R}\left(B\ox_{R}C\right)\cong \left(A\ox_{R}B\right)\ox_{R}C$ as $K$-algebras.
    \item neutral element: for all $A\in \CSA$, $[K]_{\sim_{\Br}}\cdot[A]_{\sim_{\Br}}=[A]_{\sim_{\Br}}=[A]_{\sim_{\Br}}\cdot [K]_{\sim_{\Br}}$. Since $[K]_{\sim_{\Br}}\cdot[A]_{\sim_{\Br}}=[K\ox_{K} A]_{\sim_{\Br}}$, in~\cref{con:matrix-tensor-matrix}, we see that $\Mat_{n}(A)\cong A\ox_{K}\Mat_{n}(K)$, by~\cref{lem:br-mul-wd}, $A\ox_{K}\Mat_{n}(K)$ is Brauer equivalent to $A\ox_{K} K$ since $K\sim_{\Br}\Mat_{n}(K)$.
    \item cancellation: for all $A \in \CSA$, we need $[A]_{\sim_{\Br}}\cdot[A^{\opp}]_{\sim_{\Br}}$, that is we want $A\ox_{K}A^{\opp}\sim_{\Br} K$. By~\cref{con:self-tensor-opp-iso-end}, we have $A\ox_{K}A^{\opp}\cong \End_{K}A$ which is isomorphic to $\Mat_{\dim_{K}A}(K)$ as $K$-algebras.
  \end{enumerate}
  \leanok
  \lean{BrauerGroup.Bruaer_Group}
\end{construction}

\begin{theorem}
  If $K$ is algebraically closed, $\Br(K)$ is trivial; in particular $\Mat_{n}(\mathbb{C})$ is trivial.
\end{theorem}
\begin{proof}
  We need to show that every $A\in\CSA$ is isomorphic to $\Mat_{n}(K)$ for some $K$ when $K$ is algebraically closed. Indeed, by~\cref{thm:wed-artin-algebra}, $A \cong \Mat_{n}(D)$ for some division algebra $D$ and $n\in\nat_{\ge0}$. Since $K$ is algebraically closed and $D$ is an integral domain and finite dimensional, the structure morphism $\rho: K\to D$ is a isomorphism; therefore $A\cong \Mat_{n}(K)$.
\end{proof}

\section{Base Change}
In this section let $E/K$ be a field extension. We have seen in~\cref{cor:csa-basechange} that if $A\in\CSA_{K}$ then $E\otimes_{K}A\in\CSA_{E}$; therefore we have a set-theoretic function $\CSA_{K}\to\CSA_{E}$. In this section we prove that this descends to a group homomorphism $\Br(K)\to\Br(E)$. For brevity, if $A\in \CSA_{K}$, we dentoe $E\ox_{K}A$ as $A_{E}$ when this causes no confusion.
\begin{construction}
  \label{con:br-base-change}
  We will construct a series of isomorphisms (either over $K$ or $E$) to arrive at the conclusion that $A\sim_{\Br_{K}}B$ implies $A_{E}\sim_{\Br_{E}}B_{E}$. Assume $m,n\in\nat_{\ge0}$ are such that $\Mat_{m}(A)\cong\Mat_{n}(B)$ are $K$-algebras. Then we do the following calculation: as $E$-algebras
  \[
    \begin{aligned}
      \Mat_{m}\left(A_{E}\right)
      &\cong A_{E}\otimes_{E}\Mat_{m}(E) &\text{see~\cref{con:matrix-tensor-matrix}}\\
      &\cong A_{E}\otimes_{E}\left(E\ox_{K} \Mat_{m}(K)\right) & \text{see}~\dagger\\
      &\cong E \ox_{K}\left(A\ox_{K}\Mat_{m}(K)\right) &\text{see}~\ddagger\\
      &\cong E\ox_{K}\Mat_{m}(A) &\text{see~\cref{con:matrix-tensor-matrix} and}~\dagger\!\!\dagger \\
      \Mat_{n}\left(B_{E}\right) &\cong E\ox_{K}\Mat_{n}(B) &\text{same as the case of}~A\\
      Mat_{m}\left(A_{E}\right)&\cong \Mat_{n}\left(B_{E}\right) &\text{see}~\dagger\!\!\dagger
    \end{aligned}.
  \]

   \noindent$\dagger$: Wee need to check $\Mat_{m}(E)\cong E\otimes_{K}\Mat_{m}K$ as $E$-algebras since~\cref{con:matrix-tensor-matrix} only gives a $K$-algebra isomorphism. If $e \in E$, then its image in $E\ox_{K}\Mat_{m}(K)$ is $e\otimes 1$ and its image in $\Mat_{m}(E)$ is $\diag(e)$ which under the $K$-algebra isomorphism is mapped to $\sum_{ij}\diag(e)_{ij}\cdot \delta_{ij}=e\ox1$.

   \noindent$\ddagger$: This is defined by combining two $E$-algebra homomorphisms
   \[
     A_{E}\to A_{E}\ox_{K}\Mat_{m}(K)\to E\ox_{K}\left(A\ox_{K}\Mat_{m}(K)\right)\]
     and
     \[
       E\ox_{K}\Mat_{m}(K)\to \left(E\ox_{K}\Mat_{m}(K)\right)\ox_{K}A \to E\ox_{K}(A\ox_{K}\Mat_{m}(K)).
    \]
   Since $\left(E\ox_{K}A\right)\ox_{E}\left(E\ox_{K}\Mat_{m}(K)\right)$ is a simple ring, this morphism is automatically injective. It is surjective as well: let $x\in E\ox_{K}\left(A\ox_{K}\Mat_{m}(K)\right)$, without loss of generality, assume $x=e\ox (a\ox\delta_{ij})$ for some $e\in E$, $a\in A$. Then precisely $\left(e\ox a\right)\ox\left(1\ox\delta_{ij}\right)$ is mapped to $x$.

   \noindent$\dagger\!\dagger$: a $K$-algebra isomorphism $A\cong B$ gives an $E$-algebra isomorphism $E\ox_{K}A \cong E\ox_{K} B$.

   Thus we have a well defined function $\Br(K)\to\Br(E)$. We now check that this is a group homomorphism. $[K]_{\sim_{\Br_{K}}}$ is mapped to $[E\ox_{K}K]_{\sim_{\Br_{E}}}$ but $E\ox_{K}K\cong E$ as $E$-algebra. For $A, B\in\CSA_{K}$, we have that $[AB]_{\sim_{\Br_{K}}}$ is mapped to $\left(A\ox_{K}B\right)_{E}\cong A_{E}\ox_{E} B_{E}$ as $E$-algebras; hence $[AB]_{\sim_{\Br_{K}}}$ and $[A]_{\sim_{\Br_{K}}}\cdot[B]_{\sim_{\Br_{K}}}$ have the same image under base change.
   \leanok
   \lean{BrauerGroupHom.BaseChange}
 \end{construction}
 Denote the base change morphism in~\cref{con:br-base-change} as $\Br_{K}^{E}$.
 \begin{lemma}
   $\Br_{K}^{K}$ is identity.
   \leanok
 \end{lemma}
 \begin{proof}
   If $A\in\CSA$, then $A\sim_{\Br}K\ox_{K}A$.
 \end{proof}

 \begin{lemma}
   Consider the tower of field extension $E/F/K$,
   \[
     \Br_{K}^{E} = \Br_{E}^{F}\circ\Br_{K}^{E}.
   \]
 \end{lemma}

 \begin{proof}
   If $A\in\CSA_{K}$, then $E\ox_{F}\left(F\ox_{K}A\right)$ is isomorphic to $E \ox_{F} A$ as $E$-algebras.
   \leanok
   \lean{BrauerGroupHom.baseChange_idem}
 \end{proof}

 \begin{corollary}
   $\Br$ forms a functor from category of field to category of abelian groups.
   \leanok
   \lean{BrauerGroupHom.Br}
 \end{corollary}
 \begin{proof}
   This is exact previous two lemmas packed into categorical language.
 \end{proof}

 \begin{definition}[Relative Brauer Group]
   Let $E/K$ be a field extension, we define the relative Brauer group $\Br(E/K)$ to be the kernel of the base change morphism $\Br_{K}^{E}$.
 \end{definition}

 \section{Good Representative Lemma}

\section{The Second Galois Cohomology}

%%% Local Variables:
%%% mode: LaTeX
%%% TeX-master: "../print"
%%% End:
