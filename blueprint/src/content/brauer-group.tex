\chapter{Brauer Group}\label{cha:brauer-group}

\section{Construction of Brauer Group}
Let $K$ be a field. We denote the class of finite dimensional central simple $K$-algebras as $\CSA_{K}$. When $K$ is clear, we drop the subscript.

\begin{remark}
  By~\cref{lem:tensor-central} and~\cref{thm:tensor-csa}, $\CSA$ is closed under tensor product, that is if $A, B\in\CSA$, we have $A\otimes_{K} B\in\CSA$ as well.
\end{remark}

\begin{definition}[Brauer Equivalence]
  For any two $A, B\in\CSA$, we say $A$ and $B$ are Brauer equivalent, when there exists $m, n \in \mathbb{N}_{\ge0}$ such that $\Mat_{m}(A)\cong \Mat_{n}B$ as $K$-algebras. We denote this relation as $A\sim_{\Br_{K}} B$, when $K$ is clear, we drop the subscript.
\end{definition}

\begin{remark}
  Isomorphic $K$-algebras are Brauer equivalent.
\end{remark}

\begin{lemma}
  $\sim_{\Br}$ is reflexive.
  \leanok
  \lean{IsBrauerEquivalent.refl}
\end{lemma}
\begin{proof}
  Indeed, $A \cong \Mat_{1}(A)$ as $K$-algerbas.
\end{proof}

\begin{lemma}
  $\sim_{\Br}$ is symmetric.
  \leanok
  \lean{IsBrauerEquivalent.symm}
\end{lemma}
\begin{proof}
  Indeed, just exchange $m$ and $n$.
\end{proof}

\begin{lemma}
  $\sim_{\Br}$ is transitive.
  \leanok
  \lean{IsBrauerEquivalent.trans}
\end{lemma}
\begin{proof}
  Let $A\sim_{\Br}B$ and $B\sim_{\Br}C$; that is for some $m,n,p, q\in\mathbb{N}_{\ge0}$ we have $\Mat_{n}(A)\cong\Mat_{m}(B)$ and $\Mat_{p}(B)\cong \Mat_{q}(C)$ as $K$-algebras. Hence, from~\cref{con:matrix-matrix}, we have the following:
  \[
    \begin{aligned}
      \Mat_{np}(A)&\cong \Mat_{p}\left(\Mat_{n}(A)\right)\cong\Mat_{p}\left(\Mat_{m}(B)\right)\\
                  &\cong\Mat_{mp}(B)\cong\Mat_{m}\left(\Mat_{p}(B)\right)\\
      &\cong\Mat_{m}\left(\Mat_{q}(C)\right)\cong\Mat_{mq}(C).
    \end{aligned}
  \]
  In another word, $A\sim_{\Br}C$.
\end{proof}

Hence $\sim_{\Br}$ is really an equivalence relation, we denote the quotient ${}^{\CSA}/_{\sim_{\Br}}$ as $\Br(K)$.

\begin{lemma}\label{lem:br-mul-wd}
  $(\bullet\otimes_{K}\bullet):\CSA\times\CSA \to \CSA$ descends to a function on $\Br(K)$.
  \leanok
  \lean{BrauerGroup.eqv_tensor_eqv}
\end{lemma}
\begin{proof}
  We need to prove that for all $A, B, C, D \in \CSA$ such that $A\sim_{\Br} B$ and $C\sim_{\Br}D$, $A\otimes_{R}C \sim_{\Br} B\otimes_{R}D$ as well.
  Suppose $\Mat_{m}(A)\cong \Mat_{n}(B)$ as $K$-algebras and $\Mat_{p}(C)\cong\Mat_{q}(D)$, by~\cref{con:matrix-tensor-matrix}, we have
  \[
    \begin{aligned}
      \Mat_{mp}\left(A\otimes_{R} C\right)&\cong \Mat_{m}(A)\otimes_{R}\Mat_{p}(C) \\
                                          &\cong \Mat_{n}(B)\otimes_{R}\Mat_{q}(D)\\
      &\cong \Mat_{nq}\left(B\otimes_{R}D\right).
    \end{aligned}
  \]

\end{proof}

\begin{construction}[Brauer Group]\label{con:br}
  $\Br(K)$ forms a group under $[A]_{\sim_{\Br}}\cdot[B]_{\sim_{\Br}}=[A\otimes_{K}B]_{\sim_{\Br}}$ with neutral element $[K]_{\sim_{\Br}}$ where $A, B\in\CSA$ and $[A]_{\sim_{Br}}^{-1}=[A^{\opp}]_{\sim_{\Br}}$. We need to prove the following properties:
  \begin{enumerate}
    \item associativity: for all $A, B, C\in\CSA$, $[A]_{\sim_{\Br}}\cdot\left([B]_{\sim_{\Br}}\cdot[C]_{\sim_{\Br}}\right)=\left([A]_{\sim_{\Br}}\cdot[B]_{\sim_{\Br}}\right)\cdot [C]_{\sim_{\Br}}$ because $A\ox_{R}\left(B\ox_{R}C\right)\cong \left(A\ox_{R}B\right)\ox_{R}C$ as $K$-algebras.
    \item neutral element: for all $A\in \CSA$, $[K]_{\sim_{\Br}}\cdot[A]_{\sim_{\Br}}=[A]_{\sim_{\Br}}=[A]_{\sim_{\Br}}\cdot [K]_{\sim_{\Br}}$. Since $[K]_{\sim_{\Br}}\cdot[A]_{\sim_{\Br}}=[K\ox_{K} A]_{\sim_{\Br}}$, in~\cref{con:matrix-tensor-matrix}, we see that $\Mat_{n}(A)\cong A\ox_{K}\Mat_{n}(K)$, by~\cref{lem:br-mul-wd}, $A\ox_{K}\Mat_{n}(K)$ is Brauer equivalent to $A\ox_{K} K$ since $K\sim_{\Br}\Mat_{n}(K)$.
    \item cancellation: for all $A \in \CSA$, we need $[A]_{\sim_{\Br}}\cdot[A^{\opp}]_{\sim_{\Br}}$, that is we want $A\ox_{K}A^{\opp}\sim_{\Br} K$. By~\cref{con:self-tensor-opp-iso-end}, we have $A\ox_{K}A^{\opp}\cong \End_{K}A$ which is isomorphic to $\Mat_{\dim_{K}A}(K)$ as $K$-algebras.
  \end{enumerate}
  \leanok
  \lean{BrauerGroup.Bruaer_Group}
\end{construction}

\begin{theorem}
  \label{thm:br-triv-alg-closed}
  If $K$ is algebraically closed, $\Br(K)$ is trivial; in particular $\br_{n}(\mathbb{C})$ is trivial.
  \leanok
  \lean{BrauerGroup.Alg_closed_eq_one}
\end{theorem}
\begin{proof}
  We need to show that every $A\in\CSA$ is isomorphic to $\Mat_{n}(K)$ for some $K$ when $K$ is algebraically closed. Indeed, by~\cref{thm:wed-artin-algebra}, $A \cong \Mat_{n}(D)$ for some division algebra $D$ and $n\in\nat_{\ge0}$. Since $K$ is algebraically closed and $D$ is an integral domain and finite dimensional, the structure morphism $\rho: K\to D$ is a isomorphism; therefore $A\cong \Mat_{n}(K)$.
\end{proof}

\begin{lemma}
  \label{lem:common-div-alg}
  \leanok
  \lean{IsBrauerEquivalent.exists_common_division_algebra}
  Let $A, B \in \csa_{K}$. There exists a division $K$-algebra $D$ and non-zero $m,n\in\mathbb{N}$ such that $A\cong\mat_{m}(D)$ and $B\cong\mat_{n}(D)$ as $K$-algebras.
\end{lemma}

\begin{proof}
  By~\cref{thm:wed-artin-algebra}, we can find division algebras $S_{A}, S_{B}$ and non-zero $m, n\in\mathbb{N}$ such that $A\cong\mat_{n}\left(S_{A}\right)$ and $B\cong\mat_{m}\left(S_{B}\right)$ as $K$-algebras. Hence $B\sim_{\br}A\sim_{\br}\mat_{n}\left(S_{A}\right)\sim_{\br}S_{A}$, in another word, for some non-zero $a, a'\in\mathbb{N}$, we have $\mat_{a}(B)\cong\mat_{a'}\left(S_{A}\right)$ as $K$-algebras. Hence, by~\cref{thm:wed-artin-uniq}, we have that $S_{A}\cong S_{B}$ as $K$-algebras and the lemma is proved.
\end{proof}

\section{Base Change}
In this section let $E/K$ be a field extension. We have seen in~\cref{cor:csa-basechange} that if $A\in\CSA_{K}$ then $E\otimes_{K}A\in\CSA_{E}$; therefore we have a set-theoretic function $\CSA_{K}\to\CSA_{E}$. In this section we prove that this descends to a group homomorphism $\Br(K)\to\Br(E)$. For brevity, if $A\in \CSA_{K}$, we dentoe $E\ox_{K}A$ as $A_{E}$ when this causes no confusion.
\begin{construction}
  \label{con:br-base-change}
  We will construct a series of isomorphisms (either over $K$ or $E$) to arrive at the conclusion that $A\sim_{\Br_{K}}B$ implies $A_{E}\sim_{\Br_{E}}B_{E}$. Assume $m,n\in\nat_{\ge0}$ are such that $\Mat_{m}(A)\cong\Mat_{n}(B)$ are $K$-algebras. Then we do the following calculation: as $E$-algebras
  \[
    \begin{aligned}
      \Mat_{m}\left(A_{E}\right)
      &\cong A_{E}\otimes_{E}\Mat_{m}(E) &\text{see~\cref{con:matrix-tensor-matrix}}\\
      &\cong A_{E}\otimes_{E}\left(E\ox_{K} \Mat_{m}(K)\right) & \text{see}~\dagger\\
      &\cong E \ox_{K}\left(A\ox_{K}\Mat_{m}(K)\right) &\text{see}~\ddagger\\
      &\cong E\ox_{K}\Mat_{m}(A) &\text{see~\cref{con:matrix-tensor-matrix} and}~\dagger\!\!\dagger \\
      \Mat_{n}\left(B_{E}\right) &\cong E\ox_{K}\Mat_{n}(B) &\text{same as the case of}~A\\
      Mat_{m}\left(A_{E}\right)&\cong \Mat_{n}\left(B_{E}\right) &\text{see}~\dagger\!\!\dagger
    \end{aligned}.
  \]

   \noindent$\dagger$: Wee need to check $\Mat_{m}(E)\cong E\otimes_{K}\Mat_{m}K$ as $E$-algebras since~\cref{con:matrix-tensor-matrix} only gives a $K$-algebra isomorphism. If $e \in E$, then its image in $E\ox_{K}\Mat_{m}(K)$ is $e\otimes 1$ and its image in $\Mat_{m}(E)$ is $\diag(e)$ which under the $K$-algebra isomorphism is mapped to $\sum_{ij}\diag(e)_{ij}\cdot \delta_{ij}=e\ox1$.

   \noindent$\ddagger$: This is defined by combining two $E$-algebra homomorphisms
   \[
     A_{E}\to A_{E}\ox_{K}\Mat_{m}(K)\to E\ox_{K}\left(A\ox_{K}\Mat_{m}(K)\right)\]
     and
     \[
       E\ox_{K}\Mat_{m}(K)\to \left(E\ox_{K}\Mat_{m}(K)\right)\ox_{K}A \to E\ox_{K}(A\ox_{K}\Mat_{m}(K)).
    \]
   Since $\left(E\ox_{K}A\right)\ox_{E}\left(E\ox_{K}\Mat_{m}(K)\right)$ is a simple ring, this morphism is automatically injective. It is surjective as well: let $x\in E\ox_{K}\left(A\ox_{K}\Mat_{m}(K)\right)$, without loss of generality, assume $x=e\ox (a\ox\delta_{ij})$ for some $e\in E$, $a\in A$. Then precisely $\left(e\ox a\right)\ox\left(1\ox\delta_{ij}\right)$ is mapped to $x$.

   \noindent$\dagger\!\dagger$: a $K$-algebra isomorphism $A\cong B$ gives an $E$-algebra isomorphism $E\ox_{K}A \cong E\ox_{K} B$.

   Thus we have a well defined function $\Br(K)\to\Br(E)$. We now check that this is a group homomorphism. $[K]_{\sim_{\Br_{K}}}$ is mapped to $[E\ox_{K}K]_{\sim_{\Br_{E}}}$ but $E\ox_{K}K\cong E$ as $E$-algebra. For $A, B\in\CSA_{K}$, we have that $[AB]_{\sim_{\Br_{K}}}$ is mapped to $\left(A\ox_{K}B\right)_{E}\cong A_{E}\ox_{E} B_{E}$ as $E$-algebras; hence $[AB]_{\sim_{\Br_{K}}}$ and $[A]_{\sim_{\Br_{K}}}\cdot[B]_{\sim_{\Br_{K}}}$ have the same image under base change.
   \leanok
   \lean{BrauerGroupHom.BaseChange}
 \end{construction}
 Denote the base change morphism in~\cref{con:br-base-change} as $\Br_{K}^{E}$.
 \begin{lemma}
   $\Br_{K}^{K}$ is identity.
   \leanok
 \end{lemma}
 \begin{proof}
   If $A\in\CSA$, then $A\sim_{\Br}K\ox_{K}A$.
 \end{proof}

 \begin{lemma}\label{lem:br-base-change-self}
   Consider the tower of field extension $E/F/K$,
   \[
     \Br_{K}^{E} = \Br_{E}^{F}\circ\Br_{K}^{E}.
   \]
 \end{lemma}

 \begin{proof}
   If $A\in\CSA_{K}$, then $E\ox_{F}\left(F\ox_{K}A\right)$ is isomorphic to $E \ox_{F} A$ as $E$-algebras.
   \leanok
   \lean{BrauerGroupHom.baseChange_idem}
 \end{proof}

 \begin{corollary}\label{lem:br-base-change-tower}
   $\Br$ forms a functor from category of field to category of abelian groups.
   \leanok
   \lean{BrauerGroupHom.Br}
 \end{corollary}
 \begin{proof}
   This is the categorical version of~\cref{lem:br-base-change-self} and~\cref{lem:br-base-change-tower}.
 \end{proof}

 \begin{definition}[Relative Brauer Group]
   Let $E/K$ be a field extension, we define the relative Brauer group $\Br(E/K)$ to be the kernel of the base change morphism $\Br_{K}^{E}$.
 \end{definition}

 \begin{remark}
   Unpacking the definition of the relative Brauer group, we see that for any $A \in \CSA_{K}$, if $E\ox_{K}A\cong\mat_{n}(E)$ as $E$-algebras, then $\br^{E}_{K}\left([A]_{\sim_{\br}}\right)=1$.
 \end{remark}

 \begin{definition}[Splitting Field]
   For any field extension $E/K$ and any $K$-algebra $A$, we say $E$ is a splitting field of $A$ if and only if $E\ox_{K}A \cong \mat_{n}(E)$ as $E$-algebras for some non-zero $n$. We also say $E$ splits $A$ or $A$ is splited by $E$
   \leanok
   \lean{isSplit}
 \end{definition}

  \begin{theorem}
   \label{thm:split-iff-mem-relative}
   Let $E/K$ be a field extension and $A\in\csa_{K}$, $E$ splits $A$ if and only if $[A]_{\sim_{\br}}\in\br(E/K)$.
   \leanok
   \lean{BrauerGroup.split_iff}
 \end{theorem}

 \begin{proof}
   The ``only if'' part is by definition. For the other direction, we know by definition that $\mat_{n}(E\ox_{K}A)\cong\mat_{m}(E)$ as $E$-algebras for some non-zero $m, n$. By~\cref{thm:wed-artin-algebra}, we find some division algebra $D$ and non-zero natural number $p$ such that $E\ox_{K}A\cong \mat_{p}(D)$ as $E$-algebras. Thus $\mat_{pm}(E)\cong\mat_{pn}\left(E\ox_{K}A\right)\cong\mat_{p^{2}n}(D)$ as $E$-algebras. By~\cref{thm:wed-artin-uniq}, we conclude that $E\cong D$ as $E$-algebras. Hence $E\ox_{K}A\cong \mat_{p}(E)$, in another word, $E$ splits $A$.
 \end{proof}

 \begin{remark}
   In light of~\cref{lem:br-base-change-tower}, if $K$ is algebraic closed then $K$ splits any $K$-algebra $A$. Indeed, $K$ splits $A$ if and only if $[A]_{\sim\br}$ but $[A]_{\sim\br}$ is equal to $1$.
 \end{remark}

 \begin{remark}
   If two $\csa_{K}$ are Brauer equivalent, in another word, $A \sim_{\br_{K}} B$, then $E$ splits $A$ if and only if $E$ splits $B$. Indeed, if $A$ and $B$ are equivalent, then $[A]_{\sim\br} \in \br(E/K)$ if and only if $[B]_{\sim\br}\in\br(E/K)$.
 \end{remark}


 \section{Good Representative Lemma}
 In this section, let $K/F$ be a finite dimensional field extension.

 \begin{lemma}
   \label{lem:good-rep-inv}
   Let $A\in\csa_{F}$ splitted by $K$. There exists a $B\in\csa_{F}$ such that
   \begin{itemize}
     \item $[A]_{\sim_{\br}}[B]_{\sim_{\br}}=1$
     \item there exists $F$-algebra map $K\hookrightarrow B$
     \item ${\left(\dim_{F}K\right)}^{2}=\dim_{F}B$.
   \end{itemize}
   \leanok
   \lean{exists_embedding_of_isSplit}
 \end{lemma}

 \begin{proof}
   Since $K$ splits $A$, we find a non-zero natural number $n$ such that $K\ox_{F}A\cong\mat_{n}K\cong\End_{K}\left(K^{n}\right)$ as $K$-algebras.
   We define an $F$-algebra map $\iota: A\to\End_{F}\left(K^{n}\right)$ by
   \begin{center}
     \begin{tikzcd}
       A \arrow{r} & K \ox_{F} A \arrow{r}{\cong} & \End_K\left(K^n\right) \arrow{r}{|_{F}} & \End_F\left(K^n\right)
     \end{tikzcd},
   \end{center}
   where $|_{F}$ is restriction of scalars. Since $A$ is simple, $\iota$ is injective, therefore $A \cong \iota(A)$ as $F$-algebras. Define $B$ as $C_{\End_{F}\left(K^{n}\right)}(\iota(A))$, the centralizer of the range of $\iota$ in $\End_{F}\left(K^{n}\right)$.
   We construct an embedding $K \hookrightarrow B$ by $r \mapsto (r \cdot \bullet)$

   $B$ is a central $F$-algebra: if $x \in Z(B)$, then $x \in \iota(A)$ because by~\cref{thm:double-centralizer}, it is sufficient to prove that $x$ is in $C_{\End_{F}\left(K^{n}\right)}\left(B\right)$ which follows from the fact that $x \in Z(B)$. In fact, $x \in Z(\iota(A))$: suppose $a \in A$, we need to check $x \cdot \iota(a) = \iota(a)\cdot x$, this is the case because $B$ is defined as the centralizer of $\iota(A)$. Since $\iota(A) \cong A$ as $F$-algebras, $\iota(A)$ is $F$-central, hence $x \in F$.

   $B$ is a simple ring: by~\cref{lem:simple-centralizer}, it is sufficient to prove that $\iota(A)$ is a simple ring which comes from $A \cong \iota(A)$ as $F$-algebras.

   By~\cref{cor:self-tensor-centralizer}, we have $F$-algebra isomorphism $\End_{F}\left(K^{n}\right)\cong \iota(A)\ox_{F}B \cong A \ox_{F}B$. Since $\End_{F}\left(K^{n}\right)\cong\mat_{\dim_{F}\left(K^{n}\right)}\left(F\right)$ as $F$-algebras, we see that $[A]_{\sim_{\br}}$ and $[B]_{\sim_{\br}}$ are inverses.

   By~\cref{lem:dim-centralizer}, $\dim_{F}B\cdot \dim_{F}\iota(A) = \dim_{F}B\cdot\dim_{F}A = \dim_{F}\End_{F}\left(K^{n}\right) = {\left(\dim_{F}\left(K^{n}\right)\right)}^{2}={\left(\dim_{F}K\cdot \dim_{K}\left(K^{n}\right)\right)}^{2}=n^{2}\cdot{\left(\dim_{F}K\right)}^{2}$. On the other hand, since $K\ox_{F}A\cong\mat_{n}K$, we have $\dim_{F}K\ox_{F}A=\dim_{F}K\cdot\dim_{F}A=\dim_{F}\mat_{n}K=\dim_{F}K\dim_{K}\mat_{n}K=n^{2}\dim_{F}K$. Since $\dim_{F}K\ne 0$ , we conclude $\dim_{F}A=n^{2}$. Since $n\ne 0$ and $\dim_{F}(B)\cdot \dim_{F}(A)=n^{2}\dim_{F}(B)=n^{2}{\left(\dim_{F}K\right)}^{2}$, we get the desired result.

 \end{proof}

 \begin{corollary}\label{cor:good-rep}
   Let $A\in\csa_{F}$ splitted by $K$. There exists a $B \in \csa_{F}$ such that
   \begin{itemize}
     \item $[B]_{\sim_{\br}}=[A]_{\sim_{\br}}$
     \item there exists an $F$-algebra map $K\hookrightarrow B$
     \item  ${\left(\dim_{F}K\right)}^{2}=\dim_{F}B$.
  \end{itemize}
 \end{corollary}

 \begin{proof}
   Let $B$ and $\iota : K \hookrightarrow B$ be as in~\cref{lem:good-rep-inv}. Consider $B^{\opp}$ and $K \hookrightarrow B \to B^{\opp}$. This works.
 \end{proof}

 \begin{theorem}\label{thm:good-rep-iff-split}
   Let $A\in\csa_{F}$. $K$ splits $A$ if and only if there exists a $B\in\csa_{F}$ such that
   \begin{itemize}
     \item $[B]_{\sim_{\br}}=[A]_{\sim_{\br}}$
     \item there exists an $F$-algebra map $K \hookrightarrow B$
     \item ${\left(\dim_{F}K\right)}^{2}=\dim_{F}B$.
   \end{itemize}
   \leanok
   \lean{isSplit_iff_dimension}
 \end{theorem}
 \begin{proof}
   The ``if'' direction is~\cref{cor:good-rep}. For the ``only if'' direction, let $B \in\csa_{F}$ and $\iota : K \hookrightarrow B$ be given. We give $B$ a $K$-module structure by right multiplication, that is for any $a \in K$ and $b \in B$, we define $a \cdot b := b \cdot \iota(a)$.
   Since $B$ is a finite dimensional $F$-vector space and $K/F$ is a finite dimensional field extension, $B$ is a finite dimensional $K$-vector space as well. Since $[B]_{\sim_{\br}}=[A]_{\sim_{\br}}$, it is sufficient to show that $K$ splits $B$.
   We define an $F$-bilinear map $\mu : K \to B \to \End_{K}B$ by $(c, a) \mapsto (c \cdot a \cdot \bullet)$ which induce an $F$-linear map $\mu' : K \ox_{F} B \to \End_{K}B$. Since for any $r, c\in K$ and $a \in B$, we have $\mu'\left(r \cdot c\ox a\right)(a') = a a' \iota(rc)=aa'\iota(c)\iota(r)=r\cdot \mu'(c\ox a)$, that is $\mu'$ is $K$-linear as well. Note that \[\mu'(1)=\mu'(1\ox 1) = (1\cdot 1\cdot \bullet) = 1\] and that
   \[
     \begin{aligned}
       \mu'(c\ox a \cdot c'\ox a')(a'') &= \mu'(cc'\ox aa')(a'') \\
                                        &= cc' \cdot aa' \cdot a'' \\
                                        &= aa' a'' \iota(c c') \\
                                        &= a(a' a'' \iota(c'))\iota(c) \\
                                        &=\mu'(c \ox a)(a' a'' \iota(c'))\\
                                        &=\mu'(c\ox a)\left(\mu'(c'\ox a')(a'')\right) \\
                                        &= \left(\mu'(c\ox a)\circ\mu'(c'\ox a')\right)(a'')
     \end{aligned},
     \]
     that is, $\mu'$ is an $K$-algebra map.

   If we can show that $\mu'$ is a bijection, we will prove the result for $K\ox_{F} B \cong \End_{K} B \cong \mat_{\dim_{K} B}K$ as $K$-algebras. By~\cref{cor:alghom-bijective-of-dim-eq}, it is sufficient to show $\dim_{K}K\ox_{F}B = \dim_{K}\End_{K}B$. Let $n$ denote $\dim_{F}K$. Since, $\dim_{F}K\dim_{K}K\ox_{F}B =\dim_{F}K\ox_{F}B=\dim_{F}K \dim_{F}B$. we have $\dim_{K}K\ox_{F}B = \dim_{F}B = {\left(\dim_{F}K\right)}^{2}$. On the other hand, since ${\left(\dim_{F}K\right)}^{2}=\dim_{F} B = \dim_{F}K\dim_{K}B$, we have $\dim_{K} B = \dim_{F}K$; thus $\dim_{K}\End_{K} B = {\left(\dim_{K}B\right)}^{2}={\left(\dim_{F}K\right)}^{2}$ and the result is proved.
 \end{proof}

 In light of~\cref{thm:good-rep-iff-split}, we isolate the following useful definition:
 \begin{definition}[Good Representation]\label{def:good-rep}
   For any $X \in \br(F)$, a good representation of $X$ is an $A \in \csa_{F}$ and an $F$-algebra map $K \hookrightarrow A$ which we often denote as $\iota$ or $\iota_{A}$ such that $[A]_{\sim_{\br}} = X$ and $\dim_{F}A={\left(\dim_{F}K\right)}^{2}$.
   \leanok
   \lean{GoodRep}
 \end{definition}

 \begin{corollary}
   \label{cor:mem-relative-br-iff-good-rep}
   For any $X\in\br(F)$, $X\in\br(K/F)$ if and only if $X$ admits a good representation.
   \leanok
   \lean{mem_relativeBrGroup_iff_exists_goodRep}
 \end{corollary}
 \begin{proof}
   Rephrase of~\cref{thm:good-rep-iff-split} and~\cref{thm:split-iff-mem-relative}.
 \end{proof}

We observe the following easy result about good representations. Let $X \in \br(F)$ and $A$ be a good representation of $X$.

\begin{lemma}
  The range $\iota_{A}(A)$ is a simple ring.
  \leanok
  \lean{GoodRep.ιRange}
\end{lemma}

\begin{proof}
  Because $K$ is a simple ring, $\iota_{A}$ is injective therefore $\iota_{A}(A)\cong K$.
\end{proof}

\begin{lemma}
  $C_{A}\left(\iota_{A}(A)\right) = \iota_{A}(A)$.
  \leanok
  \lean{GoodRep.centralizerιRange}
\end{lemma}
\begin{proof}
  In the language of~\cref{sec:subfield}, $\iota_{A}(A)$ is a subfield of $A$, hence by~\cref{lem:tfae-subfield}, we only need to show $\dim_{F}A = {\left(\dim_{F}\iota_{A}(A)\right)}^{2}$. But $\dim_{F}A={\left(\dim_{F}K\right)}^{2}$ and $\iota(A)\cong K$.
\end{proof}

\begin{construction}
  We give $A$ a $K$-module structure by {\em left} multiplication, that is for any $c \in K$ and $a \in A$, we define $c\cdot a$ to be $\iota_{A}(c)a$. Then $A$ is a finite dimensional $K$-vector space and $\dim_{K}A=\dim_{F}K$: indeed $\dim_{F}K\cdot\dim_{K}A = \dim_{F}K\cdot \dim_{F}K = \dim_{F}A$.
  \leanok
  \lean{GoodRep.dim_eq'}
\end{construction}
\section{The Second Galois Cohomology}

%%% Local Variables:
%%% mode: LaTeX
%%% TeX-master: "../print"
%%% End:
