\chapter{Brauer Group}\label{cha:brauer-group}

\section{Construction of Brauer Group}
Let $K$ be a field. We denote the class of finite dimensional central simple $K$-algebras as $\CSA_{K}$. When $K$ is clear, we drop the subscript.

\begin{remark}
  By~\cref{lem:tensor-central} and~\cref{thm:tensor-csa}, $\CSA$ is closed under tensor product, that is if $A, B\in\CSA$, we have $A\otimes_{K} B\in\CSA$ as well.
\end{remark}

\begin{definition}[Brauer Equivalence]
  For any two $A, B\in\CSA$, we say $A$ and $B$ are Brauer equivalent, when there exists $m, n \in \mathbb{N}_{\ge0}$ such that $\Mat_{m}(A)\cong \Mat_{n}B$ as $K$-algebras. We denote this relation as $A\sim_{\Br_{K}} B$, when $K$ is clear, we drop the subscript.
\end{definition}

\begin{lemma}
  $\sim_{\Br}$ is reflexive.
  \leanok
  \lean{IsBrauerEquivalent.refl}
\end{lemma}
\begin{proof}
  Indeed, $A \cong \Mat_{1}(A)$ as $K$-algerbas.
\end{proof}

\begin{lemma}
  $\sim_{\Br}$ is symmetric.
  \leanok
  \lean{IsBrauerEquivalent.symm}
\end{lemma}
\begin{proof}
  Indeed, just exchange $m$ and $n$.
\end{proof}

\begin{lemma}
  $\sim_{\Br}$ is transitive.
  \leanok
  \lean{IsBrauerEquivalent.trans}
\end{lemma}
\begin{proof}
  Let $A\sim_{\Br}B$ and $B\sim_{\Br}C$; that is for some $m,n,p, q\in\mathbb{N}_{\ge0}$ we have $\Mat_{n}(A)\cong\Mat_{m}(B)$ and $\Mat_{p}(B)\cong \Mat_{q}(C)$ as $K$-algebras. Hence, from~\cref{con:matrix-matrix}, we have the following:
  \[
    \begin{aligned}
      \Mat_{np}(A)&\cong \Mat_{p}\left(\Mat_{n}(A)\right)\cong\Mat_{p}\left(\Mat_{m}(B)\right)\\
                  &\cong\Mat_{mp}(B)\cong\Mat_{m}\left(\Mat_{p}(B)\right)\\
      &\cong\Mat_{m}\left(\Mat_{q}(C)\right)\cong\Mat_{mq}(C).
    \end{aligned}
  \]
  In another word, $A\sim_{\Br}C$.
\end{proof}

Hence $\sim_{\Br}$ is really an equivalence relation, we denote the quotient ${}^{\CSA}/_{\sim_{\Br}}$ as $\Br(K)$.

\begin{lemma}\label{lem:br-mul-wd}
  $(\bullet\otimes_{K}\bullet):\CSA\times\CSA \to \CSA$ descends to a function on $\Br(K)$.
  \leanok
  \lean{BrauerGroup.eqv_tensor_eqv}
\end{lemma}
\begin{proof}
  We need to prove that for all $A, B, C, D \in \CSA$ such that $A\sim_{\Br} B$ and $C\sim_{\Br}D$, $A\otimes_{R}C \sim_{\Br} D\otimes_{R}D$ as well.
  Suppose $\Mat_{m}(A)\cong \Mat_{n}(B)$ as $K$-algebras and $\Mat_{p}(C)\cong\Mat_{q}(D)$, by~\cref{con:matrix-tensor-matrix}, we have
  \[
    \begin{aligned}
      \Mat_{mp}\left(A\otimes_{R} C\right)&\cong \Mat_{m}(A)\otimes_{R}\Mat_{p}(C) \\
                                          &\cong \Mat_{n}(B)\otimes_{R}\Mat_{q}(D)\\
      &\cong \Mat_{nq}\left(B\otimes_{R}D\right).
    \end{aligned}
  \]

\end{proof}

\section{Base Change}

\section{Good Representative Lemma}

\section{The Second Galois Cohomology}

%%% Local Variables:
%%% mode: LaTeX
%%% TeX-master: "../print"
%%% End:
