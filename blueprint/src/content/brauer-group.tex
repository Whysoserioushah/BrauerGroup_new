\chapter{Brauer Group}\label{cha:brauer-group}

\section{Construction of Brauer Group}
Let $K$ be a field. We denote the class of finite dimensional central simple $K$-algebras as $\CSA_{K}$. When $K$ is clear, we drop the subscript.

\begin{remark}
  By~\cref{lem:tensor-central} and~\cref{thm:tensor-csa}, $\CSA$ is closed under tensor product, that is if $A, B\in\CSA$, we have $A\otimes_{K} B\in\CSA$ as well.
\end{remark}

\begin{definition}[Brauer Equivalence]
  For any two $A, B\in\CSA$, we say $A$ and $B$ are Brauer equivalent, when there exists $m, n \in \mathbb{N}_{\ge0}$ such that $\Mat_{m}(A)\cong \Mat_{n}B$ as $K$-algebras. We denote this relation as $A\sim_{\Br_{K}} B$, when $K$ is clear, we drop the subscript.
\end{definition}

\begin{remark}
  Isomorphic $K$-algebras are Brauer equivalent.
\end{remark}

\begin{lemma}
  $\sim_{\Br}$ is reflexive.
  \leanok
  \lean{IsBrauerEquivalent.refl}
\end{lemma}
\begin{proof}
  Indeed, $A \cong \Mat_{1}(A)$ as $K$-algerbas.
\end{proof}

\begin{lemma}
  $\sim_{\Br}$ is symmetric.
  \leanok
  \lean{IsBrauerEquivalent.symm}
\end{lemma}
\begin{proof}
  Indeed, just exchange $m$ and $n$.
\end{proof}

\begin{lemma}
  $\sim_{\Br}$ is transitive.
  \leanok
  \lean{IsBrauerEquivalent.trans}
\end{lemma}
\begin{proof}
  Let $A\sim_{\Br}B$ and $B\sim_{\Br}C$; that is for some $m,n,p, q\in\mathbb{N}_{\ge0}$ we have $\Mat_{n}(A)\cong\Mat_{m}(B)$ and $\Mat_{p}(B)\cong \Mat_{q}(C)$ as $K$-algebras. Hence, from~\cref{con:matrix-matrix}, we have the following:
  \[
    \begin{aligned}
      \Mat_{np}(A)&\cong \Mat_{p}\left(\Mat_{n}(A)\right)\cong\Mat_{p}\left(\Mat_{m}(B)\right)\\
                  &\cong\Mat_{mp}(B)\cong\Mat_{m}\left(\Mat_{p}(B)\right)\\
      &\cong\Mat_{m}\left(\Mat_{q}(C)\right)\cong\Mat_{mq}(C).
    \end{aligned}
  \]
  In another word, $A\sim_{\Br}C$.
\end{proof}

Hence $\sim_{\Br}$ is really an equivalence relation, we denote the quotient ${}^{\CSA}/_{\sim_{\Br}}$ as $\Br(K)$.

\begin{lemma}\label{lem:br-mul-wd}
  $(\bullet\otimes_{K}\bullet):\CSA\times\CSA \to \CSA$ descends to a function on $\Br(K)$.
  \leanok
  \lean{BrauerGroup.eqv_tensor_eqv}
\end{lemma}
\begin{proof}
  We need to prove that for all $A, B, C, D \in \CSA$ such that $A\sim_{\Br} B$ and $C\sim_{\Br}D$, $A\otimes_{R}C \sim_{\Br} B\otimes_{R}D$ as well.
  Suppose $\Mat_{m}(A)\cong \Mat_{n}(B)$ as $K$-algebras and $\Mat_{p}(C)\cong\Mat_{q}(D)$, by~\cref{con:matrix-tensor-matrix}, we have
  \[
    \begin{aligned}
      \Mat_{mp}\left(A\otimes_{R} C\right)&\cong \Mat_{m}(A)\otimes_{R}\Mat_{p}(C) \\
                                          &\cong \Mat_{n}(B)\otimes_{R}\Mat_{q}(D)\\
      &\cong \Mat_{nq}\left(B\otimes_{R}D\right).
    \end{aligned}
  \]

\end{proof}

\begin{construction}[Brauer Group]\label{con:br}
  $\Br(K)$ forms a group under $[A]_{\sim_{\Br}}\cdot[B]_{\sim_{\Br}}=[A\otimes_{K}B]_{\sim_{\Br}}$ with neutral element $[K]_{\sim_{\Br}}$ where $A, B\in\CSA$ and $[A]_{\sim_{Br}}^{-1}=[A^{\opp}]_{\sim_{\Br}}$. We need to prove the following properties:
  \begin{enumerate}
    \item associativity: for all $A, B, C\in\CSA$, $[A]_{\sim_{\Br}}\cdot\left([B]_{\sim_{\Br}}\cdot[C]_{\sim_{\Br}}\right)=\left([A]_{\sim_{\Br}}\cdot[B]_{\sim_{\Br}}\right)\cdot [C]_{\sim_{\Br}}$ because $A\ox_{R}\left(B\ox_{R}C\right)\cong \left(A\ox_{R}B\right)\ox_{R}C$ as $K$-algebras.
    \item neutral element: for all $A\in \CSA$, $[K]_{\sim_{\Br}}\cdot[A]_{\sim_{\Br}}=[A]_{\sim_{\Br}}=[A]_{\sim_{\Br}}\cdot [K]_{\sim_{\Br}}$. Since $[K]_{\sim_{\Br}}\cdot[A]_{\sim_{\Br}}=[K\ox_{K} A]_{\sim_{\Br}}$, in~\cref{con:matrix-tensor-matrix}, we see that $\Mat_{n}(A)\cong A\ox_{K}\Mat_{n}(K)$, by~\cref{lem:br-mul-wd}, $A\ox_{K}\Mat_{n}(K)$ is Brauer equivalent to $A\ox_{K} K$ since $K\sim_{\Br}\Mat_{n}(K)$.
    \item cancellation: for all $A \in \CSA$, we need $[A]_{\sim_{\Br}}\cdot[A^{\opp}]_{\sim_{\Br}}$, that is we want $A\ox_{K}A^{\opp}\sim_{\Br} K$. By~\cref{con:self-tensor-opp-iso-end}, we have $A\ox_{K}A^{\opp}\cong \End_{K}A$ which is isomorphic to $\Mat_{\dim_{K}A}(K)$ as $K$-algebras.
  \end{enumerate}
  \leanok
  \lean{BrauerGroup.Bruaer_Group}
\end{construction}

\begin{theorem}
  \label{thm:br-triv-alg-closed}
  If $K$ is algebraically closed, $\Br(K)$ is trivial; in particular $\br_{n}(\mathbb{C})$ is trivial.
  \leanok
  \lean{BrauerGroup.Alg_closed_eq_one}
\end{theorem}
\begin{proof}
  We need to show that every $A\in\CSA$ is isomorphic to $\Mat_{n}(K)$ for some $K$ when $K$ is algebraically closed. Indeed, by~\cref{thm:wed-artin-algebra}, $A \cong \Mat_{n}(D)$ for some division algebra $D$ and $n\in\nat_{\ge0}$. Since $K$ is algebraically closed and $D$ is an integral domain and finite dimensional, the structure morphism $\rho: K\to D$ is a isomorphism; therefore $A\cong \Mat_{n}(K)$.
\end{proof}

\begin{lemma}
  \label{lem:common-div-alg}
  \leanok
  \lean{IsBrauerEquivalent.exists_common_division_algebra}
  Let $A, B \in \csa_{K}$. There exists a division $K$-algebra $D$ and non-zero $m,n\in\mathbb{N}$ such that $A\cong\mat_{m}(D)$ and $B\cong\mat_{n}(D)$ as $K$-algebras.
\end{lemma}

\begin{proof}
  By~\cref{thm:wed-artin-algebra}, we can find division algebras $S_{A}, S_{B}$ and non-zero $m, n\in\mathbb{N}$ such that $A\cong\mat_{n}\left(S_{A}\right)$ and $B\cong\mat_{m}\left(S_{B}\right)$ as $K$-algebras. Hence $B\sim_{\br}A\sim_{\br}\mat_{n}\left(S_{A}\right)\sim_{\br}S_{A}$, in another word, for some non-zero $a, a'\in\mathbb{N}$, we have $\mat_{a}(B)\cong\mat_{a'}\left(S_{A}\right)$ as $K$-algebras. Hence, by~\cref{thm:wed-artin-uniq}, we have that $S_{A}\cong S_{B}$ as $K$-algebras and the lemma is proved.
\end{proof}

\section{Base Change}
In this section let $E/K$ be a field extension. We have seen in~\cref{cor:csa-basechange} that if $A\in\CSA_{K}$ then $E\otimes_{K}A\in\CSA_{E}$; therefore we have a set-theoretic function $\CSA_{K}\to\CSA_{E}$. In this section we prove that this descends to a group homomorphism $\Br(K)\to\Br(E)$. For brevity, if $A\in \CSA_{K}$, we dentoe $E\ox_{K}A$ as $A_{E}$ when this causes no confusion.
\begin{construction}
  \label{con:br-base-change}
  We will construct a series of isomorphisms (either over $K$ or $E$) to arrive at the conclusion that $A\sim_{\Br_{K}}B$ implies $A_{E}\sim_{\Br_{E}}B_{E}$. Assume $m,n\in\nat_{\ge0}$ are such that $\Mat_{m}(A)\cong\Mat_{n}(B)$ are $K$-algebras. Then we do the following calculation: as $E$-algebras
  \[
    \begin{aligned}
      \Mat_{m}\left(A_{E}\right)
      &\cong A_{E}\otimes_{E}\Mat_{m}(E) &\text{see~\cref{con:matrix-tensor-matrix}}\\
      &\cong A_{E}\otimes_{E}\left(E\ox_{K} \Mat_{m}(K)\right) & \text{see}~\dagger\\
      &\cong E \ox_{K}\left(A\ox_{K}\Mat_{m}(K)\right) &\text{see}~\ddagger\\
      &\cong E\ox_{K}\Mat_{m}(A) &\text{see~\cref{con:matrix-tensor-matrix} and}~\dagger\!\!\dagger \\
      \Mat_{n}\left(B_{E}\right) &\cong E\ox_{K}\Mat_{n}(B) &\text{same as the case of}~A\\
      Mat_{m}\left(A_{E}\right)&\cong \Mat_{n}\left(B_{E}\right) &\text{see}~\dagger\!\!\dagger
    \end{aligned}.
  \]

   \noindent$\dagger$: Wee need to check $\Mat_{m}(E)\cong E\otimes_{K}\Mat_{m}K$ as $E$-algebras since~\cref{con:matrix-tensor-matrix} only gives a $K$-algebra isomorphism. If $e \in E$, then its image in $E\ox_{K}\Mat_{m}(K)$ is $e\otimes 1$ and its image in $\Mat_{m}(E)$ is $\diag(e)$ which under the $K$-algebra isomorphism is mapped to $\sum_{ij}\diag(e)_{ij}\cdot \delta_{ij}=e\ox1$.

   \noindent$\ddagger$: This is defined by combining two $E$-algebra homomorphisms
   \[
     A_{E}\to A_{E}\ox_{K}\Mat_{m}(K)\to E\ox_{K}\left(A\ox_{K}\Mat_{m}(K)\right)\]
     and
     \[
       E\ox_{K}\Mat_{m}(K)\to \left(E\ox_{K}\Mat_{m}(K)\right)\ox_{K}A \to E\ox_{K}(A\ox_{K}\Mat_{m}(K)).
    \]
   Since $\left(E\ox_{K}A\right)\ox_{E}\left(E\ox_{K}\Mat_{m}(K)\right)$ is a simple ring, this morphism is automatically injective. It is surjective as well: let $x\in E\ox_{K}\left(A\ox_{K}\Mat_{m}(K)\right)$, without loss of generality, assume $x=e\ox (a\ox\delta_{ij})$ for some $e\in E$, $a\in A$. Then precisely $\left(e\ox a\right)\ox\left(1\ox\delta_{ij}\right)$ is mapped to $x$.

   \noindent$\dagger\!\dagger$: a $K$-algebra isomorphism $A\cong B$ gives an $E$-algebra isomorphism $E\ox_{K}A \cong E\ox_{K} B$.

   Thus we have a well defined function $\Br(K)\to\Br(E)$. We now check that this is a group homomorphism. $[K]_{\sim_{\Br_{K}}}$ is mapped to $[E\ox_{K}K]_{\sim_{\Br_{E}}}$ but $E\ox_{K}K\cong E$ as $E$-algebra. For $A, B\in\CSA_{K}$, we have that $[AB]_{\sim_{\Br_{K}}}$ is mapped to $\left(A\ox_{K}B\right)_{E}\cong A_{E}\ox_{E} B_{E}$ as $E$-algebras; hence $[AB]_{\sim_{\Br_{K}}}$ and $[A]_{\sim_{\Br_{K}}}\cdot[B]_{\sim_{\Br_{K}}}$ have the same image under base change.
   \leanok
   \lean{BrauerGroupHom.BaseChange}
 \end{construction}
 Denote the base change morphism in~\cref{con:br-base-change} as $\Br_{K}^{E}$.
 \begin{lemma}
   $\Br_{K}^{K}$ is identity.
   \leanok
 \end{lemma}
 \begin{proof}
   If $A\in\CSA$, then $A\sim_{\Br}K\ox_{K}A$.
 \end{proof}

 \begin{lemma}\label{lem:br-base-change-self}
   Consider the tower of field extension $E/F/K$,
   \[
     \Br_{K}^{E} = \Br_{E}^{F}\circ\Br_{K}^{E}.
   \]
 \end{lemma}

 \begin{proof}
   If $A\in\CSA_{K}$, then $E\ox_{F}\left(F\ox_{K}A\right)$ is isomorphic to $E \ox_{F} A$ as $E$-algebras.
   \leanok
   \lean{BrauerGroupHom.baseChange_idem}
 \end{proof}

 \begin{corollary}\label{lem:br-base-change-tower}
   $\Br$ forms a functor from category of field to category of abelian groups.
   \leanok
   \lean{BrauerGroupHom.Br}
 \end{corollary}
 \begin{proof}
   This is the categorical version of~\cref{lem:br-base-change-self} and~\cref{lem:br-base-change-tower}.
 \end{proof}

 \begin{definition}[Relative Brauer Group]
   Let $E/K$ be a field extension, we define the relative Brauer group $\Br(E/K)$ to be the kernel of the base change morphism $\Br_{K}^{E}$.
 \end{definition}

 \begin{remark}
   Unpacking the definition of the relative Brauer group, we see that for any $A \in \CSA_{K}$, if $E\ox_{K}A\cong\mat_{n}(E)$ as $E$-algebras, then $\br^{E}_{K}\left([A]_{\sim_{\br}}\right)=1$.
 \end{remark}

 \begin{definition}[Splitting Field]
   For any field extension $E/K$ and any $K$-algebra $A$, we say $E$ is a splitting field of $A$ if and only if $E\ox_{K}A \cong \mat_{n}(E)$ as $E$-algebras for some non-zero $n$. We also say $E$ splits $A$ or $A$ is splited by $E$
   \leanok
   \lean{isSplit}
 \end{definition}

  \begin{theorem}
   \label{thm:split-iff-mem-relative}
   Let $E/K$ be a field extension and $A\in\csa_{K}$, $E$ splits $A$ if and only if $[A]_{\sim_{\br}}\in\br(E/K)$.
   \leanok
   \lean{BrauerGroup.split_iff}
 \end{theorem}

 \begin{proof}
   The ``only if'' part is by definition. For the other direction, we know by definition that $\mat_{n}(E\ox_{K}A)\cong\mat_{m}(E)$ as $E$-algebras for some non-zero $m, n$. By~\cref{thm:wed-artin-algebra}, we find some division algebra $D$ and non-zero natural number $p$ such that $E\ox_{K}A\cong \mat_{p}(D)$ as $E$-algebras. Thus $\mat_{pm}(E)\cong\mat_{pn}\left(E\ox_{K}A\right)\cong\mat_{p^{2}n}(D)$ as $E$-algebras. By~\cref{thm:wed-artin-uniq}, we conclude that $E\cong D$ as $E$-algebras. Hence $E\ox_{K}A\cong \mat_{p}(E)$, in another word, $E$ splits $A$.
 \end{proof}

 \begin{remark}
   In light of~\cref{lem:br-base-change-tower}, if $K$ is algebraic closed then $K$ splits any $K$-algebra $A$. Indeed, $K$ splits $A$ if and only if $[A]_{\sim\br}$ but $[A]_{\sim\br}$ is equal to $1$.
 \end{remark}

 \begin{remark}
   If two $\csa_{K}$ are Brauer equivalent, in another word, $A \sim_{\br_{K}} B$, then $E$ splits $A$ if and only if $E$ splits $B$. Indeed, if $A$ and $B$ are equivalent, then $[A]_{\sim\br} \in \br(E/K)$ if and only if $[B]_{\sim\br}\in\br(E/K)$.
 \end{remark}


 \section{Good Representative Lemma}
 In this section, let $K/F$ be a finite dimensional field extension.

 \begin{lemma}
   \label{lem:good-rep-inv}
   Let $A\in\csa_{F}$ splitted by $K$. There exists a $B\in\csa_{F}$ such that
   \begin{itemize}
     \item $[A]_{\sim_{\br}}[B]_{\sim_{\br}}=1$
     \item there exists $F$-algebra map $K\hookrightarrow B$
     \item ${\left(\dim_{F}K\right)}^{2}=\dim_{F}B$.
   \end{itemize}
   \leanok
   \lean{exists_embedding_of_isSplit}
 \end{lemma}

 \begin{proof}
   Since $K$ splits $A$, we find a non-zero natural number $n$ such that $K\ox_{F}A\cong\mat_{n}K\cong\End_{K}\left(K^{n}\right)$ as $K$-algebras.
   We define an $F$-algebra map $\iota: A\to\End_{F}\left(K^{n}\right)$ by
   \begin{center}
     \begin{tikzcd}
       A \arrow{r} & K \ox_{F} A \arrow{r}{\cong} & \End_K\left(K^n\right) \arrow{r}{|_{F}} & \End_F\left(K^n\right)
     \end{tikzcd},
   \end{center}
   where $|_{F}$ is restriction of scalars. Since $A$ is simple, $\iota$ is injective, therefore $A \cong \iota(A)$ as $F$-algebras. Define $B$ as $C_{\End_{F}\left(K^{n}\right)}(\iota(A))$, the centralizer of the range of $\iota$ in $\End_{F}\left(K^{n}\right)$.
   We construct an embedding $K \hookrightarrow B$ by $r \mapsto (r \cdot \bullet)$

   $B$ is a central $F$-algebra: if $x \in Z(B)$, then $x \in \iota(A)$ because by~\cref{thm:double-centralizer}, it is sufficient to prove that $x$ is in $C_{\End_{F}\left(K^{n}\right)}\left(B\right)$ which follows from the fact that $x \in Z(B)$. In fact, $x \in Z(\iota(A))$: suppose $a \in A$, we need to check $x \cdot \iota(a) = \iota(a)\cdot x$, this is the case because $B$ is defined as the centralizer of $\iota(A)$. Since $\iota(A) \cong A$ as $F$-algebras, $\iota(A)$ is $F$-central, hence $x \in F$.

   $B$ is a simple ring: by~\cref{lem:simple-centralizer}, it is sufficient to prove that $\iota(A)$ is a simple ring which comes from $A \cong \iota(A)$ as $F$-algebras.

   By~\cref{cor:self-tensor-centralizer}, we have $F$-algebra isomorphism $\End_{F}\left(K^{n}\right)\cong \iota(A)\ox_{F}B \cong A \ox_{F}B$. Since $\End_{F}\left(K^{n}\right)\cong\mat_{\dim_{F}\left(K^{n}\right)}\left(F\right)$ as $F$-algebras, we see that $[A]_{\sim_{\br}}$ and $[B]_{\sim_{\br}}$ are inverses.

   By~\cref{lem:dim-centralizer}, $\dim_{F}B\cdot \dim_{F}\iota(A) = \dim_{F}B\cdot\dim_{F}A = \dim_{F}\End_{F}\left(K^{n}\right) = {\left(\dim_{F}\left(K^{n}\right)\right)}^{2}={\left(\dim_{F}K\cdot \dim_{K}\left(K^{n}\right)\right)}^{2}=n^{2}\cdot{\left(\dim_{F}K\right)}^{2}$. On the other hand, since $K\ox_{F}A\cong\mat_{n}K$, we have $\dim_{F}K\ox_{F}A=\dim_{F}K\cdot\dim_{F}A=\dim_{F}\mat_{n}K=\dim_{F}K\dim_{K}\mat_{n}K=n^{2}\dim_{F}K$. Since $\dim_{F}K\ne 0$ , we conclude $\dim_{F}A=n^{2}$. Since $n\ne 0$ and $\dim_{F}(B)\cdot \dim_{F}(A)=n^{2}\dim_{F}(B)=n^{2}{\left(\dim_{F}K\right)}^{2}$, we get the desired result.

 \end{proof}

 \begin{corollary}\label{cor:good-rep}
   Let $A\in\csa_{F}$ splitted by $K$. There exists a $B \in \csa_{F}$ such that
   \begin{itemize}
     \item $[B]_{\sim_{\br}}=[A]_{\sim_{\br}}$
     \item there exists an $F$-algebra map $K\hookrightarrow B$
     \item  ${\left(\dim_{F}K\right)}^{2}=\dim_{F}B$.
  \end{itemize}
 \end{corollary}

 \begin{proof}
   Let $B$ and $\iota : K \hookrightarrow B$ be as in~\cref{lem:good-rep-inv}. Consider $B^{\opp}$ and $K \hookrightarrow B \to B^{\opp}$. This works.
 \end{proof}

 \begin{theorem}\label{thm:good-rep-iff-split}
   Let $A\in\csa_{F}$. $K$ splits $A$ if and only if there exists a $B\in\csa_{F}$ such that
   \begin{itemize}
     \item $[B]_{\sim_{\br}}=[A]_{\sim_{\br}}$
     \item there exists an $F$-algebra map $K \hookrightarrow B$
     \item ${\left(\dim_{F}K\right)}^{2}=\dim_{F}B$.
   \end{itemize}
   \leanok
   \lean{isSplit_iff_dimension}
 \end{theorem}
 \begin{proof}
   The ``if'' direction is~\cref{cor:good-rep}. For the ``only if'' direction, let $B \in\csa_{F}$ and $\iota : K \hookrightarrow B$ be given. We give $B$ a $K$-module structure by right multiplication, that is for any $a \in K$ and $b \in B$, we define $a \cdot b := b \cdot \iota(a)$.
   Since $B$ is a finite dimensional $F$-vector space and $K/F$ is a finite dimensional field extension, $B$ is a finite dimensional $K$-vector space as well. Since $[B]_{\sim_{\br}}=[A]_{\sim_{\br}}$, it is sufficient to show that $K$ splits $B$.
   We define an $F$-bilinear map $\mu : K \to B \to \End_{K}B$ by $(c, a) \mapsto (c \cdot a \cdot \bullet)$ which induce an $F$-linear map $\mu' : K \ox_{F} B \to \End_{K}B$. Since for any $r, c\in K$ and $a \in B$, we have $\mu'\left(r \cdot c\ox a\right)(a') = a a' \iota(rc)=aa'\iota(c)\iota(r)=r\cdot \mu'(c\ox a)$, that is $\mu'$ is $K$-linear as well. Note that \[\mu'(1)=\mu'(1\ox 1) = (1\cdot 1\cdot \bullet) = 1\] and that
   \[
     \begin{aligned}
       \mu'(c\ox a \cdot c'\ox a')(a'') &= \mu'(cc'\ox aa')(a'') \\
                                        &= cc' \cdot aa' \cdot a'' \\
                                        &= aa' a'' \iota(c c') \\
                                        &= a(a' a'' \iota(c'))\iota(c) \\
                                        &=\mu'(c \ox a)(a' a'' \iota(c'))\\
                                        &=\mu'(c\ox a)\left(\mu'(c'\ox a')(a'')\right) \\
                                        &= \left(\mu'(c\ox a)\circ\mu'(c'\ox a')\right)(a'')
     \end{aligned},
     \]
     that is, $\mu'$ is an $K$-algebra map.

   If we can show that $\mu'$ is a bijection, we will prove the result for $K\ox_{F} B \cong \End_{K} B \cong \mat_{\dim_{K} B}K$ as $K$-algebras. By~\cref{cor:alghom-bijective-of-dim-eq}, it is sufficient to show $\dim_{K}K\ox_{F}B = \dim_{K}\End_{K}B$. Let $n$ denote $\dim_{F}K$. Since, $\dim_{F}K\dim_{K}K\ox_{F}B =\dim_{F}K\ox_{F}B=\dim_{F}K \dim_{F}B$. we have $\dim_{K}K\ox_{F}B = \dim_{F}B = {\left(\dim_{F}K\right)}^{2}$. On the other hand, since ${\left(\dim_{F}K\right)}^{2}=\dim_{F} B = \dim_{F}K\dim_{K}B$, we have $\dim_{K} B = \dim_{F}K$; thus $\dim_{K}\End_{K} B = {\left(\dim_{K}B\right)}^{2}={\left(\dim_{F}K\right)}^{2}$ and the result is proved.
 \end{proof}

 In light of~\cref{thm:good-rep-iff-split}, we isolate the following useful definition:
 \begin{definition}[Good Representation]\label{def:good-rep}
   For any $X \in \br(F)$, a $K$-good representation of $X$ is an $A \in \csa_{F}$ and an $F$-algebra map $K \hookrightarrow A$ which we often denote as $\iota$ or $\iota_{A}$ such that $[A]_{\sim_{\br}} = X$ and $\dim_{F}A={\left(\dim_{F}K\right)}^{2}$.
   \leanok
   \lean{GoodRep}
 \end{definition}

 When $K$ is clear from context, we will simply say good representation instead of $K$-good representation

 \begin{corollary}
   \label{cor:mem-relative-br-iff-good-rep}
   For any $X\in\br(F)$, $X\in\br(K/F)$ if and only if $X$ admits a good representation.
   \leanok
   \lean{mem_relativeBrGroup_iff_exists_goodRep}
 \end{corollary}
 \begin{proof}
   Rephrase of~\cref{thm:good-rep-iff-split} and~\cref{thm:split-iff-mem-relative}.
 \end{proof}

\subsection{Basic Properties}
We observe the following easy result about good representations. Let $X \in \br(F)$ and $A$ be a good representation of $X$.

\begin{lemma}
  The range $\iota_{A}(A)$ is a simple ring.
  \leanok
  \lean{GoodRep.ιRange}
\end{lemma}

\begin{proof}
  Because $K$ is a simple ring, $\iota_{A}$ is injective therefore $\iota_{A}(A)\cong K$.
\end{proof}

\begin{lemma}\label{lem:centralizer-range-good-rep}
  $C_{A}\left(\iota_{A}(A)\right) = \iota_{A}(A)$.
  \leanok
  \lean{GoodRep.centralizerιRange}
\end{lemma}
\begin{proof}
  In the language of~\cref{sec:subfield}, $\iota_{A}(A)$ is a subfield of $A$, hence by~\cref{lem:tfae-subfield}, we only need to show $\dim_{F}A = {\left(\dim_{F}\iota_{A}(A)\right)}^{2}$. But $\dim_{F}A={\left(\dim_{F}K\right)}^{2}$ and $\iota(A)\cong K$.
\end{proof}

\begin{construction}\label{con:good-rep-mod}
  We give $A$ a $K$-module structure by {\em left} multiplication, that is for any $c \in K$ and $a \in A$, we define $c\cdot a$ to be $\iota_{A}(c)a$. Then $A$ is a finite dimensional $K$-vector space and $\dim_{K}A=\dim_{F}K$: indeed $\dim_{F}K\cdot\dim_{K}A = \dim_{F}K\cdot \dim_{F}K = \dim_{F}A$.
  \leanok
  \lean{GoodRep.dim_eq'}
\end{construction}

\begin{lemma}\label{lem:good-rep-iso}
  If $A$ and $B$ are two good representations of $X$, then $A \cong B$ as $F$-algebras.
  \leanok
  \lean{GoodRep.exists_iso}
\end{lemma}
\begin{proof}
  By~\cref{lem:common-div-alg}, we find a division $F$-algebra $D$ and non-zero natural numbers $m, n$ such that $A \cong \mat_{m}(D)$ and $B \cong \mat_{n}(D)$ as $F$-algebras. Therefore
  \[
    \begin{aligned}
      {\left(\dim_{F}K\right)}^{2} &= \dim_{F}A &= m^{2}\dim_{F}D \\
      &= \dim_{F}B &= n^{2}\dim_{F}D.
    \end{aligned}
  \]
  Therefore $m = n$ and $A \cong \mat_{m}D =\mat_{n}D\cong B$.
\end{proof}

\subsection{Conjugation Factors and Conjugation Sequences}
In this section, let $K/F$ be a field extension, $X \in \br(F)$ and $A$ be a $K$-good representation of $X$.
\begin{remark}
  Since $\gal(K/F)$ acts on $K^{\star}$, for $x\in K^{\star}$,we feel free to write $\sigma\cdot x$ when it feels more readable than $\sigma(x)$, for example when there are nested brackets.
\end{remark}


\begin{definition}[Conjugation Factor]
  \label{def:conj-factor}
  With respect to $A$, a conjugation factor of $\sigma$ is a unit $x_{\sigma} \in A^{\star}$ such that for all $c \in K$,
  \[
    x_{\sigma} \iota_{A}(c)x_{\sigma}^{-1} = \iota_{A}(\sigma \cdot c).
  \]

  A conjugation sequence is a sequence $x:\gal(K/F) \to A^{\star}$ such that for all $\sigma\in\gal(K/F)$, $x_{\sigma}$ is a conjugation factor of $\sigma$. When we want to stress $A$, we say $A$-conjugation factor and $A$-conjugation sequence.
  \leanok
  \lean{GoodRep.conjFactor}
\end{definition}

\begin{remark}\label{rem:conj-factor-alternative-eq}
  When $x_{\sigma}$ is a conjugation factor of $\sigma$, the equalities $x_{\sigma}\iota_{A}(c) = x_{\sigma}\iota_{A}(\sigma(c))$ and $\iota_{A}(c)x_{\sigma}^{-1}=x_{\sigma}^{-1}\iota_{A}(\sigma(c))$ are also useful.
\end{remark}

\begin{construction}\label{con:exists-conj-seq}
  $A$ has a conjugation sequence:
  let $\sigma \in \gal(K/F)$, we have two $F$-algebra homomorphisms $K \to A$ given by $\iota_{A}$ and $\iota_{A}\circ \sigma$. Applying~\cref{thm:skolem-noether} to $\iota_{A}$ and $\iota_{A}\circ \sigma$ gives us the desired conjugation factor.
  \leanok
  \lean{GoodRep.aribitaryConjFactor}
\end{construction}

\begin{construction}
  If $x$ is a conjugation factor of $\sigma$ and $y$ of $\tau$, then $xy$ is a conjugation factor of $\sigma\tau$. For any $c \in K$
  \[
    \iota_{A}(\sigma \cdot\tau(c)) = x\iota_{A}(\tau\cdot c)x^{-1} = xy\iota_{A}(c)y^{-1}x^{-1}=\left(xy\right)\iota_{A}{\left(xy\right)}^{-1}.
  \]
  \leanok
  \lean{GoodRep.mul'}
\end{construction}

\begin{theorem}
  If $x$ is an $A$-conjugation sequence, then $\{x_{\sigma}|\sigma\in \gal(K/F)\}$ is an $K$-linearly independent set. When $K/F$ is finite dimensional and Galois, $\{x_{\sigma}|\sigma\in \gal(K/F)\}$ is a $K$-basis for $A$.
  \leanok
  \lean{GoodRep.conjFactor_linearIndependent, GoodRep.conjFactorBasis}
\end{theorem}

\begin{proof}
  Suppose $\{x_{\sigma}\}$ is linearly dependent. Let $J \subseteq \gal(K/F)$ be such that $\{x_{\sigma}|\sigma\in J\}$ is a maximally linearly independent subset.
  Then $J \ne \gal(K/F)$, let $\sigma \in\gal(K/F)$ be an arbitrary automorphism that is not in $J$. Since $\{x_{\tau}|\tau\in J\}$ is maximally linearly independent, $\sigma \in \left\langle x_{\tau}|\tau\in J\right\rangle$. Hence, by~\cref{con:good-rep-mod} we have
  \[
    x_\sigma = \sum_{\tau\in J}\lambda_{\tau}\cdot x_{\tau} = \sum_{\tau\in J}\iota_{A}\left(\lambda_{\tau}\right)x_{\tau},
  \]
  for some $\lambda_{\tau}\in K$.
  For each $c \in K$, we have the following equality
  \[
    \begin{aligned}
      \iota_{A}\left(\sigma \cdot c\right) x_{\sigma}
      &= x_{\sigma} \iota_{A}(c) &\text{by~\cref{def:conj-factor}}\\
      &= \sum_{\tau\in J}\lambda_{\tau}\cdot x_{\tau}\iota_{A}(c) & \\
      &= \sum_{\tau\in J}\lambda_{\tau}\cdot \iota_{A}\left(\tau\cdot c\right)x_{\tau} &\text{by~\cref{def:conj-factor} again}\\
      &=\sum_{\tau\in J}\iota_{A}\left(\lambda_{\tau}\tau(c)\right)x_{\tau};\\
      \iota_{A}\left(\sigma\cdot c\right)x_{\sigma}
      &= \sum_{\tau\in J}\iota_{A}\left(\lambda_{\tau}\right) x_{\tau}\iota_{A}(c)\\
      &= \sum_{\tau\in J}\iota_{A}\left(\sigma(c)\lambda_{\tau}\right)x_{\tau}.
    \end{aligned}
  \]
  Since $\{x_{\tau}|\tau\in J\}$ is linearly independent, we have that for each $\tau \in J$, $\lambda_{\tau}\tau(c) = \sigma(c)\lambda_{\tau}$. Note that $J$ is not empty, for otherwise $x_{\sigma}=\sum_{\tau\in\emptyset}\lambda_{\tau}\cdot x_{\tau}=0$ but $x_{\sigma}$ is invertible. Hence $\sigma$ is in $J$ after all; contradiction.

  If $K/F$ is finite dimensional and Galois, then $\dim_{F}K$ is equal to the cardinality of $\gal(K/F)$, then by the linear independence of $\{x_{\sigma}|\sigma\in\gal(K/F)\}$, we conclude that it is indeed a $K$-basis for $A$.
\end{proof}

\section{The Second Galois Cohomology}

In this section, we construct a group isomorphism between $\br(K/F) \cong \HH^{2}\left(\gal(K/F), K^{\star}\right)$ where $K/F$ is a finite dimensional Galois extension. To keep alignment of the Brauer group, let us use the multiplicative notation for group cohomology. Recall:

\begin{definition}[the Second Group Cohomology]
  \label{def:second-group-coh}
  Let $G$ be a group and $M$ an abelian group (written multiplicatively) with a $G$-action.

  A function $f : G \times G \to M$ is a {\em 2-cocycle\/} if for all $g,h,j\in G$,
  \[
    f(gh, j)f(g, h) = \left(g\cdot f(h, j)\right) f(g, hj).
  \]
  We denote the subgroup of 2-cocycles as $\mathcal{Z}^{2}(G, M)$.

  A function $f : G\times G \to M$ is a {\em 2-coboundary\/} if there exists an $x : G \to M$ such that for all $g, h \in G$
  \[
    \frac{g\cdot x(h)}{x(gh)} x(g) = f(g, h).
  \]
  We denote the subgroup of 2-coboundaries as $\mathcal{B}^{2}(G, M)$.

  The second group cohomology $\hh^{2}\left(G, M\right)$ is defined to be the quotient group of 2-cocycles modulo 2-coboundaries $^{\mathcal{Z}^{2}(G, M)}/_{\mathcal{B}^{2}(G, M)}$. If $s, t \in \mathcal{Z}^{2}(G, M)$, we say $s$ and $t$ are cohomologous if their equivalence class $[s], [t] \in \HH^{2}(G, M)$ are the same; in another word $st^{-1}\in\mathcal{B}^{2}(G, M)$.
  \leanok
  \lean{groupCohomology.H2, groupCohomology.IsMulTwoCoboundary, groupCohomology.IsMulTwoCocycle}
\end{definition}

\begin{lemma}\label{lem:2-cocycle-one-one}
  If $f \in \mathcal{B}^{2}(G, M)$ is a 2-cocycle and $x\in G$, we have
  \[
    \begin{aligned}
    f(1, x) &= f(1, 1) \\
    f(x, 1) &= x\cdot f(1, 1).
    \end{aligned}
  \]
\end{lemma}
\begin{proof}
  Indeed:
  \[
    \begin{aligned}
      f(1\cdot 1, x)f(1, 1) &= (1\cdot f(1,x))f(1, 1\cdot x)  \\
      f(1, x)f(1, 1) &= f(1, x) f(1,x) \\
      f(1, x) &= f(1,1)
    \end{aligned}
  \] and
  \[
    \begin{aligned}
      f(x\cdot 1, 1)f(x, 1) &= (x\cdot f(1, 1))f(x, 1\cdot 1)\\
      f(x, 1)f(x, 1) &= (x\cdot f(1, 1))f(x, 1)\\
      f(x, 1) &= x \cdot f(1, 1).
    \end{aligned}
  \]
  \leanok
  \lean{GoodRep.CrossProduct.a_one_left, GoodRep.CrossProduct.a_one_right}
\end{proof}

In the following sections of this chapter, we assume that $X \in\br(F)$ and $A$ is a good representation of $X$. We use $\rho, \sigma, \tau$ to denote elements of $\gal(K/F)$. To improve typographic aesthetics of our proofs, we sometimes use subscript to mean function application.


\subsection{From $\br(K/F)$ to $\hh^{2}\left(\gal(K/F),K^{\star}\right)$}

\begin{lemma}[Twisting Conjugation Factors]\label{lem:twist-spec1}
  If $x$ and $y$ are two conjugation factors of $\sigma$, then there exists a unique $c \in K$ such that $x = y\iota_{A}(c)$.
  \leanok
  \lean{GoodRep.conjFactor_rel, GoodRep.conjFactorTwistCoeff}
\end{lemma}

\begin{proof}
  The uniqueness is clear: suppose $x = y\iota_{A}(c)=y\iota_{A}(c')$, then $c = c'$ because $x, y$ are units and $\iota_{A}$ is injective. We first observe that $y^{-1}x\in C_{A}(\iota(A))$: for any $z \in K$, $y^{-1}x\iota_{A}(z)=y^{-1}\iota_{A}(\sigma(z))x=\iota_{A}(z)y^{-1}x$ (by~\cref{rem:conj-factor-alternative-eq}). By~\cref{lem:centralizer-range-good-rep}, $y^{-1}x\in \iota(A)$, that is for some $z \in K$, we have that $y^{-1}x = \iota_{A}(z)$ and the claim is proved.
\end{proof}

We denote such $c$ by $\twist^{\sigma}(x, y)$ or $\twist^{\sigma}_{x, y}$, when $\sigma$ is clear from context, we often omit the superscript. With this notation, $x = y\iota_{A}(\twist_{x,y})$.

\begin{remark}
  $\twist(x, x)$ is equal to $1$ by uniqueness.
\end{remark}

\begin{remark}
  In fact, $\twist(x, y)$ is in $K^{\star}$ and $\twist(x, y)^{-1}=\twist(y, x)$.
\end{remark}

\begin{lemma}\label{lem:twist-spec2}
  If $x$ and $y$ are conjugation factors for $\sigma$, $x=\iota_{A}(\sigma(\twist_{x, y}))y$.
  \leanok
  \lean{GoodRep.conjFactorTwistCoeff_spec'}
\end{lemma}
\begin{proof}
  \[
    \begin{aligned}
      x &= x \iota_{A}\left(\twist_{x,y}\right) x^{-1} x \iota_{A}(\twist_{y,x}) \\
        &= \iota_{A}\left(\sigma\cdot\twist_{x,y}\right)x\iota_{A}\left(
          \twist_{y,x}\right)\\
      &=\iota_{A}\left(\sigma\cdot \twist_{x,y}\right)y
    \end{aligned}.
  \]
\end{proof}

\begin{construction}[Comparing Conjugation Factors]\label{con:compare-conj-factors}
  Let $x$ be a conjugation factor for $\sigma$, $y$ for $\tau$ and $z$ for $\sigma\tau$. Since $xy$ is a also a conjugation factor, we define the comparison coefficient to be $\comp^{\sigma, \tau}_{x,y,z}:=\sigma\left(\tau\left(\twist_{xy, z}\right)\right)$. We often omit superscript when the context is clear. Note that $\comp_{x,y,z}$ is a unit in $K$ with inverse $\sigma\left(\tau\left(\twist_{z, xy}\right)\right)$. By~\cref{lem:twist-spec1} and~\cref{lem:twist-spec2}, we have the following useful equalities
  \[
    \begin{aligned}
      xy &= \iota_{A}\left(\comp_{x,y,z}\right)z \\
      \iota_{A}\left(\comp_{x,y,z}^{-1}\right)xy &= z \\
      \iota_{A}\left(\comp_{x,y,z}\right) &= xyz^{-1}\\
      \iota_{A}\left(\comp_{x,y,z}^{-1}\right) &= zy^{-1}x^{-1}\\
      \dots &=\dots
    \end{aligned}.
  \]
  %If we have a conjugation sequence $x : \gal(K/F)\to A^{\star}$, instead of $\comp_{x_{\sigma},x_{\tau},x_{\sigma\tau}}^{\sigma,\tau}$, we simply write $\comp_{\sigma,\tau}^{x}$.
  \leanok
  \lean{GoodRep.conjFactorCompCoeff}
\end{construction}

\begin{lemma}\label{lem:compare-conj-factor-spec}
  Let $x : \gal(K/F) \to A^{\star}$ be a conjugation sequence. We have
  \[
    \comp_{x_{\rho}, x_{\sigma}, x_{\rho\sigma}}\,\comp_{x_{\rho\sigma},x_{\tau},x_{\rho\sigma\tau}} =
    \rho\left(\comp_{x_{\sigma},x_{\tau},x_{\sigma\tau}}\right)\,\comp_{x_{\rho}, x_{\sigma\tau}, x_{\rho\sigma\tau}}.
  \]
  \leanok
  \lean{GoodRep.conjFactorCompCoeff_comp_comp}
\end{lemma}
\begin{proof}
  It is sufficient to make the following calculations:
  \begin{align}
   x_{\rho}\,x_{\sigma}x_{\tau} &= \iota_{A}\left(\comp_{x_{\rho},x_{\sigma},x_{\rho\sigma}}\right) \, \iota_{A}\left(\comp_{x_{\rho\sigma}, x_{\tau}, x_{\rho\sigma\tau}}\right)\, x_{\rho\sigma\tau}\label{eq:1}\\
   x_{\rho}\left(x_{\sigma}x_{\tau}\right) &= \iota_{A}\left(\rho\cdot\comp_{x_{\sigma},x_{\tau},x_{\sigma\tau}}\right)\,\iota_{A}\left(\comp_{x_{\rho},x_{\sigma\tau},x_{\rho\sigma\tau}}\right) x_{\rho\sigma\tau} \label{eq:2}
 \end{align}
 Then since $x_{\rho\sigma\tau}$ is invertible and $\iota_{A}$ is injective, we proved the desired result.

 \Cref{eq:1} is because: by the first equality in~\cref{con:compare-conj-factors} (twice)
 \[
   x_{\rho}x_{\sigma}x_{\tau}=\iota_{A}\left(\comp_{x_{\rho},x_{\sigma}, x_{\rho\sigma}}\right)x_{\rho\sigma}x_{\tau}=\iota_{A}\left(\comp_{x_{\rho},x_{\sigma},x_{\rho\sigma}}\right)\iota_{A}\left(\comp_{x_{\rho\sigma},x_{\tau},x_{\rho\sigma\tau}}\right)x_{\rho\sigma\tau}.
 \]
 \Cref{eq:2} is because: by~\cref{def:conj-factor}, we have
 \[\iota_{A}\left(\rho\cdot\comp_{x_{\sigma},x_{\tau},x_{\sigma\tau}}\right)x_{\rho}=x_{\rho}\iota_{A}\left(\comp_{x_{\sigma},x_{\tau},x_{\sigma\tau}}\right),\]
 therefore by~\cref{con:br-base-change}
 \[
   \begin{aligned}
     x_{\rho}\left(x_{\sigma}x_{\tau}\right) &= x_{\rho}\iota_{A}\left(\comp_{x_{\sigma},x_{\tau},x_{\sigma\tau}}\right)x_{\sigma\tau} \\
                                             &= \iota_{A}\left(\rho\cdot \comp_{x_{\sigma},x_{\tau},x_{\sigma\tau}}\right)x_{\rho}x_{\sigma\tau}\\
     &= \iota_{A}\left(\rho\cdot\comp_{x_{\sigma},x_{\tau},x_{\sigma\tau}}\right)\,\iota_{A}\left(\comp_{x_{\rho},x_{\sigma\tau},x_{\rho\sigma\tau}}\right)x_{\rho\sigma\tau}
   \end{aligned}.
 \]
\end{proof}

\begin{construction}[from good representation to 2-cocycle]
  \label{con:good-rep-to-2-cocycles}
  Let $x$ be an $A$-conjugation sequence. We associate with $x$ a function $\mathcal{B}^{2}(x) : \gal(K/F)\times\gal(K/F)\to K^{\star}$ defined by
  \[
    (\sigma,\tau)\mapsto \comp_{x_{\sigma},x_{\tau},x_{\sigma\tau}}.
  \]
  We will write $\mathcal{B}^{2}(x)$ as $\mathcal{B}^{2}_{A, x}$, $\mathcal{B}^{2}_{A}(x)$ or $\mathcal{B}^{2}_{x}$ as well.
  \leanok
  \lean{GoodRep.toTwoCocycles}
\end{construction}

\begin{lemma}
  For any $A$-conjugation sequence $x$, $\mathcal{B}^{2}_{x} \in \mathcal{B}^{2}\left(\gal(K/F), K^{\star}\right)$, that is $\mathcal{B}_{x}$ is indeed a 2-cocycle.
  \leanok
  \lean{GoodRep.isTwoCocyles}
\end{lemma}

\begin{proof}
  We need to prove
  \[
    \mathcal{B}_{x}(\rho\sigma, \tau)\, \mathcal{B}_{x}(\rho,\sigma) = \rho\left(\mathcal{B}_{x}(\sigma, \tau)\right)\, \mathcal{B}_{x}(\rho, \sigma\tau).
  \]
  But this is exactly~\cref{lem:compare-conj-factor-spec}.
\end{proof}

For any good representation $A$ of $X \in \br(K/F)$ and any $A$-conjugation sequence $x$, we have constructed a 2-cocycle $\mathcal{B}^{2}_{A}(x)$. But to obtain a well-defined function from $\br(K/F)$ to $\HH^{2}\left(\gal(K/F), K^{\star}\right)$, we need to verify that for any other good representation $B$ of $X$ and $B$-conjugation sequence $y$, $\mathcal{B}^{2}_{A}(x)$ and $\mathcal{B}^{2}_{B}(y)$ are cohomologous. Let us fix another good representation $B$ of $X \in \br(K/F)$ and a $B$-conjugation sequence $y$.

\begin{construction}\label{con:good-rep-iso-coeff}
  By~\cref{lem:good-rep-iso}, $A$ and $B$ are isomorphic as $F$-algebras, we use $e_{A,B}$ to denote an arbitrary $F$-algebra isomorphism between $A$ and $B$. When there is no confusion, we write $e$ instead of $e_{A, B}$ Since $e \circ \iota_{A}$ and $\iota_{B}$ are two $F$-algebra homomorphism from $K$ to $B$, by~\cref{thm:skolem-noether}, there exists some $u \in B^{\star}$ such that for all $r \in K$, we have $\iota_{B}(r) = ue\left(\iota_{A}(r)\right)u^{-1}$ (or equivalently, $u^{-1}\iota_{B}(r)u = e\left(\iota_{A}(r)\right)$). When there is confusion, we write $u_{A, B}$ instead of $u$.
\end{construction}

\begin{lemma}\label{lem:iso-conj-coeff-spec2}
  For any $c \in K$, $\sigma\in\gal(K/F)$ and $A$-conjugation factor $x$ of $\sigma$, we have
  \[
    \iota_{B}\left(\sigma\cdot c\right) = ue(x)u^{-1}\,\iota_{B}(c)\,ue\left(x^{-1}\right)u^{-1}.
  \]
  \leanok
  \lean{GoodRep.isoConjCoeff_spec''}
\end{lemma}
\begin{proof}
  From~\cref{def:conj-factor}, we have $e\left(\iota_{A}(\sigma \cdot c)\right) = e\left(x\iota_{A}(c)x^{-1}\right)$. Substituting it in~\cref{con:good-rep-iso-coeff}, we get
  \[
    \begin{aligned}
      \iota_{B}(\sigma \cdot c) &= ue\left(x\iota_{A}(c)x^{-1}\right)u^{-1}\\
                                &= ue(x)e\left(\iota_{A}(c)\right)e\left(x^{-1}\right)u^{-1}\\
      &= ue(x)u^{-1}\iota_{B}(c)ue\left(x^{-1}\right)u^{-1}.
    \end{aligned}
  \]
\end{proof}

\begin{construction}
  \label{con:push-forward-conj-factor}
  If $x$ is an $A$-conjugation factor for $\sigma$, we can obtain a $B$-conjugation factor for $\sigma$ by defining $B_{\star}x := ue(x)u^{-1}$ with inverse $ue\left(x^{-1}\right)u^{-1}$. We use~\cref{lem:iso-conj-coeff-spec2} to check that $B_{\star}x$ is indeed a conjugation factor for $\sigma$. If $y$ is a $B$-conjugation factor for $\sigma$, another useful constant is $v := \sigma\left(\twist_{y, B_{\star}x}\right)$. We have
  \[
    \begin{aligned}
    y &= \iota_{B}\left(v\right)\, B_{\star}x \\
      \iota_{B}\left(v\right) &= u e\left(\iota_{A}\left(v\right)\right)u^{-1}\\
      v^{-1} &= \sigma\left(\twist_{B_{\star}x, y}\right)
    \end{aligned}.
  \]
  We also write $v_{x,y}$ or even $v^{A,B}_{x, y}$ when we stress the importance of good representation $A$ and $B$ and their conjugation factor $x$ and $y$.
\end{construction}

\begin{lemma}\label{lem:compare-to-2-cocycle-aux}
  Let $x$ be an $A$-conjugation sequence and $y$ a $B$-conjugation sequence.
  We have
  \[
    \comp_{y_{\sigma}, y_{\tau}, y_{\sigma\tau}} v_{x_{\sigma\tau},y_{\sigma\tau}} = v_{x_{\sigma},y_{\sigma}} \sigma\left(v_{x_{\tau},y_{\tau}}\right) \comp_{x_{\sigma},x_{\tau},x_{\sigma\tau}}.
  \]
  \leanok
  \lean{GoodRep.compare_conjFactorCompCoeff}
\end{lemma}
\begin{proof}
  By~\cref{con:compare-conj-factors}, we have $y_{\sigma}y_{\tau} = \iota_{B}\left(\comp_{y_{\sigma},y_{\tau},y_{\sigma\tau}}\right)y_{\sigma\tau}$.
  By repeated application of~\cref{con:push-forward-conj-factor} and~\cref{con:good-rep-iso-coeff}, we have
  \[
    \begin{aligned}
      y_{\sigma\tau}&= u\,e\left(\iota_{A}\left(v_{x_{\sigma\tau},y_{\sigma\tau}}\right)x_{\sigma\tau}\right)\,u^{-1}\\
      y_{\sigma}y_{\tau} &= u\,e\left(\iota_{A}\left(v_{x_{\sigma},y_{\sigma}}\right)x_{\sigma}\iota_{A}\left(v_{x_{\tau},y_{\tau}}\right)x_{\tau}\right)\,u^{-1}\\
                    &= \iota_{B}\left(\comp_{y_{\sigma},y_{\tau},y_{\sigma\tau}}\right)y_{\sigma\tau}\\
                    &= \iota_{B}\left(\comp_{y_{\sigma},y_{\tau},y_{\sigma\tau}}\right)u\,e\left(\iota_{A}\left(v_{x_{\sigma\tau},y_{\sigma\tau}}\right)x_{\sigma\tau}\right)\,u^{-1}\\
                    &= u\,e\left(\iota_{A}\left(\comp_{y_{\sigma},y_{\tau},y_{\sigma\tau}}\right)\right)\,u^{-1}\,\,\,ue\left(\iota_{A}\left(v_{x_{\sigma\tau},y_{\sigma\tau}}\right)x_{\sigma\tau}\right)\,u^{-1}\\
      &= ue\left(\iota_{A}\left(\comp_{y_{\sigma},y_{\tau},y_{\sigma\tau}}v_{x_{\sigma\tau},y_{\sigma\tau}}\right)x_{\sigma\tau}\right)u^{-1}.
    \end{aligned}
  \]
  Hence
  \[
    \iota_{A}\left(v_{x_{\sigma},y_{\sigma}}\right)x_{\sigma}\iota_{A}\left(v_{x_{\tau},y_{\tau}}\right)x_{\tau} = \iota_{A}\left(\comp_{y_{\sigma},y_{\tau},y_{\sigma\tau}}v_{x_{\sigma\tau},y_{\sigma\tau}}\right)x_{\sigma\tau}.
  \]
  We also have by~\cref{def:conj-factor}
  \[
    x_{\sigma} \iota_{A}\left(v_{x_{\tau},y_{\tau}}\right) x_{\tau} =
    \iota_{A}\left(\sigma\cdot v_{x_{\tau}y_{\tau}}\right) x_{\sigma}x_{\tau}.
  \]
  Hence
  \[
    \begin{aligned}    \iota_{A}\left(v_{x_{\sigma},y_{\sigma}}\right)x_{\sigma}\iota_{A}\left(v_{x_{\tau},y_{\tau}}\right)x_{\tau} &=
                                                                                                                                      \iota_{A}\left(v_{x_{\sigma},y_{\sigma}}\, \sigma\left(v_{x_{\tau},y_{\tau}}\right)\right) x_{\sigma}x_{\tau} \\
      &= \iota_{A}\left(v_{x_{\sigma},y_{\sigma}}\sigma\left(v_{x_{\tau},y_{\tau}}\right)\comp_{x_{\sigma},x_{\tau},x_{\sigma\tau}}\right)x_{\sigma\tau}\\
      &= \iota_{A}\left(\comp_{y_{\sigma},y_{\tau},y_{\sigma\tau}}v_{x_{\sigma\tau},y_{\sigma\tau}}\right)x_{\sigma\tau}.
    \end{aligned}
  \]
  Cancelling $x_{\sigma\tau}$ and by injectivity of $\iota_{A}$, the result is proved.
\end{proof}

\begin{lemma}\label{lem:compare-to-2-cocylce}
  Let $x$ be an $A$-conjugation sequence and $y$ a $B$-conjugation sequence. We have
  \[
    \mathcal{B}^{2}_{B,y}(\sigma,\tau) v_{x_{\sigma\tau},y_{\sigma\tau}} =
    v_{x_{\sigma},y_{\sigma}} \sigma\left(v_{x_{\tau},y_{\tau}}\right) \mathcal{B}^{2}_{A,y}(\sigma,\tau).
  \]
  \leanok
  \lean{GoodRep.compare_toTwoCocycles}
\end{lemma}

\begin{proof}
  If we unfold~\cref{con:good-rep-to-2-cocycles}, we discover the lemma is saying exactly~\cref{lem:compare-to-2-cocycle-aux}.
\end{proof}

We finally arrive at our main conclusion for this section.
\begin{corollary}\label{cor:to-2-cocylce-wd}
  Let $x$ be an $A$-conjugation sequence and $y$ a $B$-conjugation sequence. $\mathcal{B}^{2}_{A,x}$ and $\mathcal{B}^{2}_{B,y}$ are 2-cohomologous.
  \leanok
  \lean{GoodRep.compare_toTwoCocycles'}
\end{corollary}
\begin{proof}
  By~\cref{def:second-group-coh}, we need to find a function $f : \gal(K/F)\to K^{\star}$ such that for all $\sigma, \tau\in\gal(K/F)$,
  \[
    \frac{\sigma\left(f(\tau)\right)}{f\left(\sigma\tau\right)}f(\sigma)=\frac{\mathcal{B}^{2}_{B,y}}{\mathcal{B}^{2}_{A,x}}.
  \]
  Let $f(\rho) := v_{x_{\rho},y_{\rho}}$, by~\cref{lem:compare-to-2-cocylce} we see the equality holds.
\end{proof}

\begin{construction}[from $\br(K/F)$ to $\HH^{2}\left(\gal(K/F), K^{\star}\right)$]\label{con:br-to-snd-coh}
  Let $X \in \br(K/F)$, by~\cref{cor:mem-relative-br-iff-good-rep}, $X$ admits a good representation $A$; by~\cref{con:exists-conj-seq}, $A$ admits a conjugation sequence $x$. We associate with $X$ an element $\HH^{2}(X) := \left[\mathcal{B}^{2}_{A,x}\right]$in $\HH^{2}\left(\gal(K/F),K^{\star}\right)$. By~\cref{cor:to-2-cocylce-wd}, for any other good representation $B$ and $B$-conjugation sequence $y$, we have $\left[\mathcal{B}^{2}_{A,x}\right]=\left[\mathcal{B}^{2}_{B,y}\right]$, hence we have a well-defined function $\HH^{2} : \br(K/F) \to \HH^{2}\left(\gal(K/F),K^{\star}\right)$.
  \leanok
  \lean{RelativeBrGroup.toSnd}
  \lean{RelativeBrGroup.toSnd_wd}
\end{construction}

\subsection{From $\HH^{2}\left(\gal(K/F), K^{\star}\right)$ to $\br(K/F)$}

Let $\mathfrak{a} \in \mathcal{B}^{2}\left(\gal(K/F), K^{\star}\right)$ be any 2-cocycle. In this section, we construct the {\em cross product\/} associated with $\mathfrak{a}$ which we prove to be $F$-central simple. Finally, we show that if $\mathfrak{a}, \mathfrak{b}\in\mathcal{B}^{2}\left(\gal(K/F), K^{\star}\right)$ are cohomologous, the cross products associated with $a$ and $b$ are Brauer equivalent.

\begin{construction}[Cross product]\label{con:cross-product}
  Denote $\cross_{\mathfrak{a}}$ to be $\gal(K/F) \to K$, i.e. functions from $\gal(K/F)$ to $K$. Notationally, elements of $\cross_{\mathfrak{a}}$ are sequences in $K$ indexed by $\gal(K/F)$; we denote $\Delta_{\sigma, c}$ to be the sequence with value $c$ at $\sigma$-th index and zero elsewhere. We give $\cross_{\mathfrak{a}}$ the usual zero, addition, negation, that is, we give $\cross_{\mathfrak{a}}$ the normal additive abelian group structure. Since for each $c \in \cross_{\mathfrak{a}}$,
  \[
    c = \sum_{\sigma\in\gal(K/F)}\Delta_{\sigma, c(\sigma)},
    \]
    it is often, if not always, sufficient to consider the special cases of $\Delta_{\sigma, c}$ and extend the result linearly.
  For multiplications, we define the result of multiplying $\Delta_{\sigma, c},\Delta_{\tau, d} \cross_{\mathfrak{a}}$ to be $\Delta_{\sigma\tau, c\sigma(d)\, \mathfrak{a}(\sigma,\tau)}$. Immediately, if either $c$ or $d$ is $0$, the result of multiplication is also zero. That is, for all $c \in \cross_{\mfa}$, we have $c\cdot 0=0\cdot c =0$. For any $r \in F$ and $\Delta_{\sigma, c} \in \cross_{\mathfrak{a}}$, we define $r \cdot \Delta_{\sigma, c}$ to be $\Delta_{\sigma, r\cdot c}$.
  \leanok
  \lean{GoodRep.crossProductMul, crossProductSMul, GoodRep.CrossProduct.addCommGroup}
\end{construction}

\begin{remark}
  When $K/F$ is infinite dimensional, the correct definition of $\cross_{\mathfrak{a}}$ is perhaps $\bigoplus_{\sigma\in\gal(K/F)}K$. But in Lean4, function type is easier to manipulate than direct sums. Since our scope is finite dimensional Galois extension, our definition is still accurate.
\end{remark}

\begin{lemma}
  The cross product $\cross_{\mathfrak{a}}$ is a ring with the multiplicative unit $\Delta_{\mathsf{id}, \mathfrak{a}(\mathsf{id}, \mathsf{id})^{-1}}$. The $F$-action on $\cross_{\mathfrak{a}}$ defined by $r \cdot \Delta_{\sigma, c} := \Delta_{\sigma, r\cdot c}$ makes it an $F$-algebra.
  \leanok
  \lean{GoodRep.CrossProduct.algebra}
\end{lemma}
\begin{proof}
  We verify the axioms of rings on elements of the form $\Delta_{\sigma, c}$. Let $\sigma,\tau,\rho\in\gal(K/F)$ and $a,b,c\in K$.
  \begin{itemize}
    \item associativity of multiplication. We need to check that $\Delta_{\sigma,a}\left(\Delta_{\tau,b}\Delta_{\rho,c}\right) = \left(\Delta_{\sigma,a}\Delta_{\tau,b}\right)\Delta_{\rho,c}$:
          \[
          \begin{aligned}
            \Delta_{\sigma, a}\left(\Delta_{\tau,b}\Delta_{\rho,c}\right)
            &= \Delta_{\sigma,a}\Delta_{\tau\rho,b\tau(c) \mathfrak{a}(\sigma,\tau)}\\
            &= \Delta_{\sigma\tau\rho,a\sigma(b)\sigma(\tau(c)) \sigma(\mathfrak{a}(\sigma, \tau))};\\
            %
            \left(\Delta_{\sigma,a}\Delta_{\tau,b}\right)\Delta_{\rho,c}
            &= \Delta_{\sigma\tau,a\sigma(b)\mathfrak{a}(\sigma,\tau)}\Delta_{\rho,c}\\
            &= \Delta_{\sigma\tau\rho,a\sigma(b)\mathfrak{a}(\sigma, \tau)\sigma(\tau(c))\mathfrak{a}(\sigma\tau, \rho)}.
          \end{aligned}
          \]
          Hence it is sufficient to check
          \[
          \sigma(\tau(c))\sigma(\mathfrak{a}(\sigma,\tau)) =
          \mathfrak{a}(\sigma,\tau)\sigma(\tau(c))\mathfrak{a}(\sigma\tau, \rho).
          \]
          This is the 2-cocycle condition in~\cref{def:second-group-coh} (modulo commutativity of $K$).
    \item multiplicative unit: we need to check $\Delta_{\sigma, a}\Delta_{\mathsf{id},\mathfrak{a}(\mathsf{id}, \mathsf{id})} = \Delta_{\mathsf{id},\mathfrak{a}(\mathsf{id},\mathsf{id})}\Delta_{\sigma, a} = \Delta_{\sigma, a}$. By multiple applications of~\cref{lem:2-cocycle-one-one}
          \[
          \begin{aligned}
            \Delta_{\mathsf{id},\mathfrak{a}(\mathsf{id}, \mathsf{id})^{-1}}\Delta_{\sigma, a}
            &= \Delta_{\sigma, \mathfrak{a}(\mathsf{id}, \mathsf{id})^{-1}a\mathfrak{a}(\mathsf{id},\sigma)}\\
            &= \Delta_{\sigma, \mathfrak{a}(\mathsf{id}, \mathsf{id}) a\mathfrak{a}(\mathsf{id},\mathsf{id})}\\
            &= \Delta_{\sigma, a} \\
            \Delta_{\sigma, a}\Delta_{\mathsf{id},\mathfrak{a}(\mathsf{id},\mathsf{id})^{-1}}
            &= \Delta_{\sigma, a\sigma(\mathfrak{a}(\id,\id))^{-1}\mathfrak{a}(\sigma,\id)}\\
            &=\Delta_{\sigma, a\sigma(\mfa(\id,\id))^{-1}\sigma(\mfa(\id, id))}\\
            &=\Delta_{\sigma, a}
          \end{aligned}
          \]
    \item distributivity: We need to check left-distributivity $\Delta_{\sigma, a}\left(\Delta_{\tau, b} + \Delta_{\rho, c}\right) = \Delta_{\sigma,a}\Delta_{\tau,b} + \Delta_{\sigma, a}\Delta_{\rho, c}$ and right distributivity $\left(\Delta_{\tau,b} + \Delta_{\rho,c}\right)\Delta_{\sigma,a} = \Delta_{\tau,b}\Delta_{\sigma,a} + \Delta_{\rho,c}\Delta_{\sigma,a}$. This is precisely what ``extend linearly'' means.
    \item $F$-algebra: We need to check for all $r \in F$, $\left(r\cdot \Delta_{\id,\mfa(\id,\id)^{-1}}\right)\Delta_{\sigma,c} = \Delta_{\sigma,c}\left(r\cdot\Delta_{\id,\mfa(\id,\id)^{-1}}\right)$. By~\cref{lem:2-cocycle-one-one}
          \[
          \begin{aligned}
            \left(r\cdot \Delta_{\id,\mfa(\id,\id)^{-1}}\right)\Delta_{\sigma,c}
            &= \Delta_{\id,r\cdot\mfa(\id,\id)^{-1}}\Delta_{\sigma,c}\\
            &= \Delta_{\sigma,\left(r\cdot\mfa(\id,\id)^{-1}\right)c\mfa(\id,\sigma)} \\
            &=\Delta_{\sigma,\left(r\cdot\mfa(\id,\id)^{-1}\right)c\mfa(\id,\id)} \\
            &= \Delta_{\sigma, r\cdot c}\\
            %
            \Delta_{\sigma,c}\left(r\cdot\Delta_{\id,\mfa(\id,\id)^{-1}}\right)
            &= \Delta_{\sigma,c}\Delta_{\id,r\cdot \mfa(\id,\id)^{-1}} \\
            &= \Delta_{\sigma,c\sigma\left(r\cdot\mfa(\id,\id)^{-1}\right)\mfa(\sigma,\id)}\\
            &=\Delta_{\sigma,c\left(r\cdot\sigma\left(\mfa(\id,\id)^{-1}\right)\right)\mfa(\id,\id)^{-1}}\\
            &= \Delta_{\sigma, c\left(r\cdot 1\right)}\\
            &=\Delta_{\sigma,r\cdot c}.
          \end{aligned}
          \]
  \end{itemize}
\end{proof}

From now on, we feel free to write $1 \in \cross_{\mfa}$ instead of $\Delta_{\id,\mfa(\id,\id)^{-1}}$. Then the algebra map $F \hookrightarrow \cross_{\mfa}$ is the map $r \mapsto r \cdot 1$.

\begin{construction}[$K$-embedding]\label{con:cross-product-iota}
  The map $\iota_{\cross_{\mfa}}: K \to \cross_{\mfa}$ defined by
  \[
    b \mapsto \Delta_{\id, b\mfa(\id,\id)^{-1}}
  \]
  is an $F$-algebra map.
  Checking that $\iota_{\cross_{\mfa}}$ preserves $1$, multiplication and addition uses nothing but axioms of ring.
  For any $r \in F$, we need to check $\iota_{\corss_{\mfa}}(r) = r \cdot 1$. Indeed $\iota_{\cross_{\mfa}}(r) = \Delta_{\id, r\cdot \mfa(\id,\id)^{-1}}$ and $r\cdot 1 = r \cdot \Delta_{\id,\mfa(\id,\id)^{-1}} =\Delta_{\id,r\cdot\mfa(\id,\id)^{-1}}$.
  When the context is clear, we also write $\iota_{\mfa}$ instead of $\iota_{\cross_{\mfa}}$.

  We note the following useful equality: for any $b \in K$

  \[
    \iota_{\mfa}(b)\Delta_{\sigma, c} = \Delta_{\sigma, bc},
  \]
  indeed:
  $\iota_{\mfa}(b)\Delta_{\sigma, c} = \Delta_{\id,b \mfa(\id,\id)^{-1}}\Delta_{\sigma, c} = \Delta_{\sigma, b\mfa(\id,\id)^{-1}c\mfa(\id,\sigma)} = \Delta_{\sigma,b\mfa(\id,\id)^{-1}c\mfa(\id,\id)}$ by~\cref{lem:2-cocycle-one-one}.
  \leanok
  \lean{GoodRep.CrossProduct.ι,GoodRep.CrossProduct.identity_double_cross'}
\end{construction}

%%% Local Variables:
%%% mode: LaTeX
%%% TeX-master: "../print"
%%% End:
