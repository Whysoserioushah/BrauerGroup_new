\section{Skolem-Noether Theorem}\label{sec:skolem-noether}

Let $K$ be a field, $A$, $B$ be $K$-algebras where $A$ is central simple and
finite $K$-dimensional and $B$ is simple. Let $M$ be a simple $A$-module.

\begin{construction}
  For any $K$-algebra homomorphism $f : B \to A$, we give $M$ a
  $B \otimes_{K}\End_{A}M$-module structure by defining $(b\otimes l)\cdot m$ to
  be $f(b)\cdot l(m)$. To emphasis $f$, we denote $M$ with the
  $B\otimes_{K}\End_{A}M$-module structure by $M^{f}$.
  \leanok
  \lean{IsMod}
\end{construction}

\begin{lemma}
  Let $f : B \to A$ be a $K$-algebra homomorphism, $M^{f}$ is finitely generated
  as a $B\otimes_{K}\End_{A}M$-module.
  \leanok
  \lean{module_inst_findim}
\end{lemma}
\begin{proof}
  Since $M$ is a finite $A$-module and $A$ a finite dimensional $K$-vector
  space, $M$ is a finite dimensional $K$-vector space as well. Suppose
  $S \subseteq M$ generates $M$ as $K$-module, the claim is that $S$ generates
  $M^{f}$ as well. Let $x \in M^{f}$, we write
  $x = \sum_{i}\lambda_{i}\cdot s_{i}$ with $\lambda_{i}\in K$ and $s_{i}\in S$.
  Note that $\lambda_{i}\cdot s_{i} = (\rho(\lambda_{i})\otimes\mathbf{1}_{M})$
  in $M^{f}$ where $\rho : K \to B$ is the map giving $B$ its $K$-algebra
  structure. Hence $x$ is in the span of $S$ in $M^{f}$ as well.
\end{proof}

\begin{remark}
  Given that $B$ is simple, any $k$-algebra homomorphism $f : B\to A$ injective;
  therefore by finite $K$-dimensionality of $A$, $B$ is finite $K$-dimensional
  as well.
\end{remark}

\begin{lemma}
  \label{lem:iso-fg}
  Let $f,g : B \to A$ be two $K$-algebra homomorphisms. Then $M^{f}$ and $M^{g}$
  are isomorphic as $B\otimes_{K}\End_{A}M$-module.
  \leanok
  \lean{iso_fg}
  % \uses{lem:iso-of-dim-eq}
\end{lemma}

\begin{proof}
  By~\cref{lem:iso-of-dim-eq}, it is sufficient to prove
  $\dim_{K}M^{f}=\dim_{K}M^{g}$. But as $K$-vector space, $M^{f}$ and $M^{g}$
  are {\em literally\/} $M$.
\end{proof}

\begin{theorem}[Skolem-Noether]\label{thm:skolem-noether}
  Let $f, g : B \to A$ be two $K$-algebra homomorphism. Then $f$ and $g$ differ
  only by a conjugation. That is there exists a unit $x \in A^{\times}$ such
  that $g = x f x^{-1}$.
  \leanok
  \lean{SkolemNoether'}
  % \uses{lem:exists-simple-mod, lem:iso-fg, lem:iso-end-end}
\end{theorem}

\begin{proof}
  Let $M$ be any simple $A$-module (which exists
  by~\cref{lem:exists-simple-mod}). By~\cref{lem:iso-fg}, we have some
  isomorphism $\phi : M^{f}\cong M^{g}$ as $B\otimes_{K}\End_{A}M$-module. Since
  $M$ is simple, we have that $M$ is a balanced $A$-module
  by~\cref{lem:iso-end-end}. Let $e$ denote the $k$-algerba isomorphism
  $A \cong \End_{\End_{A}M}M$ given by the $A$-action on $M$. Since both $\phi$
  and $\phi^{-1}$ defines an element of $\End_{\End_{A}M}M$, we define
  $a := e^{-1}(\phi)$ and $b := e^{-1}(\phi^{-1})$. Then $ab = 1$ since
  $e(ab)=e(a)\cdot e(b)=\phi\phi^{-1}=1$. We prove that the image of $f$ and
  $afb$ under $e$ are the same; that is for all $x\in B$ and $m\in M$,
  $e(g(x))(m) = e(a f(x) b)(m)$. The right hand side is equal to
  \[
    \begin{aligned}
      e(af(x)b)(m) &=\left(e(a)\circ e(f(x))\circ e(b)\right)(m) \\
                   &=\left(\phi\circ e(f(x))\circ \phi^{-1}\right)(m)\\
                   &=\phi\left(f(x)\cdot\phi^{-1}(m)\right) \\
    \end{aligned}.
  \]
  Similarly, the left hand side is equal to $g(x)\cdot m$. Note that $\phi$ is
  $B\otimes \End_{A}M$-linear. Therefore
  $\phi\left(\left(x\otimes\mathbf{1}\right)\cdot \phi^{-1}(m)\right) = \left(x\otimes\mathbf{1}\right)\cdot m$.
  Unfolding the definition of $M^{f}$ and $M^{g}$, we see this is saying
  precisely $\phi\left(f(x)\cdot\phi^{-1}(m)\right)=g(x)\cdot m$.
\end{proof}



%%% Local Variables:
%%% mode: LaTeX
%%% TeX-master: "../print"
%%% End:
