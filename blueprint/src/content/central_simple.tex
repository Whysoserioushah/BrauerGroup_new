\chapter{Central Simple Algebras}\label{sec:csa}

\section{Basic Theory}

In this chapter we define central simple algebras. We used some results
in~\cref{sec:unclassified}.

\begin{definition}[Simple Ring] A ring $R$ is simple if the only
  two-sided-ideals of $R$ are ${0}$ and $R$. An algebra is simple if it is
  simple as a ring.
  \leanok
  \lean{IsSimpleRing}
\end{definition}

\begin{remark}
  Division rings are simple.
\end{remark}

\begin{lemma}
  \label{lem:center-simple-ring}
  Let $A$ be a simple ring, then centre of $A$ is a field. \leanok
  \lean{IsSimpleRing.isField_center}
\end{lemma}
\begin{proof}
  Let $0\ne x$ be an element of centre of $A$. Then $I := \{xy | y\in A\}$ is a
  two-sided-ideal of $A$. Since $0\ne x\in I$, we have that $I = A$. Therefore
  $1 \in I$, hence $x$ is invertible.
\end{proof}

\begin{definition}[Central Algebras]
  Let $R$ be a ring and $A$ an $R$-algebra, we say $A$ is central if and only if
  the centre of $A$ is $R$
  \leanok
  \lean{Algebra.IsCentral}
\end{definition}

\begin{remark}
  Every commutative ring is a central algebra over itself.
\end{remark}

\begin{remark}
  Simpleness is invariant under ring isomorphism and centrality is invariant
  under algebra isomorphism.
\end{remark}

\begin{lemma}
  \label{lem:opp-central}
  If $A$ is a central $R$-algerba, $A^{\opp}$ is also central.
  \leanok
  \lean{CSA_op_is_CSA}.
\end{lemma}

\begin{lemma}\label{lem:simple-ring-iff}
  $R$ is a simple ring if and only if any ring homomorphism $f : R \to S$ either
  injective or $S$ is the trivial ring. \leanok
  \lean{IsSimpleRing.iff_injective_ringHom_or_subsingleton_codomain}
\end{lemma}
\begin{proof}
  If $R$ is simple, then the $\ker f$ is either $\{0\}$ or $R$. The former case
  implies that $f$ is injective while the latter case implies that $S$ is the
  trivial ring. Conversely, let $I\subseteq R$ be a two-sided-ideal. Consider
  $\pi: R \to {}^{R}/_{I}$, either $\pi$ is injective implying that $I = \{0\}$
  or that ${}^{R}/_{I}$ is the trivial ring implying that $I = R$.
\end{proof}
\begin{remark}
  If $A$ is a simple $R$-algebra, ``ring homomorphism'' in \cref{lem:simple-ring-iff} can be replaced with $R$-algebra homomorphism.
\end{remark}

\begin{corollary}\label{cor:alghom-bijective-of-dim-eq}
  Assume $R$ is a field. Let $A, B$ be finite dimensional $R$-algebras where $A$
  is simple as well. Then any $R$-algebra homomorphism $f:A\to B$ is bijective
  if $\dim_{R}A=\dim_{R}B$. 
  \leanok
  \lean{bijective_of_dim_eq_of_isCentralSimple}
  % \uses{lem:simple-ring-iff}
\end{corollary}
\begin{proof}
  By~\cref{lem:simple-ring-iff}, $f$is injective. Then
  $\dim_{K}\im f = \dim_{K} B - \dim_{K}\ker f = \dim_{K}B$ meaning that $f$ is
  surjective.
\end{proof}

Let $K$ be a field and $A, B$ be $K$-algebras.
\begin{lemma}
  \label{lem:tensor-central}
  If $A$ and $B$ are central $K$-algebras, $A\otimes_{K}B$ is a central
  $K$-algebra as well.
  \leanok
  \lean{IsCentralSimple.TensorProduct.isCentral}
  % \uses{cor:center-tensor-center}
\end{lemma}
\begin{proof}
  Assume $A$ and $B$ are central algebras, then
  by~\cref{cor:center-tensor-center}
  $Z\left(A\otimes_{R}B\right)=Z\left(A\right)\otimes_{R}Z\left(B\right)=R\otimes_{R}R=R$.
\end{proof}

\begin{theorem}
  \label{thm:tensor-csa}
  If $A$ is a simple $K$-algebra and $B$ is a central simple $K$-algebra,
  $A\otimes_{K}B$ is a central simple $K$-algebra as well. 
  \leanok
  \lean{IsCentralSimple.TensorProduct.simple}
  % \uses{lem:tensor-central, lem:expand-tensor-in-basis}
\end{theorem}
\begin{proof}
  By~\cref{lem:tensor-central}, we need to prove $A\otimes_{K}B$ is a simple
  ring. Denote $f$ as the map $A\to A\otimes_{K}B$. It is sufficient to prove
  that for every two-sided-ideal $I\subseteq A\otimes_{K}B$, we have
  $I = \left\langle f\left(f^{-1}\left(I\right)\right)\right\rangle$. Indeed,
  since $A$ is simple $f^{-1}\left(I\right)$ is either $\left\{0\right\}$ or
  $A$, if it is $\left\{0\right\}$, then $I=\left\{0\right\}$; if it is $A$,
  then $I$ is $A$ as well.

  We will prove that
  $I \le \left\langle f\left(f^{-1}\left(I\right)\right)\right\rangle$, the
  other direction is straightforward. Without loss of generality assume
  $I\ne\left\{0\right\}$. Let $\McA$ be an arbitrary basis of $A$, by
  \cref{lem:expand-tensor-in-basis}, we see that every element
  $x \in A\otimes_{K}B$ can be written as $\sum_{i=0}^{n}\McA_{i}\otimes b_{i}$
  for some natrual number $n$ and some choice of $b_{i}\in B$ and
  $\McA_{i}\in \McA$. Since $I$ is not empty, we see there exists a non-zero
  element $\omega\in I$ such that its expansion
  $\sum_{i=0}^{n}\McA_{i}\otimes b_{i}$ has the minimal $n$. In particular, all
  $b_{i}$ are non-zero and $n\ne0$. We have
  $\omega=\McA_{0}\otimes b_{0}+\sum_{i=1}^{n}\McA_{i}\otimes b_{i}$. Since $B$
  is simple, $1 \in B = \left\langle\langle b_{0} \right\rangle$; hence we write
  $1\in\sum_{j=0}^{m}x_{i}b_{0}y_{i}$ for some $x_{i},y_{i}\in B$. Define
  $\Omega := \sum_{j=0}^{m}(1\otimes x_{i})\omega(1\otimes y_{i})$ which is also
  in $I$. We write
  \[
    \begin{aligned}
      \Omega &= \McA_{0} \otimes \left(\sum_{j=0}^{m}x_{j}b_{0}y_{j}\right) + \sum_{i=1}^{n}\McA_{i}\otimes\left(\sum_{j=0}^{m}x_{j}b_{i}y_{j}\right) \\
             &= \McA_{0}\otimes 1 + \sum_{i=1}^{n}\McA_{i}\otimes\left(\sum_{j=0}^{m}x_{j}b_{i}y_{j}\right)
    \end{aligned}
  \]
  For every $\beta\in B$, we have that
  $\left(1\otimes \beta\right)\Omega - \Omega\left(1\otimes \beta\right)$ is in
  I and is equal to
  \[
    \sum_{i=1}^{n}\McA_{i}\otimes\left(\sum_{j=0}^{m}\beta x_{j}b_{i}y_{j}-x_{j}b_{i}y_{j}\beta\right),
  \]
  which is an expansion of $n-1$ terms, thus
  $\left(1\otimes\beta\right)\Omega-\Omega\left(1\otimes\beta\right)$ must be
  $0$. Hence we conclude that for all $i=1,\dots,n$,
  $\sum_{j=0}^{m}x_{j}b_{i}y_{j}\in Z\left(B\right)=K$. Hence for all
  $i=1,\dots,n$, we find a $\kappa_{i}\in K$ such that
  $\kappa_{i}=\sum_{j=0}^{m}x_{j}b_{i}y_{j}$. Hence we can calculate $\Omega$ as
  \[
    \begin{aligned}
      \Omega &= \McA_{0}\otimes 1 + \sum_{i=1}^{n}\McA_{i}\otimes\left(\sum_{j=0}^{m}\right) \\
             &= \McA_{0}\otimes 1 + \sum_{i=1}^{n}\McA_{i}\otimes\kappa_{i} \\
             &= \left(\McA_{0}+\sum_{i=1}^{n}\kappa_{i}\cdot\McA_{i}\right) \otimes 1
    \end{aligned}.
  \]
  From this, we note that
  $\McA_{0}+\sum_{i}^{n}\kappa_{i}\cdot\McA_{i}\in f^{-1}\left(I\right)$; since
  $A$ is simple, we immediately conclude that $f^{-1}\left(I\right) = A$, once
  we know $\McA_{0}+\sum_{i=1}^{n}\kappa_{i}\cdot \McA_{i}$ is not zero. If it
  is zero, by the fact that $\McA$ is a linearly independent set, we conclude
  that $1,\kappa_{1},\dots,\kappa_{n}$ are all zero; which is a contradiction.
  Since $f^{-1}\left(I\right) = A$, we know
  $\left\langle f\left(f^{-1}I\right)\right\rangle = A \otimes_{K} B$.
\end{proof}

\begin{corollary}\label{cor:csa-basechange}
  Central simple algebras are stable under base change. That is, if $L/K$ is a
  field extension and $D$ is a central simple algebra over $K$, then
  $L\otimes_{K} D$ is central simple over $L$. \leanok
  \lean{IsCentralSimple.baseChange}
  % \uses{thm:tensor-csa, cor:center-tensor-center}
\end{corollary}

\begin{proof}
  By~\cref{thm:tensor-csa}, $L\otimes_{K} D$ is simple. Let
  $x\in Z\left(L\otimes_{K}D\right)$, by~\cref{cor:center-tensor-center}, we
  have $x\in Z\left(L\right)\otimes Z\left(D\right)=Z\left(L\right)$. Without
  loss of generality, we can assume that $x = l \otimes d$ is a pure tensor,
  then $l \in Z\left(L\right)$ and $d \in K$. Therefore $x = d\cdot l \in L$.
\end{proof}

\begin{theorem}\label{thm:tensor-simple-implies-simple}
  If $A\otimes_{K} B$ is a simple ring, then $A$ and $B$ are both simple.
  \leanok
  \lean{IsCentralSimple.TensorProduct.simple}
  % \uses{lem:simple-ring-iff}
\end{theorem}
\begin{proof}
  By symmetry, we only prove that $A$ is simple.
  If $A$ or $B$ is the trivial ring then $A\otimes_{K} B$ is the trivial ring, a contradiction. Thus we assume both $A$ and $B$ are non-trivial. Suppose $A$ is not simple, by~\cref{lem:simple-ring-iff}, there exists a non-trivial $K$-algebra $A'$ and a $K$-algebra homomorphism $f : A \to A'$ such that $\ker f \not= \{0\}$. Let $F : A \otimes_{K} B\to A'\otimes_{K} B$ be the base change of $f$, then since $A\otimes_{K} B$ is simple and $A'\otimes B$ is non-trivial ($A'$ is non-trivial and $B$ is faithfully flat because $B$ is free), we conclude that $F$ is injective. Then we have that
  \begin{center}
    \begin{tikzcd}
      0 \arrow{r}{0} & A \otimes_{K} B \arrow{r}{F} & A'\otimes_K B
    \end{tikzcd}
  \end{center}
  is exact. Since $B$ is faithfully flat as a $K$-module, tensorig with $B$ reflects exact sequences, therefore
  \begin{center}
    \begin{tikzcd}
      0 \arrow{r}{0} & A \arrow{r}{f} & A'
    \end{tikzcd}
  \end{center}
  is exact as well. This is contradiction since $f$ is not injective.
\end{proof}

\section{Subfields of Central Simple Algebras}\label{sec:subfield}

\begin{definition}[Subfield]
  For any field $K$ and $K$-algebra $A$, a subfield $B \subseteq A$ is a commutative $K$-subalgebra of $A$ that is closed under inverse for any non-zero member.
  \leanok
  \lean{SubField}
\end{definition}

\begin{remark}
  Subfields inherit a natural ordering from subalgebras.
\end{remark}

Let $K$ be any field and $D$ a finite dimensional central division $K$-algebra and $A$ a finite dimensional central simple algebra of $A$.
\begin{lemma}\label{lem:dim-maximal-subfield}
  Let $k$ be a maximal subfield of $D$,
  \[
    \dim_{K}D = {\left(\dim_{K}k\right)}^{2}.
  \]
  \leanok
  \lean{dim_max_subfield}
  % \uses{lem:dim-centralizer}
\end{lemma}

\begin{proof}
  By~\cref{lem:dim-centralizer}, we have that $\dim_{K} D = \dim_{K}C_{D}(k)\cdot \dim_{K}k$. Hence it is sufficient to show that $C_{D}(k) = k$. By the commutativity of $k$, we have that $k \le C_{D}(k)$. Suppose $k \ne C_{D}(k)$: let $a \in C_{D}(k)$ that is not in $k$. We see that $L := k(a)$ is another subalgebra of $D$ that is strictly larger than $k$; a contradiction. Therefore $k = C_{D}(k)$ and the theorem is proved.
\end{proof}

\begin{lemma}\label{lem:tfae-subfield}
  Suppose $L$ is a subfield of $A$, the following are equivalent:
  \begin{enumerate}
    \item $L$ = $C_{A}(L)$
    \item $\dim_{K}A = {\left(\dim_{K}L\right)}^{2}$
    \item for any commutative $K$-subalgebra $L' \subseteq A$, $L \subseteq L'$ implies $L = L'$
  \end{enumerate}
  \leanok
  \lean{cor_two_1to2, cor_two_2to3, cor_two_3to1}
  % \uses{lem:dim-centralizer}
\end{lemma}
\begin{proof}
  We prove the following:
  \begin{itemize}
    \item ``1. implies 2.'': this is~\cref{lem:dim-centralizer}.
    \item ``2. implies 1.'': Since $L$ is commutative, we always have $L \subseteq C_{A}(L)$. Hence we only need to show $\dim_{K}L = \dim_{K}C_{A}(L)$. This is because by~\cref{lem:dim-centralizer}, we have that $\dim_{K}A = \dim_{K}L\cdot \dim_{K}C_{A}(L)$ and by 2. we have $\dim_{K}L\cdot \dim_{K}C_{A}(L)= \dim_{K}L\cdot \dim_{K}L$.
    \item ``2. implies 3.'': Since 2. implies 1., we assume $L = C_{A}(L)$, therefore all we need is to prove $L' \subseteq C_{A}(L)$. Let $x \in L'$ and $y \in L \subseteq L'$, we need to show $xy = yx$ which is commutativity of $L'$.
    \item ``3. implies 1.'': By commutativity of $L$, we always have $L \subseteq C_{A}(L)$. For the other direction, suppose $C_{A}(L)\not\subseteq L$, then there exists some $a \in C_{A}(L)$ but not in $L$. Consider $L' = L(a)$, by 3., we have $L' = L$ which is a contradiction.
  \end{itemize}
\end{proof}

%%% Local Variables:
%%% mode: LaTeX
%%% TeX-master: "../print"
%%% End:
