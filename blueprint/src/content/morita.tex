\chapter{Morita Equivalence}\label{sec:morita}

This chapter intertwine with \cref{sec:wed-artin}: \cref{sec:stacks-074e}
depends on \cref{sec:wed-artin-proof}; while \cref{sec:wed-artin-unique} depends
on \cref{sec:stacks-074e}.

\section{Construction of the equivalence}\label{sec:morita-construction}

Let $R$ be a ring and $0 \ne n\in \mathbb{N}$. In this chapter, we prove that
the category $R$-modules and the category of $\Mat_{n}(R)$-modules are
equivalent. Then we use the equivalence to prove several useful lemmas.

\begin{construction}\label{con:morita-eqv-functor0}
  \leanok
  \lean{matrix_smul_vec_def}
  If $M$ is an $R$-module, we have a natural
  $\Mat_{n}(R)$-module structure on $\widehat{M}:=M^{n}$ given by
  $(m_{ij})\cdot (v_{k})=\sum_{j}m_{ij}\cdot v_{j}$. If $f : M \to N$ is an
  $R$-linear map, then $\widehat{f} : M^{n}\to N^{n}$ given by
  $(v_{i}) \mapsto (f(v_{i}))$ is a $\Mat_{n}(R)$-linear map. Thus we have a
  well-defined functor $\MOD_{R} \Longrightarrow \MOD_{\Mat_{n}(R)}$.
\end{construction}

\begin{remark}
  Note that all modules are assumed to be left modules; when we need to consider
  right $R$-modules, we will consider left $R^{\opp}$-modules instead. We use
  $\delta_{ij}$ to denote the matrix whose $(i,j)$-th entry is $1$ and $0$
  elsewhere. $\delta_{ij}$ forms a basis for matrices.
\end{remark}

\begin{construction}\label{con:morita-eqv-functor1}
  \leanok
  \lean{fromModuleCatOverMatrix.smul_α_coe} If $M$ is a
  $\Mat_{n}(R)$-module, then
  $\widetilde{M} := \{\delta_{ij}\cdot m | m \in M\} \subseteq M$ is an
  $R$-module whose $R$-action is given by
  $r \cdot (\delta_{ij}\cdot m) := (r\cdot \delta_{ij})\cdot m$. More over if
  $f : M \to N$ is a $\Mat_{n}(R)$-linear map,
  $\widetilde{f} : \widetilde{M} \to \widetilde{N}$ given by the restriction of
  $f$ is $R$-linear. Hence, we have a functor
  $\MOD_{\Mat_{n}(R)}\Longrightarrow \MOD_{R}$.
\end{construction}

\begin{theorem}[Morita Equivalence]\label{thm:morita}
  The functors constructed in \cref{con:morita-eqv-functor0} and
  \cref{con:morita-eqv-functor1} form an equivalence of category. \leanok
  \lean{moritaEquivalentToMatrix}
\end{theorem}

\begin{proof}
  Let $M$ be an $R$-module, then the unit $\widetilde{\widehat{M}} \cong M$ is
  given by
  \[
    \begin{aligned}
      x \mapsto \sum_{j} x_{j} \\
      (x,0,\dots,0) \mapsfrom x
    \end{aligned}
  \]

  Let $M$ be an $\Mat_{n}(R)$-module, then the counit
  $\widehat{\widetilde{M}}\cong M$ is given by $m \mapsto (\delta_{i0}\cdot m)$.
  This map is both injective and surjective.
\end{proof}

\section{Stacks 074E}\label{sec:stacks-074e}

Let $A$ be a finite dimensional simple $k$-algebra.

\begin{lemma}\label{lem:isomorphic-simple-mod}
  Let $M$ and $N$ be simple $A$-modules, then $M$ and $N$ are isomorphic as
  $A$-modules.
  \leanok
  \lean{linearEquiv_of_isSimpleModule_over_simple_ring}
\end{lemma}

\begin{proof}
  By~\cref{thm:wed-artin-algebra}, there exists non-zero $n\in\mathbb N$,
  $k$-division algebra $D$ such that $A\cong \Mat_{n}(D)$ as $k$-algebras. Then
  by~\cref{thm:morita}, we have equivalence of category
  $e : \MOD_{A}\cong\MOD_{D}$. Since simple module is a categorical notion (it
  can be defined in terms monomorphisms), $e(M)$ and $e(N)$ are simple
  $D$-modules. Since $D$ is a division ring, $e(M)$ and $e(N)$ are isomorphic as
  $D$-modules, therefore $M$ and $N$ are isomorphic as $A$-modules.
\end{proof}

\begin{lemma}\label{lem:direct-sum-simple-mod}
  Let $M$ be an $A$-module, there exists a simple $A$-module $S$ such that $M$
  is a direct sum of copies of $S$, i.e. $M \cong \bigoplus_{i \in \iota} S$ for
  some indexing set $\iota$. \leanok
  \lean{directSum_simple_module_over_simple_ring}
\end{lemma}

\begin{proof}
  By~\cref{thm:wed-artin-algebra}, there exists non-zero $n\in\mathbb N$,
  $k$-division algebra $D$ such that $A\cong \Mat_{n}(D)$ as $k$-algebras. Then
  by~\cref{thm:morita}, we have equivalence of category
  $e : \MOD_{A}\cong\MOD_{D}$. Since simple module is a categorical notion (it
  can be defined in terms monomorphisms), $e^{-1}(D)$ is a simple module over
  $A$. Since $e(M)$ is a free module over $D$, we can write $e(M)$ as
  $\bigoplus_{i\in \iota} D$ for some indexing set $\iota$. By precomposing the
  unit of $e$, we get an isomorphism
  $M \cong e^{-1}\left(\bigoplus_{i\in\iota} D\right)$. We only need to prove
  $e^{-1}\left(\bigoplus_{i\in\iota} D\right)\cong\bigoplus_{i\in\iota}e^{-1}\left(D\right)$.
  This is because direct sum is the categorical coproduct.
\end{proof}

\begin{remark}
  Note that by~\cref{lem:isomorphic-simple-mod}, any two simple $A$-module are
  isomorphic, hence for any $A$-module $M$ and {\em any} simple $A$-module $S$,
  we can write $M$ as a direct sum of copies of $S$.
\end{remark}

\begin{lemma}\label{lem:iso-of-dim-eq}
  Let $M$ and $N$ be two finite $A$-module with compatible $k$-action. Then $M$
  and $N$ are isomorphic as $A$-module if and only if $\dim_{k} M$ and
  $\dim_{k} N$ are equal. \leanok
  \lean{linearEquiv_iff_finrank_eq_over_simple_ring}
\end{lemma}

\begin{proof}
  The forward direction is trivial as an $A$-linear isomorphism is a $k$-linear
  isomorphism as well. Conversely, suppose $\dim_{k}M = \dim_{k}N$.
  By~\cref{lem:direct-sum-simple-mod}, there exists a simple $A$-module $S$ such
  that $M \cong \bigoplus_{i\in\iota} S$ and $N \cong \bigoplus_{i\in\iota'} S$
  as $A$-modules. Without loss of generality $S \ne 0$, for otherwise we have
  $M \cong N$ anyway. If either of $\iota$ or $\iota'$ is empty, then
  $\dim_{k}M = \dim_{k} N = 0$ implying that $M = N = 0$, we again have
  $M \cong N$. Thus, we assume both $\iota$ and $\iota'$ are non-empty. By
  pulling back the $A$-module structure on $S$ to a $k$-module structure along
  $k \hookrightarrow A$,
  $M, N, S, \bigoplus_{i\in\iota} S, \bigoplus_{i\in\iota'} S$ are all finite
  dimensional $k$-vector spaces. Hence $\iota$ and $\iota'$ are finite. The
  equality $\dim_{k}M=\dim_{k}N$ tells us that $\iota \cong \iota'$ as set,
  hence $M \cong \bigoplus_{i\in\iota}S\cong\bigoplus_{i\in\iota'}S\cong N$ as
  required.
\end{proof}

Let $A \cong \Mat_{n}(D)$ as $k$-algebras for some $k$-division algebra and
$n\ne0$.

\begin{lemma}\label{lem:exists-simple-mod}
  $D^{n}$ is a simple $A$-module where the module structure is given by pulling
  back the $\Mat_{n}(D)$-module structure of $D^{n}$. \leanok
  \lean{simple_mod_of_wedderburn}
\end{lemma}
\begin{proof}
  By~\cref{thm:morita}, we have $\MOD_{A}\cong\MOD_{D}\cong\MOD_{\Mat_{n}(D)}$.
  Since $D$ is a simple $D$-module, $D^{n}$ is a simple $\Mat_{n}(D)$ module and
  consequently, a simple $A$-module.
\end{proof}

\begin{remark}
  Note that any $A$-linear endomorphism of $D^{n}$ is $\Mat_{n}(D)$-linear, and
  vice versa. Thus we have
  $\End_{A}\left(D^{n}\right)\cong \End_{\Mat_{n}(D)}\left(D^{n}\right)$ as
  $k$-algebras.
\end{remark}

\begin{lemma}\label{lem:end-vec-iso}
  $\End_{A}\left(D^{n}\right)$ is isomorphic to $D^{\opp}$ as $k$-algebras.
  \leanok
  \lean{end_simple_mod_of_wedderburn}
\end{lemma}

\begin{proof}
  Indeed, we calculate:
  \[
    \begin{aligned}
      \End_{A}\left(D^{n}\right) &\cong \End_{\Mat_{n}(D)}\left(D^{n}\right) & \\
                                 &\cong \End_{D}D & \text{by~\cref{thm:morita},}~\MOD_{D}\cong\MOD_{\Mat_{n}D}\\
                                 &\cong D^{\opp}
    \end{aligned}
  \]
\end{proof}

\begin{lemma}\label{lem:end-simple-iso}
  Let $M$ be a simple $A$-module, then $\End_{A} M\cong D^{\opp}$ as
  $k$-algebras.
  \leanok
  \lean{end_simple_mod_of_wedderburn'}
\end{lemma}
\begin{proof}
  By~\cref{thm:morita}, $D^{n}$ is simple as $A$-module; hence
  by~\cref{lem:isomorphic-simple-mod}, $D^{n}$ and $M$ are isomorphic as
  $A$-module. \Cref{lem:end-vec-iso} gives the desired result.
\end{proof}

\begin{remark}
  In particular, if $M$ is a simple $A$-module, then $\End_{A}M$ is a simple
  $k$-algbera.
\end{remark}

\begin{lemma}
  Let $M$ be a simple $A$-module, then $\End_{A}M$ has finite $k$-dimension.
  \leanok
  \lean{end_simple_mod_finite}
\end{lemma}
\begin{proof}
  By~\cref{thm:wed}, such $D$ and $n$ always exists. Hence we only need to show
  $D^{\opp}$ has finite $k$-dimension. Since $\dim_{k}A=\dim_{k}\Mat_{n}(D)$ are
  both finite, we conclude $D^{\opp}$ is finite as a $k$-vector space by pulling
  back the finiteness along $D \hookrightarrow \Mat_{n}(D)$.
\end{proof}

\begin{remark}
  Note that for all $A$-module $M$, $\End_{\End_{A}M}M$ is a $k$-algebra as
  well, with $k \hookrightarrow \End_{\End_{A}M}M$ given by
  $a \mapsto (x \mapsto a\cdot x)$. Thus, we always have a $k$-algebra
  homomorphism $A \to \End_{\End_{A}M}M$ given by the $A$-action on $M$. When
  $A$ is a simple ring, this map is injective.
\end{remark}

\begin{definition}[Balanced Module]\label{def:balanced-mod}
  \leanok
  \lean{IsBalanced}
  For any ring $A$ and $A$-module $M$, we say $M$ is a
  balanced $A$-module, if the $A$-linear map $A \to \End_{\End_{A}M}M$ is
  surjective.
\end{definition}

\begin{remark}
  Balancedness is invariant under linear isomorphism.
\end{remark}

\begin{lemma}\label{lem:balanced-self}
  For any ring $A$, $A$ is balanced as $A$-module. \leanok
\end{lemma}
\begin{proof}
  If $f \in \End_{\End_{A}M}A$, then the image of $f(1)$ under
  $A \to\End_{\End_{A}}A$ is $f$ again.
\end{proof}

We assume again that $A$ is a finite dimensional simple $k$-algebra.
\begin{lemma}
  Any simple $A$-module is balanced.
  \leanok
  \lean{isBalanced_of_simpleMod}
\end{lemma}
\begin{proof}
  Indeed, if $M$ is a simple $A$-module, then $A \cong \bigoplus_{i\in\iota} M$
  for some indexing set $\iota$ by~\cref{lem:direct-sum-simple-mod}. Since $A$
  is balanced, $\bigoplus_{i\in\iota}M$ is balanced. Let
  $g \in \End_{\End_{A}M}M$, we can define a corresponding
  $G \in \End_{\End_{\bigoplus_{i}M}}\left(\bigoplus_{i}M\right)$ by sending
  $(v_{i})$ to $(g(v_{i}))$. Since $\bigoplus_{i}M$ is balanced, we know that
  for some $a\in A$, $G$ is the image of $a$ under
  $A \to \End_{\End_{\bigoplus_{i}M}}\left(\bigoplus_{i}M\right)$. Then the
  image of $a$ under $A \to \End_{\End_{A}M}M$ is $g$.
\end{proof}

\begin{lemma}\label{lem:iso-end-end}
  For any simple $A$-module $M$, we have $A \cong \End_{\End_{A}M}M$ as
  $k$-algebras.
\end{lemma}
\begin{proof}
  The canonical map $A \to \End_{\End_{A}M}M$ is both injective and surjective,
  as $M$ is a balanced $A$-module and $A$ is a simple ring.
\end{proof}

%%% Local Variables:
%%% mode: LaTeX
%%% TeX-master: "../print"
%%% End:
