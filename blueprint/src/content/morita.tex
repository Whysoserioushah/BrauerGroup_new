\chapter{Morita Equivalence}\label{chap:morita}

This chapter intertwine with \cref{chap:wed-artin}: \cref{sec:stacks-074e} depends on \cref{sec:wed-artin-proof}; while \cref{sec:wed-artin-unique} depends on \cref{sec:stacks-074e}.

\section{Construction of the equivalence}\label{sec:morita-construction}

Let $R$ be a ring and $0 \ne n\in \mathbb{N}$. In this chapter, we prove that the category $R$-modules and the category of $\Mat_{n}(R)$-modules are equivalent. Then we use the equivalence to prove several useful lemmas.

\begin{construction}\label{con:morita-eqv-functor0}
  \leanok
  \lean{matrix_smul_vec_def}
  If $M$ is an $R$-module, we have a natural $\Mat_{n}(R)$-module structure on $\hat{M}:=M^{n}$ given by $(m_{ij})\cdot (v_{k})=\sum_{j}m_{ij}\cdot v_{j}$.
  If $f : M \to N$ is an $R$-linear map, then $\hat{f} : M^{n}\to N^{n}$ given by $(v_{i}) \mapsto (f(v_{i}))$ is a $\Mat_{n}(R)$-linear map. Thus we have a well-defined functor $\MOD_{R} \Longrightarrow \MOD_{\Mat_{n}(R)}$.
\end{construction}

Note that all modules are assumed to be left modules; when we need to consider right $R$-modules, we will consider left $R^{\opp}$-modules instead. We use $\delta_{ij}$ to denote the matrix whose $(i,j)$-th entry is $1$ and $0$ elsewhere. $\delta_{ij}$ forms a basis for matrices.

\begin{construction}\label{con:morita-eqv-functor1}
  \leanok
  \lean{fromModuleCatOverMatrix.smul_α_coe}
  If $M$ is a $\Mat_{n}(R)$-module, then $\tilde{M} := \{\delta_{ij}\cdot m | m \in M\} \subseteq M$ is an $R$-module given by $r \cdot (\delta_{ij}\cdot m) := (r\cdot \delta_{ij})\cdot m$. More over if $f : M \to N$ is a $\Mat_{n}(R)$-linear map, $\tilde{f} : \tilde{M} \to \tilde{N}$ given by $f$ is $R$-linear. Hence, we have a functor $\MOD_{\Mat_{n}(R)}\Longrightarrow \MOD_{R}$.
\end{construction}

\begin{theorem}[Morita Equivalence]\label{thm:morita}
  The functors constructed in \cref{con:morita-eqv-functor0} and \cref{con:morita-eqv-functor1} form an equivalence of category.
  \leanok
  \lean{moritaEquivalentToMatrix}
\end{theorem}

\begin{proof}
  Let $M$ be an $R$-module, then the unit $\tilde{\hat{M}} \cong M$ is given by
  \[
    \begin{aligned}
      x \mapsto \sum_{j} x_{j} \\
      (x,0,\dots,0) \mapsfrom x
    \end{aligned}
  \]

  Let $M$ be an $\Mat_{n}(R)$-module, then the counit $\hat{\tilde{M}}\cong M$ is given by $m \mapsto (\delta_{i0}\cdot m)$. This map is both injective and surjective.
\end{proof}

\section{Stacks 074E}\label{sec:stacks-074e}

Let $A$ be a finite dimensional simple $k$-algebra.

\begin{lemma}
  Let $M$ and $N$ be simple $A$-modules, then $M$ and $N$ are isomorphic as $A$-modules.
  \leanok
  \lean{linearEquiv_of_isSimpleModule_over_simple_ring}
\end{lemma}

\begin{proof}
  By~\cref{thm:wed-artin-algebra}, there exists non-zero $n\in\mathbb N$, $k$-division algebra $D$ such that $A\cong \Mat_{n}(D)$ as $k$-algebras. Then by~\cref{thm:morita}, we have equivalence of category $e : \MOD_{A}\cong\MOD_{D}$. Since simple module is a categorical notion (it can be defined in terms monomorphisms), $e(M)$ and $e(N)$ are simple $D$-modules. Since $D$ is a division ring, $e(M)$ and $e(N)$ are isomorphic as $D$-modules, therefore $M$ and $N$ are isomorphic as $A$-modules.
\end{proof}

\begin{lemma}
  Let $M$ be an $A$-module, there exists a simple $A$-module $S$ such that $M$ is a direct sum of copies of $S$, i.e. $M \cong \bigoplus_{i \in \iota} S$ for some indexing set $\iota$.
  \leanok
  \lean{directSum_simple_module_over_simple_ring}
\end{lemma}

\begin{proof}
  By~\cref{thm:wed-artin-algebra}, there exists non-zero $n\in\mathbb N$, $k$-division algebra $D$ such that $A\cong \Mat_{n}(D)$ as $k$-algebras. Then by~\cref{thm:morita}, we have equivalence of category $e : \MOD_{A}\cong\MOD_{D}$. Since simple module is a categorical notion (it can be defined in terms monomorphisms), $e^{-1}(D)$ is a simple module over $A$. Since $e(M)$ is a free module over $D$, we can write $e(M)$ as $\bigoplus_{i\in \iota} D$ for some indexing set $\iota$. By precomposing the unit of $e$, we get an isomorphism $M \cong e^{-1}\left(\bigoplus_{i\in\iota} D\right)$. We only need to prove $e^{-1}\left(\bigoplus_{i\in\iota} D\right)\cong\bigoplus_{i\in\iota}e^{-1}\left(D\right)$. This is because direct sum is the categorical coproduct.
\end{proof}

%%% Local Variables:
%%% mode: LaTeX
%%% TeX-master: "../print"
%%% End:
