\section{Double Centralizer Theorem}\label{sec:double-centralizer}

In this section let $F$ be a field and $A$ an $F$-algebra. Define
$\McL_{A} \subseteq \End_{F}A$ to be
\[\left\{f : A \to A|f(x)=ax~\text{for some}~a\in A\right\},\]
i.e. $F$-linear maps defined by left multiplication; similarly define
$\McR_{A}$. Note that $\McL_{A}$ and $\McR_A$ are $F$-subalgebras of
$\End_{F}A$. When we need to stree the underlying field is $F$, we also write
$\McL_{A}^{F}$ and $\McR_{A}^{F}$. We assume $A$ to be a finite dimensional
central simple $F$-algebra.
\begin{lemma}
  \label{lem:centralizer-mul-left-le}
  The centralizer of $\McL_{A}$ in $\End_{F}A$ is smaller than or equal to
  $\McR_{A}$:
  \[
    C_{\End_{F}A}\left(\McL_{A}\right) \le \McR_{A}.
  \]
  \leanok
  \lean{centralizer_mulLeft_le_of_isCentralSimple}
\end{lemma}

\begin{proof}
  Indeed, let $x \in C_{\End_{F}A}\left(\McL_{A}\right)$. Recall
  from~\cref{con:self-tensor-opp-iso-end} that
  $e : A\otimes_{F}A^{\opp}\cong \End_{F}A$ as $F$-algebras. Then $e^{-1}(x)$ is
  in
  $C_{A\otimes_{F}A^{\opp}}\left(\im\left(A\to A\otimes_{F}A^{\opp}\right)\right)$
  (for $e$ sends $a\otimes 1$ to the $F$-linear map $(a\cdot\bullet)$). Since
  $C_{A\otimes_{F}A^{\opp}}\left(\im\left(A\to A\otimes_{F}A^{\opp}\right)\right) = Z(A)\otimes_{F}A^{\opp}=F\otimes_{F}A^{\opp}=\im\left(A^{\opp}\to A\otimes_{F}A^{\opp}\right)$,
  we find some $y\in A^{\opp}$ such that $1 \otimes y = e^{-1}(x)$. Therefore
  $e\left(1\otimes y\right) = x$; but $e\left(1\otimes y\right)$ is in
  $\McR_{A}$ for it is the linear map $(\bullet\cdot y)$.
\end{proof}

\begin{remark}\label{rem:mul-left-center-linear}
  For any $F$-algebra $B$, every element in $C_{\End_{F}B}\left(\McL_{B}\right)$
  is in fact $Z(B)$-linear. Let $x\in C_{\End_{F}B}\left(\McL_{B}\right)$,
  $z\in Z(B)$ and $b \in B$, we have $x(z\cdot b) = z\cdot x(b)$ because $x$
  commutes with the linear map $\left(z\cdot\bullet\right)$.
\end{remark}

\begin{remark}
  $A$ is a $Z(A)$-algebra whose algebra structure is given by
  $Z(A)\hookrightarrow A$. By~\cref{lem:center-simple-ring}, $Z(A)$ is a field.
  $A$ is finite dimensional as a $Z(A)$-module because of the tower $A/Z(A)/F$.
\end{remark}

 \begin{lemma}\label{lem:mul-right-iso-opp}
   As $F$-algebras, we have $\McR_{A}\cong A^{\opp}$. \leanok
   \lean{Module.End.rightMulEquiv}
 \end{lemma}
 \begin{proof}
   We prove the map $A^{\opp} \to \McR_{A}$ is bijective. It is injective
   because if $\left(\bullet \cdot a\right) = \left(\bullet\cdot b\right)$, then
   $a = 1 \cdot a = 1 \cdot b = b$. The map is surjective by the definition of
   $\McR_{A}$.
 \end{proof}
 \begin{lemma}\label{lem:centralizer-mul-left-eq-mul-right}
   Let $B$ be any simple $F$-algebra ({\em not\/} necessarily central). The
   centralizer of $\McL_{B}$ in $\End_{F}B$ is equal to $\McR_{B}$. \leanok
   \lean{centralizer_mulLeft}
 \end{lemma}
 \begin{proof}
   It is straightforward to show
   $\McR_{B}^{F}\le C_{\End_{F}A}\left(\McL_{B}^{F}\right)$. So we only need to
   prove $C_{\End_{F}A}\left(\McL_{B}^{F}\right)\le \McR_{B}^{F}$.
   By~\cref{lem:centralizer-mul-left-le}, since $B$ is a central simple finite
   dimensional $Z(B)$-algebra, we have that
   \[
     C_{\End_{Z(B)}B}\left(\McL_{B}^{Z(B)}\right) \le \McR_{B}^{Z(B)}.
   \]
   Suppose $f\in\End_{F}B$ is in $C_{\End_{F}B}B$, by
   ~\cref{rem:mul-left-center-linear}, $f$ is $Z(B)$-linear as well. Then $f$ is
   in $\McR_{B}^{Z(B)}$; that is $f$ is equal to $\left(\bullet\cdot b\right)$
   for some $b \in B$ as $Z(B)$-linear maps. Then $f$ is also equal to
   $\left(\bullet\cdot b\right)$ as $F$-linear maps.
 \end{proof}

 \begin{construction}
   Let $B$ be any $F$-algebra and $S\subseteq B$ an $F$-subalgebra. For any
   $x\in B^{\times}$, we have that $x S x^{-1}:=\{xsx^{-1}|s\in S\}$ is an
   $F$-subalgebra of $B$ as well. We have the obvious $F$-algebra isomorphism
   $S \cong xSx^{-1}$ given by $s\mapsto xsx^{-1}$ and $x^{-1}tx\mapsfrom t$.
   Therefore $\dim_{F}S=\dim_{F}xSx^{-1}$ and $S$ is a simple ring if and only
   if $xSx^{-1}$ is a simple ring.
   \leanok
   \lean{Subalgebra.conj}
 \end{construction}

 \begin{lemma}
   \label{lem:centralizer-conj}
   Let $B$ be any $F$-algebra, $x \in B^{\times}$ and $S\subseteq B$ be an
   $F$-subalgebra of $B$, then $C_{B}(xSx^{-1})=x\left(C_{B}(S)\right)x^{-1}$.
   \leanok
   \lean{Subalgebra.conj_centralizer}
 \end{lemma}
 \begin{proof}
   If $a \in C_{B}\left(xSx^{-1}\right)$, then $x^{-1}ax$ is in $C_{B}(S)$.
   Conversely if $a$ is equal to $xbx^{-1}$ with $b\in C_{B}(S)$, then it is in
   $C_{B}\left(xSx^{-1}\right)$ as well.
 \end{proof}

 \begin{remark}
   For any finite dimensional $F$-module $B$, we have isomorphism
   $\End_{F}B \cong \Mat_{\dim_{F}B}F$ as $F$-algebras. Hence $\End_{F} B$ is a
   finite-dimensional central simple algebra over $F$.
 \end{remark}

 \begin{lemma}\label{lem:tensor-mul-right-simple}
   Let $S\subseteq A$ be a simple $F$-subalgebra, then
   $A \otimes_{F} \McR_{S}$ is a simple ring. \leanok
 \end{lemma}

 \begin{proof}
   By~\cref{lem:mul-right-iso-opp}, we have
   $A\otimes \McR_{S}\cong A\otimes S^{\opp}$ as $F$-algebras. The claim follows
   from~\cref{thm:tensor-csa}.
 \end{proof}

 \begin{lemma}\label{lem:centralizer-tensor-end-eq-conj-tensor-mul-right}
   Let $S\subseteq A$ be a simple $F$-subalgebra, then there
   exists an $x\in \left(A\otimes_{F}\End_{F}S\right)^{\times}$ such that
   $C_{A}(S)\otimes_{F}\End_{F}S$ is isomorphic to
   $x\left(A\otimes_{F}\McR_{S}\right)x^{-1}$ as $F$-algebras. \leanok
   \lean{centralizer_isSimple.aux.step1}
 \end{lemma}

 \begin{proof}
   By~\cref{lem:tensor-central} and~\cref{thm:tensor-csa},
   $A \otimes_{F}C_{A}(S)$ is a central simple $F$-algebra. Let
   $f : S \to A\otimes_{F}\End_{F}S$ be an $F$-algebra homomorphism defined by
   $s \mapsto s \otimes \mathbf{1}_{S}$ and $g : S \to A \otimes_{F}\End_{F}S$
   be an $F$-algebra homomorphism defined by
   $\mathbf{1}_{A} \otimes \left(s\cdot\bullet\right)$. Then
   by~\cref{thm:skolem-noether}, we that there exists some
   $x\in \left(A\otimes_{F}\End_{F}S\right)^{\times}$ such that $f = xgx^{-1}$.
   Then we have $S\otimes_{F}\End_{F}S$ is equal to
   $x\left(A\otimes_{F}\McL_{S}\right)x^{-1}$: indeed the left hand side is
   $\im f$ while the right handside is $x\left(\im g\right)x^{-1}$. Therefore
   $C_{A\otimes_{F}\End_{F}S}\left(S\otimes_{F}\End_{F}S\right) = C_{A\otimes_{F}\End_{F}S}\left(x\left(A\otimes_{F}\McR_{S}\right)x^{-1}\right)$.
   By~\cref{lem:centralizer-conj}, the right hand side is equal to
   $xC_{A\otimes_{F}\End_{F}S}\left(A\otimes_{F}\McL_{S}\right)x^{-1}$ which is
   $x\left(A\otimes_{F}C_{\End_{F}S}\left(\McL_{S}\right)\right)x^{-1}$
   by~\cref{lem:centralizer-inclusion-tensor} which is
   $x\left(A\otimes_{F}\McR_{S}\right)x^{-1}$
   by~\cref{lem:centralizer-mul-left-eq-mul-right}.
 \end{proof}

 \begin{lemma}\label{lem:simple-centralizer}
   Let $S\subseteq A$ be a simple $F$-subalgebra, then
   $C_{A}(S)$ is simple as well.
   \leanok
   \lean{centralizer_isSimple}
 \end{lemma}

 \begin{proof}
   By~\cref{lem:centralizer-tensor-end-eq-conj-tensor-mul-right},
   $C_{A}(S) \otimes_{F}\End_{F}S$ is isomorphic to
   $x\left(A\otimes_{F}\McR_{S}\right)x^{-1}$ as $F$-algebras. Then
   $C(S)\otimes_{F}\End_{F}S$ is simple since $A\otimes_{F}\McR_{S}$ is simple
   by~\cref{lem:tensor-mul-right-simple}.
   By~\cref{thm:tensor-simple-implies-simple}, $C_{A}(S)$ is simple.
 \end{proof}

 \begin{lemma}
   \label{lem:dim-centralizer}
   Let $S\subseteq A$ be a simple $F$-subalgebra. Then
   \[
     \dim_{F}C_{A}(S)\cdot\dim_{F}S=\dim_{F}A.
   \]
   \leanok
   \lean{dim_centralizer}
 \end{lemma}
 \begin{proof}
   By~\cref{lem:centralizer-tensor-end-eq-conj-tensor-mul-right},
   $C_{A}(S) \otimes_{F}\End_{F}S$ is isomorphic to
   $x\left(A\otimes_{F}\McR_{S}\right)x^{-1}$ as $F$-algebras. Hence $\dim_{F}\left(C_{A}(S)\otimes_{F}\End_{F}S\right)=\dim_{F}\left(A\otimes_{F}\McR_{S}\right)$ where the left hand side is $\dim_{F}C_{A}(S)\cdot \dim_{F}\End_{F}S$ and the right hand side is $\dim_{F}A\cdot\dim_{F}\McR_{S}$. Since $\dim_{F}\End_{F}S=\dim_{F}S^{2}$ and $\dim_{F}\McR_{S}=\dim S$ (by~\cref{lem:mul-right-iso-opp}), we proved this lemma.
 \end{proof}

\begin{corollary}
  \label{cor:self-tensor-centralizer}
  Let $S \subseteq A$ be a central simple $F$-subalgebra,
  \[
    A \cong B \ox_{F} C_{A}(B).
  \]
  \leanok
  \lean{writeAsTensorProduct}
\end{corollary}

\begin{proof}
  By~\cref{lem:simple-centralizer}, $C_{A}(B)$ is simple and by~\cref{thm:tensor-csa}, $B \ox_{F} C_{A}(B)$ is simple. Hence the map $B \ox_{F} C_{A}(B) \to A$ induced by $B \hookrightarrow A$ and $C_{A}(B) \hookrightarrow A$ is injective. By~\cref{cor:alghom-bijective-of-dim-eq}, we only need to show $\dim_{F}B\ox_{F}C_{A}(B) =\dim_{F}A$ which is precisely~\cref{lem:dim-centralizer}.
\end{proof}

 \begin{theorem}[Double Centralizer]\label{thm:double-centralizer}
   Let $S\subseteq A$ be a simple $F$-subalgebra, we have
   \[
     C_{A}\left(C_{A}(S)\right) = S.
   \]
   \leanok
   \lean{double_centralizer}
 \end{theorem}
 \begin{proof}
  It is straightforward that $S \le C_{A}\left(C_{A}(S)\right)$. By~\cref{lem:simple-centralizer}, $C_{A}(S)$ is simple, hence $\dim_{F}C_{A}\left(C_{A}(S)\right)\cdot \dim_{F}C_{A}(S)=\dim_{F}A=\dim_{F}C_{A}(S)\cdot \dim_{F}S$ (by applying~\cref{lem:dim-centralizer} twice), i.e. $\dim_{F}C_{A}\left(C_{A}(S)\right) = \dim_{F}S$. This equality of dimension gives us the desired result.
\end{proof}

 
%%% Local Variables:
%%% mode: LaTeX
%%% TeX-master: "../print"
%%% End:
