% This file makes a printable version of the blueprint
% It should include all the \usepackage needed for the pdf version.
% The template version assume you want to use a modern TeX compiler
% such as xeLaTeX or luaLaTeX including support for unicode
% and Latin Modern Math font with standard bugfixes applied.
% It also uses expl3 in order to support macros related to the dependency graph.
% It also includes standard AMS packages (and their improved version
% mathtools) as well as support for links with a sober decoration
% (no ugly rectangles around links).
% It is otherwise a very minimal preamble (you should probably at least
% add cleveref and tikz-cd).

\documentclass[a4paper]{report}

% \setcounter{secnumdepth}{3}
\setcounter{tocdepth}{3}

\usepackage{geometry}

\usepackage{expl3}

\usepackage{stmaryrd}

\usepackage{tikz-cd}

\usepackage{amssymb, amsthm, mathtools}
\usepackage[unicode,colorlinks=true,linkcolor=blue,urlcolor=magenta,
citecolor=blue]{hyperref}

% \usepackage{beton} \usepackage[euler-digits,euler-hat-accent]{eulervm}

\usepackage[warnings-off={mathtools-colon,mathtools-overbracket}]{unicode-math}

\usepackage{cleveref}

% In this file you should put all LaTeX macros to be used
% both by the pdf version and the web version.
% This should be most of your macros.

\theoremstyle{plain}
\newtheorem{theorem}{Theorem}[section]
\newtheorem{lemma}[theorem]{Lemma}
\newtheorem{corollary}[theorem]{Corollary}

\theoremstyle{remark}
\newtheorem{remark}{Remark}[section]

\theoremstyle{definition}
\newtheorem{definition}{Definition}[section]
\newtheorem{construction}[definition]{Construction}

\DeclareMathOperator{\Mat}{Mat}
\DeclareMathOperator{\mat}{Mat}
\DeclareMathOperator{\End}{End}
\DeclareMathOperator{\CSA}{\mathsf{C}\!\mathsf{S}\!\mathsf{A}}
\DeclareMathOperator{\csa}{\mathsf{C}\!\mathsf{S}\!\mathsf{A}}

\DeclareMathOperator{\Br}{Br}
\DeclareMathOperator{\br}{Br}

\DeclareMathOperator{\MOD}{\mathfrak{Mod}}

\DeclareMathOperator{\range}{range}
\DeclareMathOperator{\im}{im}

\DeclareMathOperator{\diag}{diag}

\newcommand{\opp}{\mathsf{opp}}

\DeclareMathOperator{\HH}{H}
\DeclareMathOperator{\hh}{H}
\DeclareMathOperator{\gal}{Gal}
\DeclareMathOperator{\twist}{twist}
\DeclareMathOperator{\comp}{comp}

\newcommand{\McA}{\mathcal{A}}
\newcommand{\McB}{\mathcal{B}}
\newcommand{\McL}{\mathcal{L}}
\newcommand{\McR}{\mathcal{R}}

\newcommand{\ox}{\otimes}
\newcommand{\Ox}{\bigotimes}

\newcommand{\nat}{\mathbb{N}}
%%% Local Variables:
%%% mode: LaTeX
%%% TeX-master: "../print"
%%% End:
 % This file makes a printable version of the blueprint
% It should include all the \usepackage needed for the pdf version.
% The template version assume you want to use a modern TeX compiler
% such as xeLaTeX or luaLaTeX including support for unicode
% and Latin Modern Math font with standard bugfixes applied.
% It also uses expl3 in order to support macros related to the dependency graph.
% It also includes standard AMS packages (and their improved version
% mathtools) as well as support for links with a sober decoration
% (no ugly rectangles around links).
% It is otherwise a very minimal preamble (you should probably at least
% add cleveref and tikz-cd).

\documentclass[a4paper]{article}

\usepackage{geometry}

\usepackage{expl3}


\usepackage{tikz-cd}

\usepackage{amssymb, amsthm, mathtools}
\usepackage[unicode,colorlinks=true,linkcolor=blue,urlcolor=magenta, citecolor=blue]{hyperref}

\usepackage[warnings-off={mathtools-colon,mathtools-overbracket}]{unicode-math}

\usepackage{cleveref}

% In this file you should put all LaTeX macros to be used
% both by the pdf version and the web version.
% This should be most of your macros.

\theoremstyle{plain}
\newtheorem{theorem}{Theorem}[section]
\newtheorem{lemma}[theorem]{Lemma}
\newtheorem{corollary}[theorem]{Corollary}

\theoremstyle{remark}
\newtheorem{remark}{Remark}[section]

\theoremstyle{definition}
\newtheorem{definition}{Definition}[section]
\newtheorem{construction}[definition]{Construction}

\DeclareMathOperator{\Mat}{Mat}
\DeclareMathOperator{\mat}{Mat}
\DeclareMathOperator{\End}{End}
\DeclareMathOperator{\CSA}{\mathsf{C}\!\mathsf{S}\!\mathsf{A}}
\DeclareMathOperator{\csa}{\mathsf{C}\!\mathsf{S}\!\mathsf{A}}

\DeclareMathOperator{\Br}{Br}
\DeclareMathOperator{\br}{Br}

\DeclareMathOperator{\MOD}{\mathfrak{Mod}}

\DeclareMathOperator{\range}{range}
\DeclareMathOperator{\im}{im}

\DeclareMathOperator{\diag}{diag}

\newcommand{\opp}{\mathsf{opp}}

\DeclareMathOperator{\HH}{H}
\DeclareMathOperator{\hh}{H}
\DeclareMathOperator{\gal}{Gal}
\DeclareMathOperator{\twist}{twist}
\DeclareMathOperator{\comp}{comp}

\newcommand{\McA}{\mathcal{A}}
\newcommand{\McB}{\mathcal{B}}
\newcommand{\McL}{\mathcal{L}}
\newcommand{\McR}{\mathcal{R}}

\newcommand{\ox}{\otimes}
\newcommand{\Ox}{\bigotimes}

\newcommand{\nat}{\mathbb{N}}
%%% Local Variables:
%%% mode: LaTeX
%%% TeX-master: "../print"
%%% End:

% This file makes a printable version of the blueprint
% It should include all the \usepackage needed for the pdf version.
% The template version assume you want to use a modern TeX compiler
% such as xeLaTeX or luaLaTeX including support for unicode
% and Latin Modern Math font with standard bugfixes applied.
% It also uses expl3 in order to support macros related to the dependency graph.
% It also includes standard AMS packages (and their improved version
% mathtools) as well as support for links with a sober decoration
% (no ugly rectangles around links).
% It is otherwise a very minimal preamble (you should probably at least
% add cleveref and tikz-cd).

\documentclass[a4paper]{article}

\usepackage{geometry}

\usepackage{expl3}


\usepackage{tikz-cd}

\usepackage{amssymb, amsthm, mathtools}
\usepackage[unicode,colorlinks=true,linkcolor=blue,urlcolor=magenta, citecolor=blue]{hyperref}

\usepackage[warnings-off={mathtools-colon,mathtools-overbracket}]{unicode-math}

\usepackage{cleveref}

% In this file you should put all LaTeX macros to be used
% both by the pdf version and the web version.
% This should be most of your macros.

\theoremstyle{plain}
\newtheorem{theorem}{Theorem}[section]
\newtheorem{lemma}[theorem]{Lemma}
\newtheorem{corollary}[theorem]{Corollary}

\theoremstyle{remark}
\newtheorem{remark}{Remark}[section]

\theoremstyle{definition}
\newtheorem{definition}{Definition}[section]
\newtheorem{construction}[definition]{Construction}

\DeclareMathOperator{\Mat}{Mat}
\DeclareMathOperator{\mat}{Mat}
\DeclareMathOperator{\End}{End}
\DeclareMathOperator{\CSA}{\mathsf{C}\!\mathsf{S}\!\mathsf{A}}
\DeclareMathOperator{\csa}{\mathsf{C}\!\mathsf{S}\!\mathsf{A}}

\DeclareMathOperator{\Br}{Br}
\DeclareMathOperator{\br}{Br}

\DeclareMathOperator{\MOD}{\mathfrak{Mod}}

\DeclareMathOperator{\range}{range}
\DeclareMathOperator{\im}{im}

\DeclareMathOperator{\diag}{diag}

\newcommand{\opp}{\mathsf{opp}}

\DeclareMathOperator{\HH}{H}
\DeclareMathOperator{\hh}{H}
\DeclareMathOperator{\gal}{Gal}
\DeclareMathOperator{\twist}{twist}
\DeclareMathOperator{\comp}{comp}

\newcommand{\McA}{\mathcal{A}}
\newcommand{\McB}{\mathcal{B}}
\newcommand{\McL}{\mathcal{L}}
\newcommand{\McR}{\mathcal{R}}

\newcommand{\ox}{\otimes}
\newcommand{\Ox}{\bigotimes}

\newcommand{\nat}{\mathbb{N}}
%%% Local Variables:
%%% mode: LaTeX
%%% TeX-master: "../print"
%%% End:

% This file makes a printable version of the blueprint
% It should include all the \usepackage needed for the pdf version.
% The template version assume you want to use a modern TeX compiler
% such as xeLaTeX or luaLaTeX including support for unicode
% and Latin Modern Math font with standard bugfixes applied.
% It also uses expl3 in order to support macros related to the dependency graph.
% It also includes standard AMS packages (and their improved version
% mathtools) as well as support for links with a sober decoration
% (no ugly rectangles around links).
% It is otherwise a very minimal preamble (you should probably at least
% add cleveref and tikz-cd).

\documentclass[a4paper]{article}

\usepackage{geometry}

\usepackage{expl3}


\usepackage{tikz-cd}

\usepackage{amssymb, amsthm, mathtools}
\usepackage[unicode,colorlinks=true,linkcolor=blue,urlcolor=magenta, citecolor=blue]{hyperref}

\usepackage[warnings-off={mathtools-colon,mathtools-overbracket}]{unicode-math}

\usepackage{cleveref}

\input{macros/common}
\input{macros/print}

\title{Brauer Group and Galois Cohomology}
\author{Yunzhou Xie \and Jujian Zhang}

\begin{document}
\maketitle
\input{content}
\end{document}


\title{Brauer Group and Galois Cohomology}
\author{Yunzhou Xie \and Jujian Zhang}

\begin{document}
\maketitle
% In this file you should put the actual content of the blueprint.
% It will be used both by the web and the print version.
% It should *not* include the \begin{document}
%
% If you want to split the blueprint content into several files then
% the current file can be a simple sequence of \input. Otherwise It
% can start with a \section or \chapter for instance.

\input{content/galois_descent}
\end{document}


\title{Brauer Group and Galois Cohomology}
\author{Yunzhou Xie \and Jujian Zhang}

\begin{document}
\maketitle
% In this file you should put the actual content of the blueprint.
% It will be used both by the web and the print version.
% It should *not* include the \begin{document}
%
% If you want to split the blueprint content into several files then
% the current file can be a simple sequence of \input. Otherwise It
% can start with a \section or \chapter for instance.

\chapter{Galois Descent}

\section{Preliminary results}
Let $k$ be a field and $V$ a $k$-vector space. We denote by $V^{\star}$ the dual space of $V$.

\begin{lemma}\label{lem:dual-tensor-power}
  There is a map $(V^{\star})^{\otimes p} \to (V^{\otimes p})^\star$
  \leanok
  \lean{dualTensorPower}
\end{lemma}
\begin{proof}
Indeed, let $f_1 \otimes, \dots, \otimes f_p \in (V^{\star})^{\otimes p}$ and $v_1 \otimes, \dots, \otimes v_p \in V^{\otimes p}$. We define the map by considering the product $\prod_{i} f_i (v_i)$.
\end{proof}

\begin{lemma}[Extending scalars]\label{lem:extend-scalars}
  \leanok%
  \lean{PiTensorProduct.extendScalars, Module.Dual.extendScalars}
  Let $K/k$ be a field extension, then we can extend scalars:
  \begin{itemize}
    \item a $k$-linear map $\bigotimes_{i} V \to \bigotimes_{i} K \otimes_{k} V$;
    \item a $K$-linear map $K \otimes V^{\star} \to \left(K \otimes_{k} V\right)^{\star}$.
  \end{itemize}
\end{lemma}

\begin{proof}
  $\bigotimes_{i} v_{i} \mapsto \bigotimes_{i} (1\otimes v_{i})$ and $(a \otimes f) \mapsto (x \otimes v) \mapsto f(v)\cdot ax$ will do.
\end{proof}

\section{$(p, q)$-tensor}\label{sec:p-q-tensor}

Let $k$ be a field and $V$ a $k$-vector space.

\begin{definition}\label{def:tensor-of-type}%
  For any $p, q \in \mathbb{N}$, a $V$-tensor of type $(p,q)$ or a $(p, q)$-tensor is an element of the tensor product $V^{\otimes p} \otimes {(V^*)}^{\otimes q}$.
  \leanok%
  \lean{TensorOfType}
\end{definition}


\begin{lemma}
  If $v$ is a $V$-tensor of type $(p,q)$, then it induces a linear map $V^{\otimes q}\to V^{\otimes p}$.
  \leanok%
  \lean{TensorOfType.toHom}
\end{lemma}
\begin{proof}
  Let $v = x \otimes f$ where $f$ can be seen as ${(V^{\otimes q})}^{\star}$ under
  \cref{lem:dual-tensor-power} and $x \in V^{\otimes p}$, thus if $y \in V^{\otimes q}$, we can obtain an elment in $V^{\otimes p}$ by $f(y)\cdot x$
\end{proof}

Let $W$ be another $k$-vector space such that $e : V \cong W$.

\begin{lemma}\label{lem:tensor-of-type-congr}
  $e$ induces an isomorphism between $V$-tensor of type $(p,q)$ and $W$-tensor of type $(p, q)$. Furthermore, $\mathsf{1} : V \cong V$ induces the identity and the linear map induced by $e_{1}\circ e_{2}$ is equal to the composition of linear map induced by $e_{1}$ and $e_{2}$.
  \leanok%
  \lean{TensorOfType.congr}
\end{lemma}
\begin{proof}
  Indeed, $e$ induces an automorphism on $V^{\otimes p}$ while $e^{-1}$ induces an automorphism on $W^{\otimes q}$. Checking the functoriality is trivial.
\end{proof}

\begin{lemma}[extend scalars]\label{lem:extend-scalars-pq-tensor}
  \leanok%
  \lean{TensorOfType.extendScalars}
  Let $K/k$ be a field extension, $\Phi$ be $V$-tensor of type $(p,q)$ over $k$, then there is a $K\otimes V$-tensor of type $(p,q)$ over $K$.
\end{lemma}
\begin{proof}
  We construct a bilinear map $V^{\otimes p} \to (V^{\star})^{\otimes q} \to (K \otimes_{k} V)^{\otimes p}\otimes_{K} \left({{(K \otimes_{k} V)}^{\star}}\right)^{\otimes q}$
  We have a map $\alpha : (V^{\star})^{\otimes} q \to ((K \otimes_{k} V)^{\star})^{\otimes q}$ by composing $(V^{\star})^{\otimes q} \to (K \otimes_{k} V^{\star})^{\otimes q}$ and $(K \otimes_{k} V^{\star}) \to (K \otimes_{k} V)^{\star}$ from \cref{lem:extend-scalars}. Thus we can define $v \mapsto f \mapsto \hat{v}\otimes \alpha f$ where $\hat{v}$ is $v$ extended to $(K\otimes_{k} V)^{\otimes p}$ by \cref{lem:extend-scalars} again.
\end{proof}

\section{Vector space with tensor of type $(p,q)$}

Let $k$ be a field.

\begin{definition}
  A $k$-vector space with tensor of type $(p, q)$ is a pair of $(V, \Phi)$ where $V$ is a $k$-vector space and $\Phi$ a $V$-tensor of type $(p, q)$.
  \leanok%
  \lean{VectorSpaceWithTensorOfType}
\end{definition}

\begin{definition}
  An equivalence between $k$-vector spaces $(V, \Phi)$ and $(W, \Psi)$ with tensor of type $(p, q)$ is a $k$-equivalence $e : V \cong W$ such that $e(\Phi) = \Psi$ where $e$ is seen as an isomorphism between $V$-tensor of type $(p, q)$ and $W$-tensor of type $(p, q)$ under \cref{lem:tensor-of-type-congr}.
  \leanok%
\end{definition}

\begin{lemma}
  The notion of equivalence between $k$-vector spaces with tensor of type $(p, q)$ is an equivalence relation:
  \begin{itemize}
    \item $(V, \Phi)$ is equivalent to itself.
    \item If $(V, \Phi)$ is equivalent to $(W, \Psi)$, then $(W, \Psi)$ is equivalent to $(V, \Phi)$.
    \item If $(V_{1}, \Phi_{1})$ is equivalent to $(V_{2}, \Psi_{2})$ and $(V_{2}, \Psi_{2})$ is equivalent to $(V_{3}, \Psi_{3})$, then $(V_{1}, \Phi_{1})$ is equivalent to $(V_{3}, \Psi_{3})$.
  \end{itemize}
  \leanok%
  \lean{VectorSpaceWithTensorOfType.Equiv.refl, VectorSpaceWithTensorOfType.Equiv.symm,  VectorSpaceWithTensorOfType.Equiv.trans}
\end{lemma}

\begin{lemma}[extend scalars]\label{lem:extend-scalars-vector-space-with-tensor}
  \leanok%
  \lean{VectorSpaceWithTensorOfType.extendScalars}
  Let $K/k$ be a field extension and $(V, \Phi)$ a $k$-vector space with tensor of type $(p,q)$, then $(V, \Phi)$ can be extended to a $K$-vector space with tensor of type $(p,q)$.
\end{lemma}
\begin{proof}
  Indeed, $(K \otimes_{k} V, \Phi_{K})$ works where $\Phi_{K}$ is obtained via~\cref{lem:extend-scalars-pq-tensor}.
\end{proof}

%%% Local Variables:
%%% mode: LaTeX
%%% TeX-master: "../print"
%%% End:

\end{document}


\title{Brauer Group and Galois Cohomology} \author{Jujian Zhang \and Yunzhou
  Xie}

\begin{document}
\maketitle
\tableofcontents


\section*{Preface}
\addcontentsline{toc}{section}{Preface}
In this exposition, we describe a excruciatingly detailed proof of the following theorem:
\begin{theorem*}
  For a finite dimensional and Galois field extension $K/F$, the relative Brauer group $\br(K/F)$ is isomorphic to the second group cohomology $\HH^{2}\left(\gal(K/F),K^{\star}\right)$.
\end{theorem*}

\paragraph{} The reason for the detailed-ness is because we are aiming to formalise the proof described in the following chapters; therefore the more details, the better. We apologise for the unconventional organisation in advance --- earlier chapters sometimes use results from later chapter. For our defence, we try to categorise all the results by topics and, since this is a formalisation project, we can guarantee the readers that there is no circular reasoning.

% In this file you should put the actual content of the blueprint.
% It will be used both by the web and the print version.
% It should *not* include the \begin{document}
%
% If you want to split the blueprint content into several files then
% the current file can be a simple sequence of \input. Otherwise It
% can start with a \section or \chapter for instance.

\chapter{Galois Descent}

\section{Preliminary results}
Let $k$ be a field and $V$ a $k$-vector space. We denote by $V^{\star}$ the dual space of $V$.

\begin{lemma}\label{lem:dual-tensor-power}
  There is a map $(V^{\star})^{\otimes p} \to (V^{\otimes p})^\star$
  \leanok
  \lean{dualTensorPower}
\end{lemma}
\begin{proof}
Indeed, let $f_1 \otimes, \dots, \otimes f_p \in (V^{\star})^{\otimes p}$ and $v_1 \otimes, \dots, \otimes v_p \in V^{\otimes p}$. We define the map by considering the product $\prod_{i} f_i (v_i)$.
\end{proof}

\begin{lemma}[Extending scalars]\label{lem:extend-scalars}
  \leanok%
  \lean{PiTensorProduct.extendScalars, Module.Dual.extendScalars}
  Let $K/k$ be a field extension, then we can extend scalars:
  \begin{itemize}
    \item a $k$-linear map $\bigotimes_{i} V \to \bigotimes_{i} K \otimes_{k} V$;
    \item a $K$-linear map $K \otimes V^{\star} \to \left(K \otimes_{k} V\right)^{\star}$.
  \end{itemize}
\end{lemma}

\begin{proof}
  $\bigotimes_{i} v_{i} \mapsto \bigotimes_{i} (1\otimes v_{i})$ and $(a \otimes f) \mapsto (x \otimes v) \mapsto f(v)\cdot ax$ will do.
\end{proof}

\section{$(p, q)$-tensor}\label{sec:p-q-tensor}

Let $k$ be a field and $V$ a $k$-vector space.

\begin{definition}\label{def:tensor-of-type}%
  For any $p, q \in \mathbb{N}$, a $V$-tensor of type $(p,q)$ or a $(p, q)$-tensor is an element of the tensor product $V^{\otimes p} \otimes {(V^*)}^{\otimes q}$.
  \leanok%
  \lean{TensorOfType}
\end{definition}


\begin{lemma}
  If $v$ is a $V$-tensor of type $(p,q)$, then it induces a linear map $V^{\otimes q}\to V^{\otimes p}$.
  \leanok%
  \lean{TensorOfType.toHom}
\end{lemma}
\begin{proof}
  Let $v = x \otimes f$ where $f$ can be seen as ${(V^{\otimes q})}^{\star}$ under
  \cref{lem:dual-tensor-power} and $x \in V^{\otimes p}$, thus if $y \in V^{\otimes q}$, we can obtain an elment in $V^{\otimes p}$ by $f(y)\cdot x$
\end{proof}

Let $W$ be another $k$-vector space such that $e : V \cong W$.

\begin{lemma}\label{lem:tensor-of-type-congr}
  $e$ induces an isomorphism between $V$-tensor of type $(p,q)$ and $W$-tensor of type $(p, q)$. Furthermore, $\mathsf{1} : V \cong V$ induces the identity and the linear map induced by $e_{1}\circ e_{2}$ is equal to the composition of linear map induced by $e_{1}$ and $e_{2}$.
  \leanok%
  \lean{TensorOfType.congr}
\end{lemma}
\begin{proof}
  Indeed, $e$ induces an automorphism on $V^{\otimes p}$ while $e^{-1}$ induces an automorphism on $W^{\otimes q}$. Checking the functoriality is trivial.
\end{proof}

\begin{lemma}[extend scalars]\label{lem:extend-scalars-pq-tensor}
  \leanok%
  \lean{TensorOfType.extendScalars}
  Let $K/k$ be a field extension, $\Phi$ be $V$-tensor of type $(p,q)$ over $k$, then there is a $K\otimes V$-tensor of type $(p,q)$ over $K$.
\end{lemma}
\begin{proof}
  We construct a bilinear map $V^{\otimes p} \to (V^{\star})^{\otimes q} \to (K \otimes_{k} V)^{\otimes p}\otimes_{K} \left({{(K \otimes_{k} V)}^{\star}}\right)^{\otimes q}$
  We have a map $\alpha : (V^{\star})^{\otimes} q \to ((K \otimes_{k} V)^{\star})^{\otimes q}$ by composing $(V^{\star})^{\otimes q} \to (K \otimes_{k} V^{\star})^{\otimes q}$ and $(K \otimes_{k} V^{\star}) \to (K \otimes_{k} V)^{\star}$ from \cref{lem:extend-scalars}. Thus we can define $v \mapsto f \mapsto \hat{v}\otimes \alpha f$ where $\hat{v}$ is $v$ extended to $(K\otimes_{k} V)^{\otimes p}$ by \cref{lem:extend-scalars} again.
\end{proof}

\section{Vector space with tensor of type $(p,q)$}

Let $k$ be a field.

\begin{definition}
  A $k$-vector space with tensor of type $(p, q)$ is a pair of $(V, \Phi)$ where $V$ is a $k$-vector space and $\Phi$ a $V$-tensor of type $(p, q)$.
  \leanok%
  \lean{VectorSpaceWithTensorOfType}
\end{definition}

\begin{definition}
  An equivalence between $k$-vector spaces $(V, \Phi)$ and $(W, \Psi)$ with tensor of type $(p, q)$ is a $k$-equivalence $e : V \cong W$ such that $e(\Phi) = \Psi$ where $e$ is seen as an isomorphism between $V$-tensor of type $(p, q)$ and $W$-tensor of type $(p, q)$ under \cref{lem:tensor-of-type-congr}.
  \leanok%
\end{definition}

\begin{lemma}
  The notion of equivalence between $k$-vector spaces with tensor of type $(p, q)$ is an equivalence relation:
  \begin{itemize}
    \item $(V, \Phi)$ is equivalent to itself.
    \item If $(V, \Phi)$ is equivalent to $(W, \Psi)$, then $(W, \Psi)$ is equivalent to $(V, \Phi)$.
    \item If $(V_{1}, \Phi_{1})$ is equivalent to $(V_{2}, \Psi_{2})$ and $(V_{2}, \Psi_{2})$ is equivalent to $(V_{3}, \Psi_{3})$, then $(V_{1}, \Phi_{1})$ is equivalent to $(V_{3}, \Psi_{3})$.
  \end{itemize}
  \leanok%
  \lean{VectorSpaceWithTensorOfType.Equiv.refl, VectorSpaceWithTensorOfType.Equiv.symm,  VectorSpaceWithTensorOfType.Equiv.trans}
\end{lemma}

\begin{lemma}[extend scalars]\label{lem:extend-scalars-vector-space-with-tensor}
  \leanok%
  \lean{VectorSpaceWithTensorOfType.extendScalars}
  Let $K/k$ be a field extension and $(V, \Phi)$ a $k$-vector space with tensor of type $(p,q)$, then $(V, \Phi)$ can be extended to a $K$-vector space with tensor of type $(p,q)$.
\end{lemma}
\begin{proof}
  Indeed, $(K \otimes_{k} V, \Phi_{K})$ works where $\Phi_{K}$ is obtained via~\cref{lem:extend-scalars-pq-tensor}.
\end{proof}

%%% Local Variables:
%%% mode: LaTeX
%%% TeX-master: "../print"
%%% End:


\end{document}
