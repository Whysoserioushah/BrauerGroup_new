% This file makes a printable version of the blueprint
% It should include all the \usepackage needed for the pdf version.
% The template version assume you want to use a modern TeX compiler
% such as xeLaTeX or luaLaTeX including support for unicode
% and Latin Modern Math font with standard bugfixes applied.
% It also uses expl3 in order to support macros related to the dependency graph.
% It also includes standard AMS packages (and their improved version
% mathtools) as well as support for links with a sober decoration
% (no ugly rectangles around links).
% It is otherwise a very minimal preamble (you should probably at least
% add cleveref and tikz-cd).

\documentclass[a4paper]{report}

% \setcounter{secnumdepth}{3}
\setcounter{tocdepth}{3}

\usepackage{geometry}

\usepackage{expl3}

\usepackage{stmaryrd}

\usepackage{tikz-cd}

\usepackage{amssymb, amsthm, mathtools}
\usepackage[unicode,colorlinks=true,linkcolor=blue,urlcolor=magenta,
citecolor=blue]{hyperref}

\usepackage{beton} \usepackage[euler-digits,euler-hat-accent]{eulervm}

% \usepackage[warnings-off={mathtools-colon,mathtools-overbracket}]{unicode-math}

\usepackage{cleveref}

% In this file you should put all LaTeX macros to be used
% both by the pdf version and the web version.
% This should be most of your macros.

\theoremstyle{plain}
\newtheorem*{theorem*}{Theorem}
\newtheorem{theorem}{Theorem}[section]
\newtheorem{lemma}[theorem]{Lemma}
\newtheorem{corollary}[theorem]{Corollary}

\newtheorem{remark}[theorem]{Remark}

\theoremstyle{definition}
\newtheorem{definition}{Definition}[section]
\newtheorem{construction}[definition]{Construction}

\DeclareMathOperator{\Mat}{Mat}
\DeclareMathOperator{\mat}{Mat}
\DeclareMathOperator{\End}{End}
\DeclareMathOperator{\CSA}{\mathsf{C}\!\mathsf{S}\!\mathsf{A}}
\DeclareMathOperator{\csa}{\mathsf{C}\!\mathsf{S}\!\mathsf{A}}

\DeclareMathOperator{\Br}{Br}
\DeclareMathOperator{\br}{Br}

\DeclareMathOperator{\MOD}{\mathfrak{Mod}}

\DeclareMathOperator{\range}{range}
\DeclareMathOperator{\im}{im}

\DeclareMathOperator{\diag}{diag}

\newcommand{\opp}{\mathsf{opp}}

\DeclareMathOperator{\HH}{H}
\DeclareMathOperator{\hh}{H}
\DeclareMathOperator{\gal}{Gal}
\DeclareMathOperator{\twist}{twist}
\DeclareMathOperator{\comp}{comp}

\newcommand{\McA}{\mathcal{A}}
\newcommand{\McB}{\mathcal{B}}
\newcommand{\McL}{\mathcal{L}}
\newcommand{\McR}{\mathcal{R}}
\newcommand{\McZ}{\mathcal{Z}}

\newcommand{\mcA}{\mathcal{A}}
\newcommand{\mcB}{\mathcal{B}}
\newcommand{\mcL}{\mathcal{L}}
\newcommand{\mcR}{\mathcal{R}}
\newcommand{\mcZ}{\mathcal{Z}}

\newcommand{\ox}{\otimes}
\newcommand{\Ox}{\bigotimes}

\newcommand{\id}{\mathsf{id}}

\newcommand{\mfa}{\mathfrak{a}}
\newcommand{\mfb}{\mathfrak{b}}
\newcommand{\mfc}{\mathfrak{c}}
\DeclareMathOperator{\cross}{\mathfrak{C}}

\newcommand{\nat}{\mathbb{N}}
%%% Local Variables:
%%% mode: LaTeX
%%% TeX-master: "../print"
%%% End:
 % This file makes a printable version of the blueprint
% It should include all the \usepackage needed for the pdf version.
% The template version assume you want to use a modern TeX compiler
% such as xeLaTeX or luaLaTeX including support for unicode
% and Latin Modern Math font with standard bugfixes applied.
% It also uses expl3 in order to support macros related to the dependency graph.
% It also includes standard AMS packages (and their improved version
% mathtools) as well as support for links with a sober decoration
% (no ugly rectangles around links).
% It is otherwise a very minimal preamble (you should probably at least
% add cleveref and tikz-cd).

\documentclass[a4paper]{report}

% \setcounter{secnumdepth}{3}
\setcounter{tocdepth}{3}

\usepackage{geometry}

\usepackage{expl3}

\usepackage{stmaryrd}

\usepackage{tikz-cd}

\usepackage{amssymb, amsthm, mathtools}
\usepackage[unicode,colorlinks=true,linkcolor=blue,urlcolor=magenta,
citecolor=blue]{hyperref}

\usepackage{beton} \usepackage[euler-digits,euler-hat-accent]{eulervm}

% \usepackage[warnings-off={mathtools-colon,mathtools-overbracket}]{unicode-math}

\usepackage{cleveref}

% In this file you should put all LaTeX macros to be used
% both by the pdf version and the web version.
% This should be most of your macros.

\theoremstyle{plain}
\newtheorem*{theorem*}{Theorem}
\newtheorem{theorem}{Theorem}[section]
\newtheorem{lemma}[theorem]{Lemma}
\newtheorem{corollary}[theorem]{Corollary}

\newtheorem{remark}[theorem]{Remark}

\theoremstyle{definition}
\newtheorem{definition}{Definition}[section]
\newtheorem{construction}[definition]{Construction}

\DeclareMathOperator{\Mat}{Mat}
\DeclareMathOperator{\mat}{Mat}
\DeclareMathOperator{\End}{End}
\DeclareMathOperator{\CSA}{\mathsf{C}\!\mathsf{S}\!\mathsf{A}}
\DeclareMathOperator{\csa}{\mathsf{C}\!\mathsf{S}\!\mathsf{A}}

\DeclareMathOperator{\Br}{Br}
\DeclareMathOperator{\br}{Br}

\DeclareMathOperator{\MOD}{\mathfrak{Mod}}

\DeclareMathOperator{\range}{range}
\DeclareMathOperator{\im}{im}

\DeclareMathOperator{\diag}{diag}

\newcommand{\opp}{\mathsf{opp}}

\DeclareMathOperator{\HH}{H}
\DeclareMathOperator{\hh}{H}
\DeclareMathOperator{\gal}{Gal}
\DeclareMathOperator{\twist}{twist}
\DeclareMathOperator{\comp}{comp}

\newcommand{\McA}{\mathcal{A}}
\newcommand{\McB}{\mathcal{B}}
\newcommand{\McL}{\mathcal{L}}
\newcommand{\McR}{\mathcal{R}}
\newcommand{\McZ}{\mathcal{Z}}

\newcommand{\mcA}{\mathcal{A}}
\newcommand{\mcB}{\mathcal{B}}
\newcommand{\mcL}{\mathcal{L}}
\newcommand{\mcR}{\mathcal{R}}
\newcommand{\mcZ}{\mathcal{Z}}

\newcommand{\ox}{\otimes}
\newcommand{\Ox}{\bigotimes}

\newcommand{\id}{\mathsf{id}}

\newcommand{\mfa}{\mathfrak{a}}
\newcommand{\mfb}{\mathfrak{b}}
\newcommand{\mfc}{\mathfrak{c}}
\DeclareMathOperator{\cross}{\mathfrak{C}}

\newcommand{\nat}{\mathbb{N}}
%%% Local Variables:
%%% mode: LaTeX
%%% TeX-master: "../print"
%%% End:
 % This file makes a printable version of the blueprint
% It should include all the \usepackage needed for the pdf version.
% The template version assume you want to use a modern TeX compiler
% such as xeLaTeX or luaLaTeX including support for unicode
% and Latin Modern Math font with standard bugfixes applied.
% It also uses expl3 in order to support macros related to the dependency graph.
% It also includes standard AMS packages (and their improved version
% mathtools) as well as support for links with a sober decoration
% (no ugly rectangles around links).
% It is otherwise a very minimal preamble (you should probably at least
% add cleveref and tikz-cd).

\documentclass[a4paper]{report}

% \setcounter{secnumdepth}{3}
\setcounter{tocdepth}{3}

\usepackage{geometry}

\usepackage{expl3}

\usepackage{stmaryrd}

\usepackage{tikz-cd}

\usepackage{amssymb, amsthm, mathtools}
\usepackage[unicode,colorlinks=true,linkcolor=blue,urlcolor=magenta,
citecolor=blue]{hyperref}

\usepackage{beton} \usepackage[euler-digits,euler-hat-accent]{eulervm}

% \usepackage[warnings-off={mathtools-colon,mathtools-overbracket}]{unicode-math}

\usepackage{cleveref}

% In this file you should put all LaTeX macros to be used
% both by the pdf version and the web version.
% This should be most of your macros.

\theoremstyle{plain}
\newtheorem*{theorem*}{Theorem}
\newtheorem{theorem}{Theorem}[section]
\newtheorem{lemma}[theorem]{Lemma}
\newtheorem{corollary}[theorem]{Corollary}

\newtheorem{remark}[theorem]{Remark}

\theoremstyle{definition}
\newtheorem{definition}{Definition}[section]
\newtheorem{construction}[definition]{Construction}

\DeclareMathOperator{\Mat}{Mat}
\DeclareMathOperator{\mat}{Mat}
\DeclareMathOperator{\End}{End}
\DeclareMathOperator{\CSA}{\mathsf{C}\!\mathsf{S}\!\mathsf{A}}
\DeclareMathOperator{\csa}{\mathsf{C}\!\mathsf{S}\!\mathsf{A}}

\DeclareMathOperator{\Br}{Br}
\DeclareMathOperator{\br}{Br}

\DeclareMathOperator{\MOD}{\mathfrak{Mod}}

\DeclareMathOperator{\range}{range}
\DeclareMathOperator{\im}{im}

\DeclareMathOperator{\diag}{diag}

\newcommand{\opp}{\mathsf{opp}}

\DeclareMathOperator{\HH}{H}
\DeclareMathOperator{\hh}{H}
\DeclareMathOperator{\gal}{Gal}
\DeclareMathOperator{\twist}{twist}
\DeclareMathOperator{\comp}{comp}

\newcommand{\McA}{\mathcal{A}}
\newcommand{\McB}{\mathcal{B}}
\newcommand{\McL}{\mathcal{L}}
\newcommand{\McR}{\mathcal{R}}
\newcommand{\McZ}{\mathcal{Z}}

\newcommand{\mcA}{\mathcal{A}}
\newcommand{\mcB}{\mathcal{B}}
\newcommand{\mcL}{\mathcal{L}}
\newcommand{\mcR}{\mathcal{R}}
\newcommand{\mcZ}{\mathcal{Z}}

\newcommand{\ox}{\otimes}
\newcommand{\Ox}{\bigotimes}

\newcommand{\id}{\mathsf{id}}

\newcommand{\mfa}{\mathfrak{a}}
\newcommand{\mfb}{\mathfrak{b}}
\newcommand{\mfc}{\mathfrak{c}}
\DeclareMathOperator{\cross}{\mathfrak{C}}

\newcommand{\nat}{\mathbb{N}}
%%% Local Variables:
%%% mode: LaTeX
%%% TeX-master: "../print"
%%% End:
 % This file makes a printable version of the blueprint
% It should include all the \usepackage needed for the pdf version.
% The template version assume you want to use a modern TeX compiler
% such as xeLaTeX or luaLaTeX including support for unicode
% and Latin Modern Math font with standard bugfixes applied.
% It also uses expl3 in order to support macros related to the dependency graph.
% It also includes standard AMS packages (and their improved version
% mathtools) as well as support for links with a sober decoration
% (no ugly rectangles around links).
% It is otherwise a very minimal preamble (you should probably at least
% add cleveref and tikz-cd).

\documentclass[a4paper]{report}

% \setcounter{secnumdepth}{3}
\setcounter{tocdepth}{3}

\usepackage{geometry}

\usepackage{expl3}

\usepackage{stmaryrd}

\usepackage{tikz-cd}

\usepackage{amssymb, amsthm, mathtools}
\usepackage[unicode,colorlinks=true,linkcolor=blue,urlcolor=magenta,
citecolor=blue]{hyperref}

\usepackage{beton} \usepackage[euler-digits,euler-hat-accent]{eulervm}

% \usepackage[warnings-off={mathtools-colon,mathtools-overbracket}]{unicode-math}

\usepackage{cleveref}

\input{macros/common} \input{macros/print}

\title{Brauer Group and Galois Cohomology} \author{Jujian Zhang \and Yunzhou
  Xie}

\begin{document}
\maketitle
\tableofcontents


\section*{Preface}
\addcontentsline{toc}{section}{Preface}
In this exposition, we describe a excruciatingly detailed proof of the following theorem:
\begin{theorem*}
  For a finite dimensional and Galois field extension $K/F$, the relative Brauer group $\br(K/F)$ is isomorphic to the second group cohomology $\HH^{2}\left(\gal(K/F),K^{\star}\right)$.
\end{theorem*}

\paragraph{} The reason for the detailed-ness is because we are aiming to formalise the proof described in the following chapters; therefore the more details, the better. We apologise for the unconventional organisation in advance --- earlier chapters sometimes use results from later chapter. For our defence, we try to categorise all the results by topics and, since this is a formalisation project, we can guarantee the readers that there is no circular reasoning.

\input{content}

\end{document}


\title{Brauer Group and Galois Cohomology} \author{Jujian Zhang \and Yunzhou
  Xie}

\begin{document}
\maketitle
\tableofcontents


\section*{Preface}
\addcontentsline{toc}{section}{Preface}
In this exposition, we describe a excruciatingly detailed proof of the following theorem:
\begin{theorem*}
  For a finite dimensional and Galois field extension $K/F$, the relative Brauer group $\br(K/F)$ is isomorphic to the second group cohomology $\HH^{2}\left(\gal(K/F),K^{\star}\right)$.
\end{theorem*}

\paragraph{} The reason for the detailed-ness is because we are aiming to formalise the proof described in the following chapters; therefore the more details, the better. We apologise for the unconventional organisation in advance --- earlier chapters sometimes use results from later chapter. For our defence, we try to categorise all the results by topics and, since this is a formalisation project, we can guarantee the readers that there is no circular reasoning.

\input{content/central_simple}
\input{content/morita}
\chapter{Results in Noncommutative
  Algebra}\label{sec:results-noncommutative-algebra}
\input{content/unclassified}
\input{content/wed_artin}
\input{content/skolem_noether}
\input{content/double_centralizer}
\input{content/brauer-group}
\input{content/brauer-group-2}

%%% Local Variables:
%%% mode: LaTeX
%%% TeX-master: "print"
%%% End:


\end{document}


\title{Brauer Group and Galois Cohomology} \author{Jujian Zhang \and Yunzhou
  Xie}

\begin{document}
\maketitle
\tableofcontents


\section*{Preface}
\addcontentsline{toc}{section}{Preface}
In this exposition, we describe a excruciatingly detailed proof of the following theorem:
\begin{theorem*}
  For a finite dimensional and Galois field extension $K/F$, the relative Brauer group $\br(K/F)$ is isomorphic to the second group cohomology $\HH^{2}\left(\gal(K/F),K^{\star}\right)$.
\end{theorem*}

\paragraph{} The reason for the detailed-ness is because we are aiming to formalise the proof described in the following chapters; therefore the more details, the better. We apologise for the unconventional organisation in advance --- earlier chapters sometimes use results from later chapter. For our defence, we try to categorise all the results by topics and, since this is a formalisation project, we can guarantee the readers that there is no circular reasoning.

\chapter{Central Simple Algebras}\label{sec:csa}

In this chapter we define central simple algebras. We used some results
in~\cref{sec:unclassified}.

\begin{definition}[Simple Ring] A ring $R$ is simple if the only
  two-sided-ideals of $R$ are ${0}$ and $R$. An algebra is simple if it is
  simple as a ring. \leanok \lean{IsSimpleRing}
\end{definition}

\begin{lemma}
  \label{lem:center-simple-ring}
  Let $A$ be a simple ring, then centre of $A$ is a field. \leanok
  \lean{IsSimpleRing.isField_center}
\end{lemma}
\begin{proof}
  Let $0\ne x$ be an element of centre of $A$. Then $I := \{xy | y\in A\}$ is a
  two-sided-ideal of $A$. Since $0\ne x\in I$, we have that $I = A$. Therefore
  $1 \in I$, hence $x$ is invertible.
\end{proof}

\begin{definition}[Central Algebras]
  Let $R$ be a ring and $A$ an $R$-algebra, we say $A$ is central if and only if
  the centre of $A$ is $R$ \leanok \lean{Algebra.IsCentral}
\end{definition}

\begin{remark}
  Simpleness is invariant under ring isomorphism and centrality is invariant
  under algebra isomorphism.
\end{remark}

\begin{lemma}\label{lem:simple-ring-iff}
  $R$ is a simple ring if and only if any ring homomorphism $f : R \to S$ either
  injective or $S$ is the trivial ring. \leanok
  \lean{IsSimpleRing.iff_injective_ringHom_or_subsingleton_codomain}
\end{lemma}
\begin{proof}
  If $R$ is simple, then the $\ker f$ is either $\{0\}$ or $R$. The former case
  implies that $f$ is injective while the latter case implies that $S$ is the
  trivial ring. Conversely, let $I\subseteq R$ be a two-sided-ideal. Consider
  $\pi: R \to {}^{R}/_{I}$, either $\pi$ is injective implying that $I = \{0\}$
  or that ${}^{R}/_{I}$ is the trivial ring implying that $I = R$.
\end{proof}

\begin{corollary}\label{cor:alghom-bijective-of-dim-eq}
  Assume $R$ is a field. Let $A, B$ be finite dimensional $R$-algebras where $A$
  is simple as well. Then any $R$-algebra homomorphism $f:A\to B$ is bijective
  if $\dim_{R}A=\dim_{R}B$. \leanok
  \lean{bijective_of_dim_eq_of_isCentralSimple}
\end{corollary}
\begin{proof}
  By~\cref{lem:simple-ring-iff}, $f$ is injective. Then
  $\dim_{K}\im f = \dim_{K} B - \dim_{K}\ker f = \dim_{K}B$ meaning that $f$ is
  surjective.
\end{proof}

Let $K$ be a field and $A, B$ be $K$-algebras.
\begin{lemma}
  \label{lem:tensor-central}
  If $A$ and $B$ are central $K$-algebras, $A\otimes_{K}B$ is a central
  $K$-algebra as well. \leanok \lean{IsCentralSimple.TensorProduct.isCentral}
\end{lemma}
\begin{proof}
  Assume $A$ and $B$ are central algebras, then
  by~\cref{cor:center-tensor-tensor}
  $Z\left(A\otimes_{R}B\right)=Z\left(A\right)\otimes_{R}Z\left(B\right)=R\otimes_{R}R=R$.
\end{proof}

\begin{theorem}
  \label{thm:tensor-csa}
  If $A$ is a simple $K$-algebra and $B$ is a central simple $K$-algebra,
  $A\otimes_{K}B$ is a central simple $K$-algebra as well. \leanok
  \lean{IsCentralSimple.TensorProduct.simple}
\end{theorem}
\begin{proof}
  By~\cref{lem:tensor-central}, we need to prove $A\otimes_{K}B$ is a simple
  ring. Denote $f$ as the map $A\to A\otimes_{K}B$. It is sufficient to prove
  that for every two-sided-ideal $I\subseteq A\otimes_{K}B$, we have
  $I = \left\langle f\left(f^{-1}\left(I\right)\right)\right\rangle$. Indeed,
  since $A$ is simple $f^{-1}\left(I\right)$ is either $\left\{0\right\}$ or
  $A$, if it is $\left\{0\right\}$, then $I=\left\{0\right\}$; if it is $A$,
  then $I$ is $A$ as well.

  We will prove that
  $I \le \left\langle f\left(f^{-1}\left(I\right)\right)\right\rangle$, the
  other direction is straightforward. Without loss of generality assume
  $I\ne\left\{0\right\}$. Let $\McA$ be an arbitrary basis of $A$, by
  \cref{lem:expand-tensor-in-basis}, we see that every element
  $x \in A\otimes_{K}B$ can be written as $\sum_{i=0}^{n}\McA_{i}\otimes b_{i}$
  for some natrual number $n$ and some choice of $b_{i}\in B$ and
  $\McA_{i}\in \McA$. Since $I$ is not empty, we see there exists a non-zero
  element $\omega\in I$ such that its expansion
  $\sum_{i=0}^{n}\McA_{i}\otimes b_{i}$ has the minimal $n$. In particular, all
  $b_{i}$ are non-zero and $n\ne0$. We have
  $\omega=\McA_{0}\otimes b_{0}+\sum_{i=1}^{n}\McA_{i}\otimes b_{i}$. Since $B$
  is simple, $1 \in B = \left\langle\langle b_{0} \right\rangle$; hence we write
  $1\in\sum_{j=0}^{m}x_{i}b_{0}y_{i}$ for some $x_{i},y_{i}\in B$. Define
  $\Omega := \sum_{j=0}^{m}(1\otimes x_{i})\omega(1\otimes y_{i})$ which is also
  in $I$. We write
  \[
    \begin{aligned}
      \Omega &= \McA_{0} \otimes \left(\sum_{j=0}^{m}x_{j}b_{0}y_{j}\right) + \sum_{i=1}^{n}\McA_{i}\otimes\left(\sum_{j=0}^{m}x_{j}b_{i}y_{j}\right) \\
             &= \McA_{0}\otimes 1 + \sum_{i=1}^{n}\McA_{i}\otimes\left(\sum_{j=0}^{m}x_{j}b_{i}y_{j}\right)
    \end{aligned}
  \]
  For every $\beta\in B$, we have that
  $\left(1\otimes \beta\right)\Omega - \Omega\left(1\otimes \beta\right)$ is in
  I and is equal to
  \[
    \sum_{i=1}^{n}\McA_{i}\otimes\left(\sum_{j=0}^{m}\beta x_{j}b_{i}y_{j}-x_{j}b_{i}y_{j}\beta\right),
  \]
  which is an expansion of $n-1$ terms, thus
  $\left(1\otimes\beta\right)\Omega-\Omega\left(1\otimes\beta\right)$ must be
  $0$. Hence we conclude that for all $i=1,\dots,n$,
  $\sum_{j=0}^{m}x_{j}b_{i}y_{j}\in Z\left(B\right)=K$. Hence for all
  $i=1,\dots,n$, we find a $\kappa_{i}\in K$ such that
  $\kappa_{i}=\sum_{j=0}^{m}x_{j}b_{i}y_{j}$. Hence we can calculate $\Omega$ as
  \[
    \begin{aligned}
      \Omega &= \McA_{0}\otimes 1 + \sum_{i=1}^{n}\McA_{i}\otimes\left(\sum_{j=0}^{m}\right) \\
             &= \McA_{0}\otimes 1 + \sum_{i=1}^{n}\McA_{i}\otimes\kappa_{i} \\
             &= \left(\McA_{0}+\sum_{i=1}^{n}\kappa_{i}\cdot\McA_{i}\right) \otimes 1
    \end{aligned}.
  \]
  From this, we note that
  $\McA_{0}+\sum_{i}^{n}\kappa_{i}\cdot\McA_{i}\in f^{-1}\left(I\right)$; since
  $A$ is simple, we immediately conclude that $f^{-1}\left(I\right) = A$, once
  we know $\McA_{0}+\sum_{i=1}^{n}\kappa_{i}\cdot \McA_{i}$ is not zero. If it
  is zero, by the fact that $\McA$ is a linearly independent set, we conclude
  that $1,\kappa_{1},\dots,\kappa_{n}$ are all zero; which is a contradiction.
  Since $f^{-1}\left(I\right) = A$, we know
  $\left\langle f\left(f^{-1}I\right)\right\rangle = A \otimes_{K} B$.
\end{proof}

\begin{corollary}\label{lem:cas-basechange}
  Central simple algebras are stable under base change. That is, if $L/K$ is a
  field extension and $D$ is a central simple algebra over $K$, then
  $L\otimes_{K} D$ is central simple over $L$. \leanok
  \lean{IsCentralSimple.baseChange}
\end{corollary}

\begin{proof}
  By~\cref{thm:tensor-csa}, $L\otimes_{K} D$ is simple. Let
  $x\in Z\left(L\otimes_{K}D\right)$, by~\cref{cor:center-tensor-center}, we
  have $x\in Z\left(L\right)\otimes Z\left(D\right)=Z\left(L\right)$. Without
  loss of generality, we can assume that $x = l \otimes d$ is a pure tensor,
  then $l \in Z\left(L\right)$ and $d \in K$. Therefore $x = d\cdot l \in L$.
\end{proof}

%%% Local Variables:
%%% mode: LaTeX
%%% TeX-master: "../print"
%%% End:

\chapter{Morita Equivalence}\label{chap:morita}

This chapter intertwine with \cref{chap:wed-artin}: \cref{sec:stacks-074e} depends on \cref{sec:wed-artin-proof}; while \cref{sec:wed-artin-unique} depends on \cref{sec:stacks-074e}.

\section{Construction of the equivalence}\label{sec:morita-construction}

Let $R$ be a ring and $0 \ne n\in \mathbb{N}$. In this chapter, we prove that the category $R$-modules and the category of $\Mat_{n}(R)$-modules are equivalent. Then we use the equivalence to prove several useful lemmas.

\begin{construction}\label{con:morita-eqv-functor0}
  \leanok
  \lean{matrix_smul_vec_def}
  If $M$ is an $R$-module, we have a natural $\Mat_{n}(R)$-module structure on $\hat{M}:=M^{n}$ given by $(m_{ij})\cdot (v_{k})=\sum_{j}m_{ij}\cdot v_{j}$.
  If $f : M \to N$ is an $R$-linear map, then $\hat{f} : M^{n}\to N^{n}$ given by $(v_{i}) \mapsto (f(v_{i}))$ is a $\Mat_{n}(R)$-linear map. Thus we have a well-defined functor $\MOD_{R} \Longrightarrow \MOD_{\Mat_{n}(R)}$.
\end{construction}

Note that all modules are assumed to be left modules; when we need to consider right $R$-modules, we will consider left $R^{\opp}$-modules instead. We use $\delta_{ij}$ to denote the matrix whose $(i,j)$-th entry is $1$ and $0$ elsewhere. $\delta_{ij}$ forms a basis for matrices.

\begin{construction}\label{con:morita-eqv-functor1}
  \leanok
  \lean{fromModuleCatOverMatrix.smul_α_coe}
  If $M$ is a $\Mat_{n}(R)$-module, then $\tilde{M} := \{\delta_{ij}\cdot m | m \in M\} \subseteq M$ is an $R$-module given by $r \cdot (\delta_{ij}\cdot m) := (r\cdot \delta_{ij})\cdot m$. More over if $f : M \to N$ is a $\Mat_{n}(R)$-linear map, $\tilde{f} : \tilde{M} \to \tilde{N}$ given by $f$ is $R$-linear. Hence, we have a functor $\MOD_{\Mat_{n}(R)}\Longrightarrow \MOD_{R}$.
\end{construction}

\begin{theorem}[Morita Equivalence]\label{thm:morita}
  The functors constructed in \cref{con:morita-eqv-functor0} and \cref{con:morita-eqv-functor1} form an equivalence of category.
  \leanok
  \lean{moritaEquivalentToMatrix}
\end{theorem}

\begin{proof}
  Let $M$ be an $R$-module, then the unit $\tilde{\hat{M}} \cong M$ is given by
  \[
    \begin{aligned}
      x \mapsto \sum_{j} x_{j} \\
      (x,0,\dots,0) \mapsfrom x
    \end{aligned}
  \]

  Let $M$ be an $\Mat_{n}(R)$-module, then the counit $\hat{\tilde{M}}\cong M$ is given by $m \mapsto (\delta_{i0}\cdot m)$. This map is both injective and surjective.
\end{proof}

\section{Stacks 074E}\label{sec:stacks-074e}

Let $A$ be a finite dimensional simple $k$-algebra.

\begin{lemma}
  Let $M$ and $N$ be simple $A$-modules, then $M$ and $N$ are isomorphic as $A$-modules.
  \leanok
  \lean{linearEquiv_of_isSimpleModule_over_simple_ring}
\end{lemma}

\begin{proof}
  By~\cref{thm:wed-artin-algebra}, there exists non-zero $n\in\mathbb N$, $k$-division algebra $D$ such that $A\cong \Mat_{n}(D)$ as $k$-algebras. Then by~\cref{thm:morita}, we have equivalence of category $e : \MOD_{A}\cong\MOD_{D}$. Since simple module is a categorical notion (it can be defined in terms monomorphisms), $e(M)$ and $e(N)$ are simple $D$-modules. Since $D$ is a division ring, $e(M)$ and $e(N)$ are isomorphic as $D$-modules, therefore $M$ and $N$ are isomorphic as $A$-modules.
\end{proof}

\begin{lemma}
  Let $M$ be an $A$-module, there exists a simple $A$-module $S$ such that $M$ is a direct sum of copies of $S$, i.e. $M \cong \bigoplus_{i \in \iota} S$ for some indexing set $\iota$.
  \leanok
  \lean{directSum_simple_module_over_simple_ring}
\end{lemma}

%%% Local Variables:
%%% mode: LaTeX
%%% TeX-master: "../print"
%%% End:

\chapter{Results in Noncommutative
  Algebra}\label{sec:results-noncommutative-algebra}
% \setcounter{chapter}{-1}
\section{A Collection of Useful Lemmas}\label{sec:unclassified}

In section, we collect some lemmas that does not really belong to anywhere.

\subsection{Tensor Product}

\begin{lemma}\label{lem:expand-tensor-in-basis}
  Let $M$ and $N$ be $R$-modules such that $\mathcal{C}_{i\in\iota}$ is a basis for $N$, then every elements of $x \in M \otimes_{R} N$ can be uniquely written as $\sum_{i\in\iota}m_{i}\otimes \mathcal{C}_{i}$ where only finitely many $m_{i}$'s are non-zero
\end{lemma}

\begin{proof}
  Given the basis $\mathcal{C}$, we have $R$-linear isomorphism $N \cong\bigoplus_{i\in\iota}R$, hence $M\otimes_{R}N \cong \bigoplus_{i\in\iota}(M\otimes_{R}R)\cong\bigoplus_{i\in\iota}M$ as $R$-modules.
\end{proof}
By switching $M$ and $N$, the symmetric statement goes without saying.

% \noindent\rule{\textwidth}{0.2pt}

\begin{lemma}

  Let $K$ be a field, $M$ and $N$ be flat $K$-modules. Suppose $p \subseteq M$ and $q \subseteq N$ are $K$-submodules, then $(p \otimes_{K} N) \sqcap (M \otimes_{K} q) = p \otimes_{K} q$ as $K$-submodules.
\end{lemma}

\begin{proof}
  The hard direction is to show $(p \otimes_{R} N) \sqcap (M \otimes_{R} q) \le p \otimes_{R} q$. Consider the following diagram:
  \begin{center}
    \begin{tikzcd}
      p \otimes_{K} q \ar[r, "u"] \ar[d, "\alpha"] & M \otimes_{K} q \ar[r, "v"] \ar[d, "\beta"] & {}^{M}/_{p} \otimes_{K} q \ar[d, "\gamma"] \\
      p \otimes_{K} N \ar[r, "u'"] & M \otimes_{K} N \ar[r, "v'"] & ^{M}/_{p} \otimes_{K} N
    \end{tikzcd}
  \end{center}
  Since $^{M}/_{p}$ is flat, $\gamma$ is injective.
  Let $z \in (p \otimes_{R} N) \sqcap (M \otimes_{R} q) = \im \beta \sqcap \im u'$. By abusing notation, replace $z$ with some elements of $M \otimes_{K} q$ and continue with $\beta(z)\in\im \beta \sqcap \im u'$. Since $v'(\beta(z))=\gamma(v(z))$ and that $\beta(z)\in \im u'$, we conclude that $\gamma(v(z))=0$, that is $z\in\ker v=\im u$. We abuse notation again, let $z \in p \otimes_{K} q$, we need to show $\beta(u(z))\in\im \beta \sqcap \im u'$, but $\beta\circ u=u'\circ\alpha$, we finish the proof.
\end{proof}

\subsection{Centralizer and Center}
Let $R$ be a commutative ring and $A$, $B$ be two $R$-algebras. We denote centralizer of $S\subseteq A$ by $C_{A}S$ and centre of $A$ by $Z(A)$.

\begin{lemma}
  Let $S, T$ be two subalgebras of $A$, then $C_{A}(S\sqcup T)=C_{A}(S)\sqcap C_{A}(T)$.
  \leanok
  \lean{Subalgebra.centralizer_sup}
\end{lemma}
This lemma can be generalized to centralizers of arbitrary supremum of subalgebras.

\begin{lemma}
  If we assume $B$ is free as $R$-module, then $C_{A\otimes_{R}B}\left(\im\left(A\to A\otimes_{R}B\right)\right)$ is $Z(A) \otimes_{R} B$
\end{lemma}
A symmetric statement goes without saying.
\begin{proof}
  Let $w\in C_{A\otimes_{R}B}\left(\im\left(A\to A\otimes_{R}B\right)\right)$. Since $B$ is free, we choose an arbitrary basis $\mathcal{B}$; by~\cref{lem:expand-tensor-in-basis}, we write $w = \sum_{i}m_{i}\otimes_{K}\mathcal{B}_{i}$. It is sufficient to show that $m_{i}\in Z(A)$ for all $i$. Let $a \in A$, we need to show that $m_{i}\cdot a = a \cdot m_{i}$. Since $w$ is in the centralizer, $w \cdot (a\otimes 1) = (a\otimes 1)\cdot w$. Hence we have $\sum_{i}(a\cdot m_{i})\otimes\mathcal{B}_{i}=\sum_{i}(m_{i}\cdot a)\otimes\mathcal{B}_{i}$. By the uniqueness of~\cref{lem:expand-tensor-in-basis}, we conclude $a\cdot m_{i}=m_{i}\cdot a$.
\end{proof}

\begin{remark}
  Since $\im\left(R\otimes_{R}B\to A\otimes_{R}B\right)=\im\left(A\to A\otimes_{R}B\right)$, we conclude its centralizer in $A\otimes_{R}B$ is $Z(A)\otimes_{R}B$ as well.
\end{remark}

\begin{lemma}
  \label{lem:center-tensor}
  Assume $R$ is a field. The centre of $A\otimes_{R} B$ is $Z\left(A\right)\otimes_{R}Z\left(B\right)$.
  \leanok
  \lean{IsCentralSimple.center_tensorProduct}
\end{lemma}
\begin{proof}
 From previous lemmas, we know that $Z\left(A\otimes_{R}B\right)$ is equal to $Z\left(A\right)\otimes_{R}B \sqcap A\otimes_{R}Z\left(B\right)$ which is $Z\left(A\right)\otimes_{R}Z\left(B\right)$
\end{proof}

%%% Local Variables:
%%% mode: LaTeX
%%% TeX-master: "../print"
%%% End:

\chapter{Wedderburn-Artin Theorem for Simple Rings}

\section{Classification of Simple Rings}

\begin{lemma}[minimal ideal of simple rings]\label{lemma:min-ideal-simple-ring}
  Let $A$ be a ring and $I$ a non-trivial minimal left ideal of $A$, then $I$ is a simple $A$-module.
  \leanok
  \lean{minimal_ideal_isSimpleModule}
\end{lemma}
\begin{proof}
Let $J\le I$ be an $A$-submodule of $I$, suppose $J$ is non-trivial, we prove that $J=I$. Then the image $J'$ of $J$ under $I \hookrightarrow A$ is a non-trivial left ideal of $A$. Since $I \hookrightarrow A$ is injective, it is sufficient to prove that $J' = I$. This is because $J'\le I$ and $J' \not< J$.
\end{proof}

\begin{lemma}
  Let $A$ be a simple ring and $I$ a non-trivial left ideal. One can write $1 \in A$ as $\sum_{i=0}^{n}x_{i}y_{i}$ for some $x_{i}\in I$ and $y_{i} \in A$.
\end{lemma}

\begin{proof}
  Let $I'$ be the two-sided ideal spanned by $I$. Then since $A$ is a simple ring, $I' = A$. Thus $1 \in I'$. One can write $1 \in A$ as $\sum_{i}a_{i}x_{i}b_{i}$ for some $x_{i}\in I$ and $a_{i},b_{i}\in A$, since $I$ is a left ideal $a_{i}x_{i}\in I$ as well.
\end{proof}

Now, we can find the smallest $n$ such that $1\in A$ can be written as $\sum_{i=0}^{n}x_{i}y_{i}$ for some $x_{i}\in I$ and $y_{i}\in A$. Let us fix the notations $n$, $x_{i}$ and $y_{i}$

\begin{lemma} The $n$, $x_{i}$ and $y_{i}$ are all non-zero.
  \leanok
  \lean{Wedderburn_Artin.aux.nxi_ne_zero}
\end{lemma}

\begin{proof}
  If $n$ is $0$, then $1 = 0$ in $A$, but all simple rings are non-trivial.
  We argue by contradiction to prove that all $x_{i}$ and $y_{i}$ are non-zero. Assume there exists a $j$ such that $y_{j}\ne0$ implies $x_{j} = 0$. Without loss of generality, we assume $j = 0$. Then $1 = \sum_{i=0}^{n}x_{i}y_{i}=\sum_{i=1}^{n}x_{i}y_{i}$. This contradicts the minimality of $n$.
\end{proof}

\begin{theorem}[Wedderburn]\label{thm:wed}
  Let $A$ be a simple ring and $I$ a non-trivial minimal left ideal. Then there exists a non-zero $n \in \mathbb{N}$ such that $A \cong I^{n}$ as $A$-modules.
  \leanok
  \lean{Wedderburn_Artin.aux.equivIdeal}
  \uses{lemma:min-ideal-simple-ring}
\end{theorem}

\begin{proof}
  We continue to write $1 = \sum_{i=0}^{n}x_{i}y_{i}$ in the shortest possible manner. Then we can define an $A$-linear map $g : I^{n}\to A$ by $(v_{i})\mapsto \sum v_{i}y_{i}$. Then $g$ is surjective: if $a \in A$, then $(ax_{i})$ is mapped to $a$ under $g$. $g$ is injective as well: support $g(v_{i})=0=\sum_{i}v_{i}y_{i}$ with $(v_{i})$ not all zero. Without loss of generality, we assume $v_{0} \ne 0$, then the ideal $\langle v_{0}\rangle$ is equal to $I$ (since $I$ is simple\cref{lemma:min-ideal-simple-ring}). Thus $x_{0}\in I = \langle v_{0}\rangle$; implying that $x_{0}=r\cdot v_{0}$ for some $r\in A$. Thus $1=1 - r\cdot 0 = \sum_{i=0}^{n}x_{i}y_{i}-\sum_{i=0}^{n}r\cdot v_{i}y_{i}$. In this way, we cancelled the term at $i=0$, contradicting the minimality of $n$. Hence $g$ is an isomorphism.
\end{proof}

\begin{theorem}[Wedderburn-Artin (Ideal)]
  Let $A$ be an Artinian simple ring. There exists a non-zero $n$ and an ideal $I \subseteq A$ such that $I$ is simple as an $A$-module and $A \cong I^{n}$ as $A$-module.
  \label{thm:wed-artin-ideal}
  \leanok
  \lean{Wedderburn_Artin_ideal_version}
  \uses{thm:wed}
\end{theorem}

\begin{proof}
  By~\cref{thm:wed}, we only need a minimal left ideal. Since $A$ is Artinian, such ideal exists.
\end{proof}

\begin{theorem}[Wedderburn-Artin (Algebra)]
  Let $K$ be a field and $B$ an finite dimensional simple algebra over $K$. There exists a non-zero $n\in\mathbb{N}$ and a division $K$-algebra $S$ such that $B \cong \Mat_{n}(S)$.
  \leanok
  \lean{Wedderburn_Artin_algebra_version}
\end{theorem}

\begin{proof}
  By~\cref{thm:wed-artin-ideal}, we can find a $n$ and a minimal left ideal $I$ such $A \cong I^{n}$ as $A$-modules. Note that ${\left(\End_{B}I\right)}^{\opp}$ is a division ring. Then since $B^{\opp}\cong\End_{B}B\cong{\End_{B}(I^{n})}\cong\Mat_{n}\left(\End_{B}I\right)$ as rings, we have $e : B\cong \Mat_{n}\left(\End_{B}I\right)^{\opp}$ as rings. We also have a $K$-algebra structure on ${\left(\End_{B} I\right)}^{\opp}$ given by $(a \cdot f)(x)=f(a\cdot x)$, and this algebra structure promotes the ring isomorphism $e$ to a $k$-algebra isomorphism.
\end{proof}

\section{Uniqueness of the classification}
In the previous section, we know that finite dimensional simple $K$-algebra $B$ over is in fact a matrix algebras of a division $K$-algebra $S$. In this section, we prove that the division algebra $S$ is essentially unique.

%%% Local Variables:
%%% mode: LaTeX
%%% End:

\chapter{Skolem-Noether Theorem}

Let $K$ be a field, $A$, $B$ be $K$-algebras where $A$ is finite $K$-dimensional. Let $M$ be a simple $A$-module.

\begin{construction}
  For any $K$-algebra homomorphism $f : B \to A$, we give $M$ a $B \otimes_{K}\End_{A}M$-module structure by defining $(b\otimes l)\cdot m$ to be $f(b)\cdot l(m)$. To emphasis $f$, we denote $M$ with the $B\otimes_{K}\End_{A}M$-module structure by $M^{f}$.
  \leanok
  \lean{IsMod}
\end{construction}

\begin{lemma}
  Let $f : B \to A$ be a $K$-algebra homomorphism, $M^{f}$ is finitely generated as a $B\otimes_{K}\End_{A}M$-module.
  \leanok
  \lean{module_inst_findim}
\end{lemma}
\begin{proof}
  Since $M$ is a finite $A$-module and $A$ a finite dimensional $K$-vector space, $M$ is a finite dimensional $K$-vector space as well. Suppose $S \subseteq M$ generates $M$ as $K$-module, the claim is that $S$ generates $M^{f}$ as well. Let $x \in M^{f}$, we write $x = \sum_{i}\lambda_{i}\cdot s_{i}$ with $\lambda_{i}\in K$ and $s_{i}\in S$.
  Note that $\lambda_{i}\cdot s_{i} = (\rho(\lambda_{i})\otimes\mathbf{1}_{M})$ in $M^{f}$ where $\rho : K \to B$ is the map giving $B$ its $K$-algebra structure. Hence $x$ is in the span of $S$ in $M^{f}$ as well.
\end{proof}

%%% Local Variables:
%%% mode: LaTeX
%%% TeX-master: "../print"
%%% End:

\section{Double Centralizer Theorem}\label{sec:double-centralizer}

In this section let $F$ be a field and $A$ an $F$-algebra. Define
$\McL_{A} \subseteq \End_{F}A$ to be
\[\left\{f : A \to A|f(x)=ax~\text{for some}~a\in A\right\},\]
i.e. $F$-linear maps defined by left multiplication; similarly define
$\McR_{A}$. Note that $\McL_{A}$ and $\McR_A$ are $F$-subalgebras of
$\End_{F}A$. When we need to stree the underlying field is $F$, we also write
$\McL_{A}^{F}$ and $\McR_{A}^{F}$. We assume $A$ to be a finite dimensional
central simple $F$-algebra.
\begin{lemma}
  \label{lem:centralizer-mul-left-le}
  The centralizer of $\McL_{A}$ in $\End_{F}A$ is smaller than or equal to
  $\McR_{A}$:
  \[
    C_{\End_{F}A}\left(\McL_{A}\right) \le \McR_{A}.
  \]
  \leanok \lean{centralizer_mulLeft_le_of_isCentralSimple}
\end{lemma}

\begin{proof}
  Indeed, let $x \in C_{\End_{F}A}\left(\McL_{A}\right)$. Recall
  from~\cref{con:self-tensor-opp-iso-end} that
  $e : A\otimes_{F}A^{\opp}\cong \End_{F}A$ as $F$-algebras. Then $e^{-1}(x)$ is
  in
  $C_{A\otimes_{F}A^{\opp}}\left(\im\left(A\to A\otimes_{F}A^{\opp}\right)\right)$
  (for $e$ sends $a\otimes 1$ to the $F$-linear map $(a\cdot\bullet)$). Since
  $C_{A\otimes_{F}A^{\opp}}\left(\im\left(A\to A\otimes_{F}A^{\opp}\right)\right) = Z(A)\otimes_{F}A^{\opp}=F\otimes_{F}A^{\opp}=\im\left(A^{\opp}\to A\otimes_{F}A^{\opp}\right)$,
  we find some $y\in A^{\opp}$ such that $1 \otimes y = e^{-1}(x)$. Therefore
  $e\left(1\otimes y\right) = x$; but $e\left(1\otimes y\right)$ is in
  $\McR_{A}$ for it is the linear map $(\bullet\cdot y)$.
\end{proof}

\begin{remark}\label{rem:mul-left-center-linear}
  For any $F$-algebra $A$, every element in $C_{\End_{F}A}\left(\McL_{A}\right)$
  is in fact $Z(A)$-linear. Let $x\in C_{\End_{F}A}\left(\McL_{A}\right)$,
  $z\in Z(A)$ and $a \in A$, we have $x(z\cdot a) = z\cdot x(a)$ because $x$
  commutes with the linear map $\left(z\cdot\bullet\right)$.
\end{remark}

\begin{remark}
  $A$ is a $Z(A)$-algebra whose algebra structure is given by
  $Z(A)\hookrightarrow A$. By~\cref{lem:center-simple-ring}, $Z(A)$ is a field.
  $A$ is finite dimensional as a $Z(A)$-module because of the tower $A/Z(A)/F$.
\end{remark}

 \begin{lemma}
   As $F$-algebras, we have $\McR_{A}\cong A^{\mop}$. \leanok
   \lean{Module.End.rightMulEquiv}
 \end{lemma}
 \begin{proof}
   We prove the map $A \to \McR_{A}$ is bijective. It is injective because if
   $\left(\bullet \cdot a\right) = \left(\bullet\cdot b\right)$, then
   $a = 1 \cdot a = 1 \cdot b = b$. The map is surjective by the definition of
   $\McR_{A}$.
 \end{proof}
 \begin{lemma}\label{lem:centralizer-mul-left-eq-mul-right}
   Let $B$ be any simple $F$-algebra ({\em not\/} necessarily central). The
   centralizer of $\McL_{B}$ in $\End_{F}A$ is equal to $\McR_{B}$. \leanok
   \lean{centralizer_mulLeft}
 \end{lemma}
 \begin{proof}
   It is straightforward to show
   $\McR_{B}^{F}\le C_{\End_{F}A}\left(\McL_{B}^{F}\right)$. So we only need to
   prove $C_{\End_{F}A}\left(\McL_{B}^{F}\right)\le \McR_{B}^{F}$.
   By~\cref{lem:centralizer-mul-left-le}, since $B$ is a central simple finite
   dimensional $Z(B)$-algebra, we have that
   \[
     C_{\End_{Z(B)}B}\left(\McL_{B}^{Z(B)}\right) \le \McR_{B}^{Z(B)}.
   \]
   Suppose $f\in\End_{F}B$ is in $C_{\End_{F}B}B$, by
   ~\cref{rem:mul-left-center-linear}, $f$ is $Z(B)$-linear as well. Then $f$ is
   in $\McR_{B}^{Z(B)}$; that is $f$ is equal to $\left(\bullet\cdot b\right)$
   for some $b \in B$ as $Z(B)$-linear maps. Then $f$ is also equal to
   $\left(\bullet\cdot b\right)$ as $F$-linear maps.
 \end{proof}

 \begin{construction}

 \end{construction}
 
%%% Local Variables:
%%% mode: LaTeX
%%% TeX-master: "../print"
%%% End:

\chapter{Brauer Group}\label{cha:brauer-group}

\section{Construction of Brauer Group}
Let $K$ be a field. We denote the class of finite dimensional central simple $K$-algebras as $\CSA_{K}$. When $K$ is clear, we drop the subscript.

\begin{remark}
  By~\cref{lem:tensor-central} and~\cref{thm:tensor-csa}, $\CSA$ is closed under tensor product, that is if $A, B\in\CSA$, we have $A\otimes_{K} B\in\CSA$ as well.
\end{remark}

\begin{definition}[Brauer Equivalence]
  For any two $A, B\in\CSA$, we say $A$ and $B$ are Brauer equivalent, when there exists $m, n \in \mathbb{N}_{\ge0}$ such that $\Mat_{m}(A)\cong \Mat_{n}B$ as $K$-algebras. We denote this relation as $A\sim_{\Br_{K}} B$, when $K$ is clear, we drop the subscript.
\end{definition}

\begin{remark}
  Isomorphic $K$-algebras are Brauer equivalent.
\end{remark}

\begin{lemma}
  $\sim_{\Br}$ is reflexive.
  \leanok
  \lean{IsBrauerEquivalent.refl}
\end{lemma}
\begin{proof}
  Indeed, $A \cong \Mat_{1}(A)$ as $K$-algerbas.
\end{proof}

\begin{lemma}
  $\sim_{\Br}$ is symmetric.
  \leanok
  \lean{IsBrauerEquivalent.symm}
\end{lemma}
\begin{proof}
  Indeed, just exchange $m$ and $n$.
\end{proof}

\begin{lemma}
  $\sim_{\Br}$ is transitive.
  \leanok
  \lean{IsBrauerEquivalent.trans}
\end{lemma}
\begin{proof}
  Let $A\sim_{\Br}B$ and $B\sim_{\Br}C$; that is for some $m,n,p, q\in\mathbb{N}_{\ge0}$ we have $\Mat_{n}(A)\cong\Mat_{m}(B)$ and $\Mat_{p}(B)\cong \Mat_{q}(C)$ as $K$-algebras. Hence, from~\cref{con:matrix-matrix}, we have the following:
  \[
    \begin{aligned}
      \Mat_{np}(A)&\cong \Mat_{p}\left(\Mat_{n}(A)\right)\cong\Mat_{p}\left(\Mat_{m}(B)\right)\\
                  &\cong\Mat_{mp}(B)\cong\Mat_{m}\left(\Mat_{p}(B)\right)\\
      &\cong\Mat_{m}\left(\Mat_{q}(C)\right)\cong\Mat_{mq}(C).
    \end{aligned}
  \]
  In another word, $A\sim_{\Br}C$.
\end{proof}

Hence $\sim_{\Br}$ is really an equivalence relation, we denote the quotient ${}^{\CSA}/_{\sim_{\Br}}$ as $\Br(K)$.

\begin{lemma}\label{lem:br-mul-wd}
  $(\bullet\otimes_{K}\bullet):\CSA\times\CSA \to \CSA$ descends to a function on $\Br(K)$.
  \leanok
  \lean{BrauerGroup.eqv_tensor_eqv}
\end{lemma}
\begin{proof}
  We need to prove that for all $A, B, C, D \in \CSA$ such that $A\sim_{\Br} B$ and $C\sim_{\Br}D$, $A\otimes_{R}C \sim_{\Br} B\otimes_{R}D$ as well.
  Suppose $\Mat_{m}(A)\cong \Mat_{n}(B)$ as $K$-algebras and $\Mat_{p}(C)\cong\Mat_{q}(D)$, by~\cref{con:matrix-tensor-matrix}, we have
  \[
    \begin{aligned}
      \Mat_{mp}\left(A\otimes_{R} C\right)&\cong \Mat_{m}(A)\otimes_{R}\Mat_{p}(C) \\
                                          &\cong \Mat_{n}(B)\otimes_{R}\Mat_{q}(D)\\
      &\cong \Mat_{nq}\left(B\otimes_{R}D\right).
    \end{aligned}
  \]

\end{proof}

\begin{construction}[Brauer Group]\label{con:br}
  $\Br(K)$ forms a group under $[A]_{\sim_{\Br}}\cdot[B]_{\sim_{\Br}}=[A\otimes_{K}B]_{\sim_{\Br}}$ with neutral element $[K]_{\sim_{\Br}}$ where $A, B\in\CSA$ and $[A]_{\sim_{Br}}^{-1}=[A^{\opp}]_{\sim_{\Br}}$. We need to prove the following properties:
  \begin{enumerate}
    \item associativity: for all $A, B, C\in\CSA$, $[A]_{\sim_{\Br}}\cdot\left([B]_{\sim_{\Br}}\cdot[C]_{\sim_{\Br}}\right)=\left([A]_{\sim_{\Br}}\cdot[B]_{\sim_{\Br}}\right)\cdot [C]_{\sim_{\Br}}$ because $A\ox_{R}\left(B\ox_{R}C\right)\cong \left(A\ox_{R}B\right)\ox_{R}C$ as $K$-algebras.
    \item neutral element: for all $A\in \CSA$, $[K]_{\sim_{\Br}}\cdot[A]_{\sim_{\Br}}=[A]_{\sim_{\Br}}=[A]_{\sim_{\Br}}\cdot [K]_{\sim_{\Br}}$. Since $[K]_{\sim_{\Br}}\cdot[A]_{\sim_{\Br}}=[K\ox_{K} A]_{\sim_{\Br}}$, in~\cref{con:matrix-tensor-matrix}, we see that $\Mat_{n}(A)\cong A\ox_{K}\Mat_{n}(K)$, by~\cref{lem:br-mul-wd}, $A\ox_{K}\Mat_{n}(K)$ is Brauer equivalent to $A\ox_{K} K$ since $K\sim_{\Br}\Mat_{n}(K)$.
    \item cancellation: for all $A \in \CSA$, we need $[A]_{\sim_{\Br}}\cdot[A^{\opp}]_{\sim_{\Br}}$, that is we want $A\ox_{K}A^{\opp}\sim_{\Br} K$. By~\cref{con:self-tensor-opp-iso-end}, we have $A\ox_{K}A^{\opp}\cong \End_{K}A$ which is isomorphic to $\Mat_{\dim_{K}A}(K)$ as $K$-algebras.
  \end{enumerate}
  \leanok
  \lean{BrauerGroup.Bruaer_Group}
\end{construction}

\begin{theorem}
  \label{thm:br-triv-alg-closed}
  If $K$ is algebraically closed, $\Br(K)$ is trivial; in particular $\br_{n}(\mathbb{C})$ is trivial.
  \leanok
  \lean{BrauerGroup.Alg_closed_eq_one}
\end{theorem}
\begin{proof}
  We need to show that every $A\in\CSA$ is isomorphic to $\Mat_{n}(K)$ for some $K$ when $K$ is algebraically closed. Indeed, by~\cref{thm:wed-artin-algebra}, $A \cong \Mat_{n}(D)$ for some division algebra $D$ and $n\in\nat_{\ge0}$. Since $K$ is algebraically closed and $D$ is an integral domain and finite dimensional, the structure morphism $\rho: K\to D$ is a isomorphism; therefore $A\cong \Mat_{n}(K)$.
\end{proof}

\begin{lemma}
  \label{lem:common-div-alg}
  \leanok
  \lean{IsBrauerEquivalent.exists_common_division_algebra}
  Let $A, B \in \csa_{K}$. There exists a division $K$-algebra $D$ and non-zero $m,n\in\mathbb{N}$ such that $A\cong\mat_{m}(D)$ and $B\cong\mat_{n}(D)$ as $K$-algebras.
\end{lemma}

\begin{proof}
  By~\cref{thm:wed-artin-algebra}, we can find division algebras $S_{A}, S_{B}$ and non-zero $m, n\in\mathbb{N}$ such that $A\cong\mat_{n}\left(S_{A}\right)$ and $B\cong\mat_{m}\left(S_{B}\right)$ as $K$-algebras. Hence $B\sim_{\br}A\sim_{\br}\mat_{n}\left(S_{A}\right)\sim_{\br}S_{A}$, in another word, for some non-zero $a, a'\in\mathbb{N}$, we have $\mat_{a}(B)\cong\mat_{a'}\left(S_{A}\right)$ as $K$-algebras. Hence, by~\cref{thm:wed-artin-uniq}, we have that $S_{A}\cong S_{B}$ as $K$-algebras and the lemma is proved.
\end{proof}

\section{Base Change}
In this section let $E/K$ be a field extension. We have seen in~\cref{cor:csa-basechange} that if $A\in\CSA_{K}$ then $E\otimes_{K}A\in\CSA_{E}$; therefore we have a set-theoretic function $\CSA_{K}\to\CSA_{E}$. In this section we prove that this descends to a group homomorphism $\Br(K)\to\Br(E)$. For brevity, if $A\in \CSA_{K}$, we dentoe $E\ox_{K}A$ as $A_{E}$ when this causes no confusion.
\begin{construction}
  \label{con:br-base-change}
  We will construct a series of isomorphisms (either over $K$ or $E$) to arrive at the conclusion that $A\sim_{\Br_{K}}B$ implies $A_{E}\sim_{\Br_{E}}B_{E}$. Assume $m,n\in\nat_{\ge0}$ are such that $\Mat_{m}(A)\cong\Mat_{n}(B)$ are $K$-algebras. Then we do the following calculation: as $E$-algebras
  \[
    \begin{aligned}
      \Mat_{m}\left(A_{E}\right)
      &\cong A_{E}\otimes_{E}\Mat_{m}(E) &\text{see~\cref{con:matrix-tensor-matrix}}\\
      &\cong A_{E}\otimes_{E}\left(E\ox_{K} \Mat_{m}(K)\right) & \text{see}~\dagger\\
      &\cong E \ox_{K}\left(A\ox_{K}\Mat_{m}(K)\right) &\text{see}~\ddagger\\
      &\cong E\ox_{K}\Mat_{m}(A) &\text{see~\cref{con:matrix-tensor-matrix} and}~\dagger\!\!\dagger \\
      \Mat_{n}\left(B_{E}\right) &\cong E\ox_{K}\Mat_{n}(B) &\text{same as the case of}~A\\
      Mat_{m}\left(A_{E}\right)&\cong \Mat_{n}\left(B_{E}\right) &\text{see}~\dagger\!\!\dagger
    \end{aligned}.
  \]

   \noindent$\dagger$: Wee need to check $\Mat_{m}(E)\cong E\otimes_{K}\Mat_{m}K$ as $E$-algebras since~\cref{con:matrix-tensor-matrix} only gives a $K$-algebra isomorphism. If $e \in E$, then its image in $E\ox_{K}\Mat_{m}(K)$ is $e\otimes 1$ and its image in $\Mat_{m}(E)$ is $\diag(e)$ which under the $K$-algebra isomorphism is mapped to $\sum_{ij}\diag(e)_{ij}\cdot \delta_{ij}=e\ox1$.

   \noindent$\ddagger$: This is defined by combining two $E$-algebra homomorphisms
   \[
     A_{E}\to A_{E}\ox_{K}\Mat_{m}(K)\to E\ox_{K}\left(A\ox_{K}\Mat_{m}(K)\right)\]
     and
     \[
       E\ox_{K}\Mat_{m}(K)\to \left(E\ox_{K}\Mat_{m}(K)\right)\ox_{K}A \to E\ox_{K}(A\ox_{K}\Mat_{m}(K)).
    \]
   Since $\left(E\ox_{K}A\right)\ox_{E}\left(E\ox_{K}\Mat_{m}(K)\right)$ is a simple ring, this morphism is automatically injective. It is surjective as well: let $x\in E\ox_{K}\left(A\ox_{K}\Mat_{m}(K)\right)$, without loss of generality, assume $x=e\ox (a\ox\delta_{ij})$ for some $e\in E$, $a\in A$. Then precisely $\left(e\ox a\right)\ox\left(1\ox\delta_{ij}\right)$ is mapped to $x$.

   \noindent$\dagger\!\dagger$: a $K$-algebra isomorphism $A\cong B$ gives an $E$-algebra isomorphism $E\ox_{K}A \cong E\ox_{K} B$.

   Thus we have a well defined function $\Br(K)\to\Br(E)$. We now check that this is a group homomorphism. $[K]_{\sim_{\Br_{K}}}$ is mapped to $[E\ox_{K}K]_{\sim_{\Br_{E}}}$ but $E\ox_{K}K\cong E$ as $E$-algebra. For $A, B\in\CSA_{K}$, we have that $[AB]_{\sim_{\Br_{K}}}$ is mapped to $\left(A\ox_{K}B\right)_{E}\cong A_{E}\ox_{E} B_{E}$ as $E$-algebras; hence $[AB]_{\sim_{\Br_{K}}}$ and $[A]_{\sim_{\Br_{K}}}\cdot[B]_{\sim_{\Br_{K}}}$ have the same image under base change.
   \leanok
   \lean{BrauerGroupHom.BaseChange}
 \end{construction}
 Denote the base change morphism in~\cref{con:br-base-change} as $\Br_{K}^{E}$.
 \begin{lemma}
   $\Br_{K}^{K}$ is identity.
   \leanok
 \end{lemma}
 \begin{proof}
   If $A\in\CSA$, then $A\sim_{\Br}K\ox_{K}A$.
 \end{proof}

 \begin{lemma}\label{lem:br-base-change-self}
   Consider the tower of field extension $E/F/K$,
   \[
     \Br_{K}^{E} = \Br_{E}^{F}\circ\Br_{K}^{E}.
   \]
 \end{lemma}

 \begin{proof}
   If $A\in\CSA_{K}$, then $E\ox_{F}\left(F\ox_{K}A\right)$ is isomorphic to $E \ox_{F} A$ as $E$-algebras.
   \leanok
   \lean{BrauerGroupHom.baseChange_idem}
 \end{proof}

 \begin{corollary}\label{lem:br-base-change-tower}
   $\Br$ forms a functor from category of field to category of abelian groups.
   \leanok
   \lean{BrauerGroupHom.Br}
 \end{corollary}
 \begin{proof}
   This is the categorical version of~\cref{lem:br-base-change-self} and~\cref{lem:br-base-change-tower}.
 \end{proof}

 \begin{definition}[Relative Brauer Group]
   Let $E/K$ be a field extension, we define the relative Brauer group $\Br(E/K)$ to be the kernel of the base change morphism $\Br_{K}^{E}$.
 \end{definition}

 \begin{remark}
   Unpacking the definition of the relative Brauer group, we see that for any $A \in \CSA_{K}$, if $E\ox_{K}A\cong\mat_{n}(E)$ as $E$-algebras, then $\br^{E}_{K}\left([A]_{\sim_{\br}}\right)=1$.
 \end{remark}

 \begin{definition}[Splitting Field]
   For any field extension $E/K$ and any $K$-algebra $A$, we say $E$ is a splitting field of $A$ if and only if $E\ox_{K}A \cong \mat_{n}(E)$ as $E$-algebras for some non-zero $n$. We also say $E$ splits $A$ or $A$ is splited by $E$
   \leanok
   \lean{isSplit}
 \end{definition}

  \begin{theorem}
   \label{thm:split-iff-mem-relative}
   Let $E/K$ be a field extension and $A\in\csa_{K}$, $E$ splits $A$ if and only if $[A]_{\sim_{\br}}\in\br(E/K)$.
   \leanok
   \lean{BrauerGroup.split_iff}
 \end{theorem}

 \begin{proof}
   The ``only if'' part is by definition. For the other direction, we know by definition that $\mat_{n}(E\ox_{K}A)\cong\mat_{m}(E)$ as $E$-algebras for some non-zero $m, n$. By~\cref{thm:wed-artin-algebra}, we find some division algebra $D$ and non-zero natural number $p$ such that $E\ox_{K}A\cong \mat_{p}(D)$ as $E$-algebras. Thus $\mat_{pm}(E)\cong\mat_{pn}\left(E\ox_{K}A\right)\cong\mat_{p^{2}n}(D)$ as $E$-algebras. By~\cref{thm:wed-artin-uniq}, we conclude that $E\cong D$ as $E$-algebras. Hence $E\ox_{K}A\cong \mat_{p}(E)$, in another word, $E$ splits $A$.
 \end{proof}

 \begin{remark}
   In light of~\cref{lem:br-base-change-tower}, if $K$ is algebraic closed then $K$ splits any $K$-algebra $A$. Indeed, $K$ splits $A$ if and only if $[A]_{\sim\br}$ but $[A]_{\sim\br}$ is equal to $1$.
 \end{remark}

 \begin{remark}
   If two $\csa_{K}$ are Brauer equivalent, in another word, $A \sim_{\br_{K}} B$, then $E$ splits $A$ if and only if $E$ splits $B$. Indeed, if $A$ and $B$ are equivalent, then $[A]_{\sim\br} \in \br(E/K)$ if and only if $[B]_{\sim\br}\in\br(E/K)$.
 \end{remark}


 \section{Good Representative Lemma}
 In this section, let $K/F$ be a finite dimensional field extension.

 \begin{lemma}
   \label{lem:good-rep-inv}
   Let $A\in\csa_{F}$ splitted by $K$. There exists a $B\in\csa_{F}$ such that
   \begin{itemize}
     \item $[A]_{\sim_{\br}}[B]_{\sim_{\br}}=1$
     \item there exists $F$-algebra map $K\hookrightarrow B$
     \item ${\left(\dim_{F}K\right)}^{2}=\dim_{F}B$.
   \end{itemize}
   \leanok
   \lean{exists_embedding_of_isSplit}
 \end{lemma}

 \begin{proof}
   Since $K$ splits $A$, we find a non-zero natural number $n$ such that $K\ox_{F}A\cong\mat_{n}K\cong\End_{K}\left(K^{n}\right)$ as $K$-algebras.
   We define an $F$-algebra map $\iota: A\to\End_{F}\left(K^{n}\right)$ by
   \begin{center}
     \begin{tikzcd}
       A \arrow{r} & K \ox_{F} A \arrow{r}{\cong} & \End_K\left(K^n\right) \arrow{r}{|_{F}} & \End_F\left(K^n\right)
     \end{tikzcd},
   \end{center}
   where $|_{F}$ is restriction of scalars. Since $A$ is simple, $\iota$ is injective, therefore $A \cong \iota(A)$ as $F$-algebras. Define $B$ as $C_{\End_{F}\left(K^{n}\right)}(\iota(A))$, the centralizer of the range of $\iota$ in $\End_{F}\left(K^{n}\right)$.
   We construct an embedding $K \hookrightarrow B$ by $r \mapsto (r \cdot \bullet)$

   $B$ is a central $F$-algebra: if $x \in Z(B)$, then $x \in \iota(A)$ because by~\cref{thm:double-centralizer}, it is sufficient to prove that $x$ is in $C_{\End_{F}\left(K^{n}\right)}\left(B\right)$ which follows from the fact that $x \in Z(B)$. In fact, $x \in Z(\iota(A))$: suppose $a \in A$, we need to check $x \cdot \iota(a) = \iota(a)\cdot x$, this is the case because $B$ is defined as the centralizer of $\iota(A)$. Since $\iota(A) \cong A$ as $F$-algebras, $\iota(A)$ is $F$-central, hence $x \in F$.

   $B$ is a simple ring: by~\cref{lem:simple-centralizer}, it is sufficient to prove that $\iota(A)$ is a simple ring which comes from $A \cong \iota(A)$ as $F$-algebras.

   By~\cref{cor:self-tensor-centralizer}, we have $F$-algebra isomorphism $\End_{F}\left(K^{n}\right)\cong \iota(A)\ox_{F}B \cong A \ox_{F}B$. Since $\End_{F}\left(K^{n}\right)\cong\mat_{\dim_{F}\left(K^{n}\right)}\left(F\right)$ as $F$-algebras, we see that $[A]_{\sim_{\br}}$ and $[B]_{\sim_{\br}}$ are inverses.

   By~\cref{lem:dim-centralizer}, $\dim_{F}B\cdot \dim_{F}\iota(A) = \dim_{F}B\cdot\dim_{F}A = \dim_{F}\End_{F}\left(K^{n}\right) = {\left(\dim_{F}\left(K^{n}\right)\right)}^{2}={\left(\dim_{F}K\cdot \dim_{K}\left(K^{n}\right)\right)}^{2}=n^{2}\cdot{\left(\dim_{F}K\right)}^{2}$. On the other hand, since $K\ox_{F}A\cong\mat_{n}K$, we have $\dim_{F}K\ox_{F}A=\dim_{F}K\cdot\dim_{F}A=\dim_{F}\mat_{n}K=\dim_{F}K\dim_{K}\mat_{n}K=n^{2}\dim_{F}K$. Since $\dim_{F}K\ne 0$ , we conclude $\dim_{F}A=n^{2}$. Since $n\ne 0$ and $\dim_{F}(B)\cdot \dim_{F}(A)=n^{2}\dim_{F}(B)=n^{2}{\left(\dim_{F}K\right)}^{2}$, we get the desired result.

 \end{proof}

 \begin{corollary}\label{cor:good-rep}
   Let $A\in\csa_{F}$ splitted by $K$. There exists a $B \in \csa_{F}$ such that
   \begin{itemize}
     \item $[B]_{\sim_{\br}}=[A]_{\sim_{\br}}$
     \item there exists an $F$-algebra map $K\hookrightarrow B$
     \item  ${\left(\dim_{F}K\right)}^{2}=\dim_{F}B$.
  \end{itemize}
 \end{corollary}

 \begin{proof}
   Let $B$ and $\iota : K \hookrightarrow B$ be as in~\cref{lem:good-rep-inv}. Consider $B^{\opp}$ and $K \hookrightarrow B \to B^{\opp}$. This works.
 \end{proof}

 \begin{theorem}\label{thm:good-rep-iff-split}
   Let $A\in\csa_{F}$. $K$ splits $A$ if and only if there exists a $B\in\csa_{F}$ such that
   \begin{itemize}
     \item $[B]_{\sim_{\br}}=[A]_{\sim_{\br}}$
     \item there exists an $F$-algebra map $K \hookrightarrow B$
     \item ${\left(\dim_{F}K\right)}^{2}=\dim_{F}B$.
   \end{itemize}
   \leanok
   \lean{isSplit_iff_dimension}
 \end{theorem}
 \begin{proof}
   The ``if'' direction is~\cref{cor:good-rep}. For the ``only if'' direction, let $B \in\csa_{F}$ and $\iota : K \hookrightarrow B$ be given. We give $B$ a $K$-module structure by right multiplication, that is for any $a \in K$ and $b \in B$, we define $a \cdot b := b \cdot \iota(a)$.
   Since $B$ is a finite dimensional $F$-vector space and $K/F$ is a finite dimensional field extension, $B$ is a finite dimensional $K$-vector space as well. Since $[B]_{\sim_{\br}}=[A]_{\sim_{\br}}$, it is sufficient to show that $K$ splits $B$.
   We define an $F$-bilinear map $\mu : K \to B \to \End_{K}B$ by $(c, a) \mapsto (c \cdot a \cdot \bullet)$ which induce an $F$-linear map $\mu' : K \ox_{F} B \to \End_{K}B$. Since for any $r, c\in K$ and $a \in B$, we have $\mu'\left(r \cdot c\ox a\right)(a') = a a' \iota(rc)=aa'\iota(c)\iota(r)=r\cdot \mu'(c\ox a)$, that is $\mu'$ is $K$-linear as well. Note that \[\mu'(1)=\mu'(1\ox 1) = (1\cdot 1\cdot \bullet) = 1\] and that
   \[
     \begin{aligned}
       \mu'(c\ox a \cdot c'\ox a')(a'') &= \mu'(cc'\ox aa')(a'') \\
                                        &= cc' \cdot aa' \cdot a'' \\
                                        &= aa' a'' \iota(c c') \\
                                        &= a(a' a'' \iota(c'))\iota(c) \\
                                        &=\mu'(c \ox a)(a' a'' \iota(c'))\\
                                        &=\mu'(c\ox a)\left(\mu'(c'\ox a')(a'')\right) \\
                                        &= \left(\mu'(c\ox a)\circ\mu'(c'\ox a')\right)(a'')
     \end{aligned},
     \]
     that is, $\mu'$ is an $K$-algebra map.

   If we can show that $\mu'$ is a bijection, we will prove the result for $K\ox_{F} B \cong \End_{K} B \cong \mat_{\dim_{K} B}K$ as $K$-algebras. By~\cref{cor:alghom-bijective-of-dim-eq}, it is sufficient to show $\dim_{K}K\ox_{F}B = \dim_{K}\End_{K}B$. Let $n$ denote $\dim_{F}K$. Since, $\dim_{F}K\dim_{K}K\ox_{F}B =\dim_{F}K\ox_{F}B=\dim_{F}K \dim_{F}B$. we have $\dim_{K}K\ox_{F}B = \dim_{F}B = {\left(\dim_{F}K\right)}^{2}$. On the other hand, since ${\left(\dim_{F}K\right)}^{2}=\dim_{F} B = \dim_{F}K\dim_{K}B$, we have $\dim_{K} B = \dim_{F}K$; thus $\dim_{K}\End_{K} B = {\left(\dim_{K}B\right)}^{2}={\left(\dim_{F}K\right)}^{2}$ and the result is proved.
 \end{proof}

 In light of~\cref{thm:good-rep-iff-split}, we isolate the following useful definition:
 \begin{definition}[Good Representation]\label{def:good-rep}
   For any $X \in \br(F)$, a good representation of $X$ is an $A \in \csa_{F}$ and an $F$-algebra map $K \hookrightarrow A$ which we often denote as $\iota$ or $\iota_{A}$ such that $[A]_{\sim_{\br}} = X$ and $\dim_{F}A={\left(\dim_{F}K\right)}^{2}$.
   \leanok
   \lean{GoodRep}
 \end{definition}

 \begin{corollary}
   \label{cor:mem-relative-br-iff-good-rep}
   For any $X\in\br(F)$, $X\in\br(K/F)$ if and only if $X$ admits a good representation.
   \leanok
   \lean{mem_relativeBrGroup_iff_exists_goodRep}
 \end{corollary}
 \begin{proof}
   Rephrase of~\cref{thm:good-rep-iff-split} and~\cref{thm:split-iff-mem-relative}.
 \end{proof}

We observe the following easy result about good representations. Let $X \in \br(F)$ and $A$ be a good representation of $X$.

\begin{lemma}
  The range $\iota_{A}(A)$ is a simple ring.
  \leanok
  \lean{GoodRep.ιRange}
\end{lemma}

\begin{proof}
  Because $K$ is a simple ring, $\iota_{A}$ is injective therefore $\iota_{A}(A)\cong K$.
\end{proof}

\begin{lemma}\label{lem:centralizer-range-good-rep}
  $C_{A}\left(\iota_{A}(A)\right) = \iota_{A}(A)$.
  \leanok
  \lean{GoodRep.centralizerιRange}
\end{lemma}
\begin{proof}
  In the language of~\cref{sec:subfield}, $\iota_{A}(A)$ is a subfield of $A$, hence by~\cref{lem:tfae-subfield}, we only need to show $\dim_{F}A = {\left(\dim_{F}\iota_{A}(A)\right)}^{2}$. But $\dim_{F}A={\left(\dim_{F}K\right)}^{2}$ and $\iota(A)\cong K$.
\end{proof}

\begin{construction}
  We give $A$ a $K$-module structure by {\em left} multiplication, that is for any $c \in K$ and $a \in A$, we define $c\cdot a$ to be $\iota_{A}(c)a$. Then $A$ is a finite dimensional $K$-vector space and $\dim_{K}A=\dim_{F}K$: indeed $\dim_{F}K\cdot\dim_{K}A = \dim_{F}K\cdot \dim_{F}K = \dim_{F}A$.
  \leanok
  \lean{GoodRep.dim_eq'}
\end{construction}

\section{The Second Galois Cohomology}

In this section, we construct a group isomorphism between $\br(K/F) \cong \HH^{2}\left(\gal(K/F), K^{\star}\right)$ where $K/F$ is a finite dimensional Galois extension. To keep alignment of the Brauer group, let us use the multiplicative notation for group cohomology. Recall:

\begin{definition}[the Second Group Cohomology]
  Let $G$ be a group and $M$ an abelian group (written multiplicatively) with a $G$-action.

  A function $f : G \times G \to M$ is a {\em 2-cocycle\/} if for all $g,h,j\in G$,
  \[
    f(gh, j)f(g, h) = \left(g\cdot f(h, j)\right) f(g, hj).
  \]

  A function $f : G\times G \to M$ is a {\em 2-coboundary\/} if there exists an $x : G \to M$ such that for all $g, h \in G$
  \[
    \frac{g\cdot x(h)}{x(gh)} x(g) = f(g, h).
  \]
  The second group cohomology $\hh^{2}\left(G, M\right)$ is defined to be the quotient group of 2-cocycles modulo 2-coboundaries.
  \leanok
  \lean{groupCohomology.H2, groupCohomology.IsMulTwoCoboundary, groupCohomology.IsMulTwoCocycle}
\end{definition}

In the following sections of this chapter, we assume that $X \in\br(F)$ and $A$ is a good representation of $X$. We use $\rho, \sigma, \tau$ to denote elements of $\gal(K/F)$. To improve typographic aesthetics of our proofs, we sometimes use subscript to mean function application.


\subsection{From $\br(K/F)$ to $\hh^{2}\left(\gal(K/F),K^{\star}\right)$}

\begin{definition}[Conjugation Factor]
  \label{def:conj-factor}
  With respect to $A$, a conjugation factor of $\sigma$ is a unit $x_{\sigma} \in A^{\star}$ such that for all $c \in K$,
  \[
    x_{\sigma} \iota_{A}(c)x_{\sigma}^{-1} = \iota_{A}(\sigma(c)).
  \]

  A conjugation sequence is a sequence $x:\gal(K/F) \to A^{\star}$ such that for all $\sigma\in\gal(K/F)$, $x_{\sigma}$ is a conjugation factor of $\sigma$. When we want to stress $A$, we say $A$-conjugation factor and $A$-conjugation sequence.
  \leanok
  \lean{GoodRep.conjFactor}
\end{definition}

\begin{remark}\label{rem:conj-factor-alternative-eq}
  When $x_{\sigma}$ is a conjugation factor of $\sigma$, the equalities $x_{\sigma}\iota_{A}(c) = x_{\sigma}\iota_{A}(\sigma(c))$ and $\iota_{A}(c)x_{\sigma}^{-1}=x_{\sigma}^{-1}\iota_{A}(\sigma(c))$ are also useful.
\end{remark}

\begin{construction}
  $A$ has a conjugation sequence:
  let $\sigma \in \gal(K/F)$, we have two $F$-algebra homomorphisms $K \to A$ given by $\iota_{A}$ and $\iota_{A}\circ \sigma$. Applying~\cref{thm:skolem-noether} to $\iota_{A}$ and $\iota_{A}\circ \sigma$ gives us the desired conjugation factor.
  \leanok
  \lean{GoodRep.aribitaryConjFactor}
\end{construction}

\begin{construction}
  If $x$ is a conjugation factor of $\sigma$ and $y$ of $\tau$, then $xy$ is a conjugation factor of $\sigma\tau$. For any $c \in K$
  \[
    \iota_{A}(\sigma(\tau(c))) = x\iota_{A}(\tau(c))x^{-1} = xy\iota_{A}(c)y^{-1}x^{-1}=\left(xy\right)\iota_{A}{\left(xy\right)}^{-1}.
  \]
  \leanok
  \lean{GoodRep.mul'}
\end{construction}

\begin{lemma}[Twisting Conjugation Factors]
  If $x$ and $y$ are two conjugation factors of $\sigma$, then there exists a unique $c \in K$ such that $x = y\iota_{A}(c)$.
  \leanok
  \lean{GoodRep.conjFactor_rel, GoodRep.conjFactorTwistCoeff}
\end{lemma}

\begin{proof}
  The uniqueness is clear: suppose $x = y\iota_{A}(c)=y\iota_{A}(c')$, then $c = c'$ because $x, y$ are units and $\iota_{A}$ is injective. We first observe that $y^{-1}x\in C_{A}(\iota(A))$: for any $z \in K$, $y^{-1}x\iota_{A}(z)=y^{-1}\iota_{A}(\sigma(z))x=\iota_{A}(z)y^{-1}x$ (by~\cref{rem:conj-factor-alternative-eq}). By~\cref{lem:centralizer-range-good-rep}, $y^{-1}x\in \iota(A)$, that is for some $z \in K$, we have that $y^{-1}x = \iota_{A}(z)$ and the claim is proved.
\end{proof}

We denote such $c$ by $\twist^{\sigma}(x, y)$ or $\twist^{\sigma}_{x, y}$, when $\sigma$ is clear from context, we often omit the superscript. With this notation, $x = y\iota_{A}(\twist_{x,y})$.

\begin{remark}
  $\twist(x, x)$ is equal to $1$ by uniqueness.
\end{remark}

\begin{remark}
  In fact, $\twist(x, y)$ is in $K^{\star}$ and $\twist(x, y)^{-1}=\twist(y, x)$.
\end{remark}

\begin{lemma}
  If $x$ and $y$ are conjugation factors for $\sigma$, $x=\iota_{A}(\sigma(\twist_{x, y}))y$.
\end{lemma}


%%% Local Variables:
%%% mode: LaTeX
%%% TeX-master: "../print"
%%% End:

\subsection{$\HH^{2} \circ \cross$ and $\cross \circ \HH^{2}$}

For a finite dimensional Galois extension of field $K/F$, we have constructed two functions $\HH^{2}$ and $\cross$ between the second cohomology group $\HH^{2}\left(\gal(K/F), K^{\star}\right)$ and the relative Brauer group $\br(K/F)$. In this section, we prove that they are mutual inverse to one another,

\begin{lemma}\label{lem:relative-br-snd-inverse-1}
  The composition of $\cross$ and $\HH^{2}$ is the identity:
  \begin{center}
    \begin{tikzcd}
      \displaystyle \HH^2\left(\gal(K/F), K^\star\right) \arrow{r}{\cross} \arrow[bend right = 15, swap]{rr}{\id} &
      \br(K/F) \arrow{r}{\HH^2} & \displaystyle \HH^2\left(\gal(K/F), K^\star\right).
    \end{tikzcd}
  \end{center}
  \leanok
  \lean{RelativeBrGroup.toSnd_fromSnd}
\end{lemma}
\begin{proof}
  Let $\mfa$ be any 2-cocycle, by~\cref{lem:cross-product-basis-conj}, we notice that $x : \sigma \mapsto \Delta_{\sigma, 1}$ is a conjugation sequence for $\cross_{\mfa}$. Hence by~\cref{con:good-rep-to-2-cocycles}, ~\cref{cor:to-2-cocylce-wd} and~\cref{thm:snd-coh-to-relative-br}, we evaluate the composition at $\mfa$ as:
  \begin{center}
    \begin{tikzcd}
      {[\mfa]} \arrow[mapsto]{r} & {\left[\cross_\mfa\right]_{\sim_{\br}}} \arrow[mapsto]{r}& {\left[(\sigma,\tau)\mapsto\comp^{x}_{\Delta_{\sigma, 1}, \Delta_{\tau, 1}, \Delta_{\sigma\tau, 1}}\right]}
    \end{tikzcd}.
  \end{center}
  That is, we need to show that $\mfa$ and $(\sigma, \tau) \mapsto \comp_{\Delta_{\sigma, 1}, \Delta_{\tau, 1}, \Delta_{\sigma\tau, 1}}$ are 2-cohomologous. In fact, they are equal.
  By\cref{con:compare-conj-factors}~, we have that $\iota_{\cross_{\mfa}}\left(\comp^{x}_{\Delta_{\sigma, 1}, \Delta_{\tau, 1}, \Delta_{\sigma\tau, 1}}\right) = \Delta_{\sigma, 1}\Delta_{\tau, 1}\Delta_{\sigma\tau, 1}^{-1} = \mfa(\sigma,\tau)\cdot \Delta_{\sigma\tau, 1}\Delta_{\sigma\tau, 1}^{-1} = \mfa(\sigma,\tau)\cdot 1 = \Delta_{\id,\mfa(\sigma,\tau)}$ which is precisely $\iota_{\cross_{\mfa}}(\mfa(\sigma,\tau))$.
\end{proof}

\begin{lemma}\label{lem:relative-br-snd-inverse-2}
  The composition of $\HH^{2}$ and $\cross$ is the identity:
  \begin{center}
    \begin{tikzcd}
      \br(K/F) \arrow{r}{\HH^2} \arrow[bend right = 15, swap]{rr}{\id} &
      \HH^2\left(\gal(K/F), K^\star\right) \arrow{r}{\cross} &
      \br(K/F)
    \end{tikzcd}.
  \end{center}
  \leanok
  \lean{RelativeBrGroup.fromSnd_toSnd}
\end{lemma}
\begin{proof}
  Let $X \in \br(K/F)$, $A$ be an arbitrary good representation of $X$ and $x$ be an arbitrary $A$-conjugation sequence which exists by~\cref{cor:mem-relative-br-iff-good-rep} and~\cref{con:exists-conj-seq}. By~\cref{def:good-rep}, $X = \left[A\right]_{\sim_{\br}}$.
  Hence by~\cref{cor:to-2-cocylce-wd} and~\cref{thm:snd-coh-to-relative-br}, we evaluate the composition at $X$ as:
  \begin{center}
    \begin{tikzcd}
      {[A]_{\sim_{\br}}} \arrow[mapsto]{r} &
      {\left[\mathcal{B}^2_x\right]} \arrow[mapsto]{r} &
      {\left[\cross_{\mathcal{B}^2_x}\right]}
    \end{tikzcd}.
  \end{center}
  Hence we need to prove that $A$ and $\cross_{\mathcal{B}^{2}_{x}}$ are Brauer equivalent. We will show that they are isomorphic as $F$-algebras. since $\{x_{\sigma}|\sigma\in\gal(K/F)\}$ is a $K$-basis for $A$ and $\{\Delta_{\sigma, 1}|\sigma \in \gal(K/F)\}$ is a $K$-basis for $\cross_{\mathcal{B}^{2}_{x}}$, they are certainly isomorphic as $K$-modules. Let $\phi : \cross_{\mathcal{B}^{2}_{x}} \cong A$ be the $K$-linear isomorphism defined by $\Delta_{\sigma, 1} \mapsto x_{\sigma}$, since the $K$-action on $A$ and the $F$-action on $A$ are compatible (\cref{con:good-rep-mod}), $\phi$ is also an $F$-linear isomorphism. Like in~\cref{thm:snd-coh-to-relative-br}, we check that $\phi(1) = 1$ and $\phi(xy)=\phi(x)\phi(y)$ for all $x,y \in A$:
  \begin{enumerate}
    \item preservation of one: by~\cref{con:compare-conj-factors}, we have
          \[
          \begin{aligned}
            \phi(1)
            &= \phi\left(\Delta_{\id,\mathcal{B}^{2}_{x}(\id,\id)^{-1}}\right) \\
            &= \mathcal{B}^{2}_{x}(\id,\id)^{-1} \phi\left(\Delta_{\id, 1}\right) \\
            &= \mathcal{B}^{2}_{x}(\id,\id)^{-1} x_{\id} \\
            &= \comp_{x_{\id},x_{\id},x_{\id}}^{-1} x_{\id} \\
            &= \comp_{x_{\id},x_{\id},x_{\id}} x_{\id} x_{\id} x_{\id}^{-1} \\
            &= x_{\id} x_{\id}^{-1} \\
            &= 1.
          \end{aligned}
          \]
    \item preservation of multiplication: let $\sigma,\tau\in\gal(K/F)$ and $c,d \in K$, by~\cref{con:compare-conj-factors} and~\cref{def:conj-factor}, we have
          \[
          \begin{aligned}
            \phi\left(\Delta_{\sigma,c}\Delta_{\tau,d}\right)
            &= \phi\left(\Delta_{\sigma\tau, c\sigma(d)\mathcal{B}^{2}_{x}(\sigma,\tau)}\right) \\
            &= c\sigma(d)\mathcal{B}^{2}_{x}(\sigma,\tau) \,\cdot\, \phi\left(\Delta_{\sigma\tau, 1}\right) \\
            &= c\sigma(d)\mathcal{B}^{2}_{x}(\sigma,\tau) \,\cdot\, x_{\sigma\tau} \\
            &= c\sigma(d)\comp_{x_{\sigma},x_{\tau},x_{\sigma\tau}} \,\cdot\, x_{\sigma\tau} \\
            &= c\sigma(d)\,\cdot\, \iota_{A}\left(\comp_{x_{\sigma},x_{\tau},x_{\sigma\tau}}\right)x_{\sigma\tau} \\
            &= c\sigma(d) \,\cdot\, x_{\sigma}x_{\tau}\\
            %%%%
            %%%%
            \phi\left(\Delta_{\sigma,c}\right)\phi\left(\Delta_{\tau,d}\right)
            &=\left(c \cdot \phi\left(\Delta_{\sigma, 1}\right)\right)
              \left(d \cdot \phi\left(\Delta_{\tau, 1}\right)\right) \\
            &= \left(c \cdot x_{\sigma}\right) \left(d \cdot x_{\tau}\right)\\
            &= c \,\cdot\, x_{\sigma}\iota_{A}(d) x_{\tau} \\
            &= c\sigma_{d} \,\cdot\, x_{\sigma}x_{\tau}.
          \end{aligned}
          \]
  \end{enumerate}
\end{proof}

\begin{corollary}
  For a finite dimensional and Galois extension of field $K/F$, the relative Brauer group $K/F$ bijects to the second cohomology group $\HH^{2}\left(\gal(K/F), K^{\star}\right)$ by the following commutative diagram
  \begin{center}
  \begin{tikzcd}
    \br(K/F) \arrow{r}{\HH^2} \ar[equal]{d} & \HH^2\left(\gal(K/F), K^\star\right) \arrow[equal]{d} \\

    \br(K/ F) & \HH^2\left(\gal(K/F), K^{\star}\right) \arrow{l}{\cross}
  \end{tikzcd}.
\end{center}
\leanok
\lean{RelativeBrGroup.equivSnd}
\end{corollary}

\begin{proof}
  Exactly~\cref{lem:relative-br-snd-inverse-1} and~\cref{lem:relative-br-snd-inverse-2}.
\end{proof}

%%% Local Variables:
%%% mode: LaTeX
%%% TeX-master: "../print"
%%% End:


%%% Local Variables:
%%% mode: LaTeX
%%% TeX-master: "print"
%%% End:


\end{document}


\title{Brauer Group and Galois Cohomology} \author{Jujian Zhang \and Yunzhou
  Xie}

\begin{document}
\maketitle
\tableofcontents


\section*{Preface}
\addcontentsline{toc}{section}{Preface}
In this exposition, we describe a excruciatingly detailed proof of the following theorem:
\begin{theorem*}
  For a finite dimensional and Galois field extension $K/F$, the relative Brauer group $\br(K/F)$ is isomorphic to the second group cohomology $\HH^{2}\left(\gal(K/F),K^{\star}\right)$.
\end{theorem*}

\paragraph{} The reason for the detailed-ness is because we are aiming to formalise the proof described in the following chapters; therefore the more details, the better. We apologise for the unconventional organisation in advance --- earlier chapters sometimes use results from later chapter. For our defence, we try to categorise all the results by topics and, since this is a formalisation project, we can guarantee the readers that there is no circular reasoning.

\chapter{Central Simple Algebras}\label{sec:csa}

In this chapter we define central simple algebras. We used some results
in~\cref{sec:unclassified}.

\begin{definition}[Simple Ring] A ring $R$ is simple if the only
  two-sided-ideals of $R$ are ${0}$ and $R$. An algebra is simple if it is
  simple as a ring. \leanok \lean{IsSimpleRing}
\end{definition}

\begin{lemma}
  \label{lem:center-simple-ring}
  Let $A$ be a simple ring, then centre of $A$ is a field. \leanok
  \lean{IsSimpleRing.isField_center}
\end{lemma}
\begin{proof}
  Let $0\ne x$ be an element of centre of $A$. Then $I := \{xy | y\in A\}$ is a
  two-sided-ideal of $A$. Since $0\ne x\in I$, we have that $I = A$. Therefore
  $1 \in I$, hence $x$ is invertible.
\end{proof}

\begin{definition}[Central Algebras]
  Let $R$ be a ring and $A$ an $R$-algebra, we say $A$ is central if and only if
  the centre of $A$ is $R$ \leanok \lean{Algebra.IsCentral}
\end{definition}

\begin{remark}
  Simpleness is invariant under ring isomorphism and centrality is invariant
  under algebra isomorphism.
\end{remark}

\begin{lemma}\label{lem:simple-ring-iff}
  $R$ is a simple ring if and only if any ring homomorphism $f : R \to S$ either
  injective or $S$ is the trivial ring. \leanok
  \lean{IsSimpleRing.iff_injective_ringHom_or_subsingleton_codomain}
\end{lemma}
\begin{proof}
  If $R$ is simple, then the $\ker f$ is either $\{0\}$ or $R$. The former case
  implies that $f$ is injective while the latter case implies that $S$ is the
  trivial ring. Conversely, let $I\subseteq R$ be a two-sided-ideal. Consider
  $\pi: R \to {}^{R}/_{I}$, either $\pi$ is injective implying that $I = \{0\}$
  or that ${}^{R}/_{I}$ is the trivial ring implying that $I = R$.
\end{proof}

\begin{corollary}\label{cor:alghom-bijective-of-dim-eq}
  Assume $R$ is a field. Let $A, B$ be finite dimensional $R$-algebras where $A$
  is simple as well. Then any $R$-algebra homomorphism $f:A\to B$ is bijective
  if $\dim_{R}A=\dim_{R}B$. \leanok
  \lean{bijective_of_dim_eq_of_isCentralSimple}
\end{corollary}
\begin{proof}
  By~\cref{lem:simple-ring-iff}, $f$ is injective. Then
  $\dim_{K}\im f = \dim_{K} B - \dim_{K}\ker f = \dim_{K}B$ meaning that $f$ is
  surjective.
\end{proof}

Let $K$ be a field and $A, B$ be $K$-algebras.
\begin{lemma}
  \label{lem:tensor-central}
  If $A$ and $B$ are central $K$-algebras, $A\otimes_{K}B$ is a central
  $K$-algebra as well. \leanok \lean{IsCentralSimple.TensorProduct.isCentral}
\end{lemma}
\begin{proof}
  Assume $A$ and $B$ are central algebras, then
  by~\cref{cor:center-tensor-tensor}
  $Z\left(A\otimes_{R}B\right)=Z\left(A\right)\otimes_{R}Z\left(B\right)=R\otimes_{R}R=R$.
\end{proof}

\begin{theorem}
  \label{thm:tensor-csa}
  If $A$ is a simple $K$-algebra and $B$ is a central simple $K$-algebra,
  $A\otimes_{K}B$ is a central simple $K$-algebra as well. \leanok
  \lean{IsCentralSimple.TensorProduct.simple}
\end{theorem}
\begin{proof}
  By~\cref{lem:tensor-central}, we need to prove $A\otimes_{K}B$ is a simple
  ring. Denote $f$ as the map $A\to A\otimes_{K}B$. It is sufficient to prove
  that for every two-sided-ideal $I\subseteq A\otimes_{K}B$, we have
  $I = \left\langle f\left(f^{-1}\left(I\right)\right)\right\rangle$. Indeed,
  since $A$ is simple $f^{-1}\left(I\right)$ is either $\left\{0\right\}$ or
  $A$, if it is $\left\{0\right\}$, then $I=\left\{0\right\}$; if it is $A$,
  then $I$ is $A$ as well.

  We will prove that
  $I \le \left\langle f\left(f^{-1}\left(I\right)\right)\right\rangle$, the
  other direction is straightforward. Without loss of generality assume
  $I\ne\left\{0\right\}$. Let $\McA$ be an arbitrary basis of $A$, by
  \cref{lem:expand-tensor-in-basis}, we see that every element
  $x \in A\otimes_{K}B$ can be written as $\sum_{i=0}^{n}\McA_{i}\otimes b_{i}$
  for some natrual number $n$ and some choice of $b_{i}\in B$ and
  $\McA_{i}\in \McA$. Since $I$ is not empty, we see there exists a non-zero
  element $\omega\in I$ such that its expansion
  $\sum_{i=0}^{n}\McA_{i}\otimes b_{i}$ has the minimal $n$. In particular, all
  $b_{i}$ are non-zero and $n\ne0$. We have
  $\omega=\McA_{0}\otimes b_{0}+\sum_{i=1}^{n}\McA_{i}\otimes b_{i}$. Since $B$
  is simple, $1 \in B = \left\langle\langle b_{0} \right\rangle$; hence we write
  $1\in\sum_{j=0}^{m}x_{i}b_{0}y_{i}$ for some $x_{i},y_{i}\in B$. Define
  $\Omega := \sum_{j=0}^{m}(1\otimes x_{i})\omega(1\otimes y_{i})$ which is also
  in $I$. We write
  \[
    \begin{aligned}
      \Omega &= \McA_{0} \otimes \left(\sum_{j=0}^{m}x_{j}b_{0}y_{j}\right) + \sum_{i=1}^{n}\McA_{i}\otimes\left(\sum_{j=0}^{m}x_{j}b_{i}y_{j}\right) \\
             &= \McA_{0}\otimes 1 + \sum_{i=1}^{n}\McA_{i}\otimes\left(\sum_{j=0}^{m}x_{j}b_{i}y_{j}\right)
    \end{aligned}
  \]
  For every $\beta\in B$, we have that
  $\left(1\otimes \beta\right)\Omega - \Omega\left(1\otimes \beta\right)$ is in
  I and is equal to
  \[
    \sum_{i=1}^{n}\McA_{i}\otimes\left(\sum_{j=0}^{m}\beta x_{j}b_{i}y_{j}-x_{j}b_{i}y_{j}\beta\right),
  \]
  which is an expansion of $n-1$ terms, thus
  $\left(1\otimes\beta\right)\Omega-\Omega\left(1\otimes\beta\right)$ must be
  $0$. Hence we conclude that for all $i=1,\dots,n$,
  $\sum_{j=0}^{m}x_{j}b_{i}y_{j}\in Z\left(B\right)=K$. Hence for all
  $i=1,\dots,n$, we find a $\kappa_{i}\in K$ such that
  $\kappa_{i}=\sum_{j=0}^{m}x_{j}b_{i}y_{j}$. Hence we can calculate $\Omega$ as
  \[
    \begin{aligned}
      \Omega &= \McA_{0}\otimes 1 + \sum_{i=1}^{n}\McA_{i}\otimes\left(\sum_{j=0}^{m}\right) \\
             &= \McA_{0}\otimes 1 + \sum_{i=1}^{n}\McA_{i}\otimes\kappa_{i} \\
             &= \left(\McA_{0}+\sum_{i=1}^{n}\kappa_{i}\cdot\McA_{i}\right) \otimes 1
    \end{aligned}.
  \]
  From this, we note that
  $\McA_{0}+\sum_{i}^{n}\kappa_{i}\cdot\McA_{i}\in f^{-1}\left(I\right)$; since
  $A$ is simple, we immediately conclude that $f^{-1}\left(I\right) = A$, once
  we know $\McA_{0}+\sum_{i=1}^{n}\kappa_{i}\cdot \McA_{i}$ is not zero. If it
  is zero, by the fact that $\McA$ is a linearly independent set, we conclude
  that $1,\kappa_{1},\dots,\kappa_{n}$ are all zero; which is a contradiction.
  Since $f^{-1}\left(I\right) = A$, we know
  $\left\langle f\left(f^{-1}I\right)\right\rangle = A \otimes_{K} B$.
\end{proof}

\begin{corollary}\label{lem:cas-basechange}
  Central simple algebras are stable under base change. That is, if $L/K$ is a
  field extension and $D$ is a central simple algebra over $K$, then
  $L\otimes_{K} D$ is central simple over $L$. \leanok
  \lean{IsCentralSimple.baseChange}
\end{corollary}

\begin{proof}
  By~\cref{thm:tensor-csa}, $L\otimes_{K} D$ is simple. Let
  $x\in Z\left(L\otimes_{K}D\right)$, by~\cref{cor:center-tensor-center}, we
  have $x\in Z\left(L\right)\otimes Z\left(D\right)=Z\left(L\right)$. Without
  loss of generality, we can assume that $x = l \otimes d$ is a pure tensor,
  then $l \in Z\left(L\right)$ and $d \in K$. Therefore $x = d\cdot l \in L$.
\end{proof}

%%% Local Variables:
%%% mode: LaTeX
%%% TeX-master: "../print"
%%% End:

\chapter{Morita Equivalence}\label{chap:morita}

This chapter intertwine with \cref{chap:wed-artin}: \cref{sec:stacks-074e} depends on \cref{sec:wed-artin-proof}; while \cref{sec:wed-artin-unique} depends on \cref{sec:stacks-074e}.

\section{Construction of the equivalence}\label{sec:morita-construction}

Let $R$ be a ring and $0 \ne n\in \mathbb{N}$. In this chapter, we prove that the category $R$-modules and the category of $\Mat_{n}(R)$-modules are equivalent. Then we use the equivalence to prove several useful lemmas.

\begin{construction}\label{con:morita-eqv-functor0}
  \leanok
  \lean{matrix_smul_vec_def}
  If $M$ is an $R$-module, we have a natural $\Mat_{n}(R)$-module structure on $\hat{M}:=M^{n}$ given by $(m_{ij})\cdot (v_{k})=\sum_{j}m_{ij}\cdot v_{j}$.
  If $f : M \to N$ is an $R$-linear map, then $\hat{f} : M^{n}\to N^{n}$ given by $(v_{i}) \mapsto (f(v_{i}))$ is a $\Mat_{n}(R)$-linear map. Thus we have a well-defined functor $\MOD_{R} \Longrightarrow \MOD_{\Mat_{n}(R)}$.
\end{construction}

Note that all modules are assumed to be left modules; when we need to consider right $R$-modules, we will consider left $R^{\opp}$-modules instead. We use $\delta_{ij}$ to denote the matrix whose $(i,j)$-th entry is $1$ and $0$ elsewhere. $\delta_{ij}$ forms a basis for matrices.

\begin{construction}\label{con:morita-eqv-functor1}
  \leanok
  \lean{fromModuleCatOverMatrix.smul_α_coe}
  If $M$ is a $\Mat_{n}(R)$-module, then $\tilde{M} := \{\delta_{ij}\cdot m | m \in M\} \subseteq M$ is an $R$-module given by $r \cdot (\delta_{ij}\cdot m) := (r\cdot \delta_{ij})\cdot m$. More over if $f : M \to N$ is a $\Mat_{n}(R)$-linear map, $\tilde{f} : \tilde{M} \to \tilde{N}$ given by $f$ is $R$-linear. Hence, we have a functor $\MOD_{\Mat_{n}(R)}\Longrightarrow \MOD_{R}$.
\end{construction}

\begin{theorem}[Morita Equivalence]\label{thm:morita}
  The functors constructed in \cref{con:morita-eqv-functor0} and \cref{con:morita-eqv-functor1} form an equivalence of category.
  \leanok
  \lean{moritaEquivalentToMatrix}
\end{theorem}

\begin{proof}
  Let $M$ be an $R$-module, then the unit $\tilde{\hat{M}} \cong M$ is given by
  \[
    \begin{aligned}
      x \mapsto \sum_{j} x_{j} \\
      (x,0,\dots,0) \mapsfrom x
    \end{aligned}
  \]

  Let $M$ be an $\Mat_{n}(R)$-module, then the counit $\hat{\tilde{M}}\cong M$ is given by $m \mapsto (\delta_{i0}\cdot m)$. This map is both injective and surjective.
\end{proof}

\section{Stacks 074E}\label{sec:stacks-074e}

Let $A$ be a finite dimensional simple $k$-algebra.

\begin{lemma}
  Let $M$ and $N$ be simple $A$-modules, then $M$ and $N$ are isomorphic as $A$-modules.
  \leanok
  \lean{linearEquiv_of_isSimpleModule_over_simple_ring}
\end{lemma}

\begin{proof}
  By~\cref{thm:wed-artin-algebra}, there exists non-zero $n\in\mathbb N$, $k$-division algebra $D$ such that $A\cong \Mat_{n}(D)$ as $k$-algebras. Then by~\cref{thm:morita}, we have equivalence of category $e : \MOD_{A}\cong\MOD_{D}$. Since simple module is a categorical notion (it can be defined in terms monomorphisms), $e(M)$ and $e(N)$ are simple $D$-modules. Since $D$ is a division ring, $e(M)$ and $e(N)$ are isomorphic as $D$-modules, therefore $M$ and $N$ are isomorphic as $A$-modules.
\end{proof}

\begin{lemma}
  Let $M$ be an $A$-module, there exists a simple $A$-module $S$ such that $M$ is a direct sum of copies of $S$, i.e. $M \cong \bigoplus_{i \in \iota} S$ for some indexing set $\iota$.
  \leanok
  \lean{directSum_simple_module_over_simple_ring}
\end{lemma}

%%% Local Variables:
%%% mode: LaTeX
%%% TeX-master: "../print"
%%% End:

\chapter{Results in Noncommutative
  Algebra}\label{sec:results-noncommutative-algebra}
% \setcounter{chapter}{-1}
\section{A Collection of Useful Lemmas}\label{sec:unclassified}

In section, we collect some lemmas that does not really belong to anywhere.

\subsection{Tensor Product}

\begin{lemma}\label{lem:expand-tensor-in-basis}
  Let $M$ and $N$ be $R$-modules such that $\mathcal{C}_{i\in\iota}$ is a basis for $N$, then every elements of $x \in M \otimes_{R} N$ can be uniquely written as $\sum_{i\in\iota}m_{i}\otimes \mathcal{C}_{i}$ where only finitely many $m_{i}$'s are non-zero
\end{lemma}

\begin{proof}
  Given the basis $\mathcal{C}$, we have $R$-linear isomorphism $N \cong\bigoplus_{i\in\iota}R$, hence $M\otimes_{R}N \cong \bigoplus_{i\in\iota}(M\otimes_{R}R)\cong\bigoplus_{i\in\iota}M$ as $R$-modules.
\end{proof}
By switching $M$ and $N$, the symmetric statement goes without saying.

% \noindent\rule{\textwidth}{0.2pt}

\begin{lemma}

  Let $K$ be a field, $M$ and $N$ be flat $K$-modules. Suppose $p \subseteq M$ and $q \subseteq N$ are $K$-submodules, then $(p \otimes_{K} N) \sqcap (M \otimes_{K} q) = p \otimes_{K} q$ as $K$-submodules.
\end{lemma}

\begin{proof}
  The hard direction is to show $(p \otimes_{R} N) \sqcap (M \otimes_{R} q) \le p \otimes_{R} q$. Consider the following diagram:
  \begin{center}
    \begin{tikzcd}
      p \otimes_{K} q \ar[r, "u"] \ar[d, "\alpha"] & M \otimes_{K} q \ar[r, "v"] \ar[d, "\beta"] & {}^{M}/_{p} \otimes_{K} q \ar[d, "\gamma"] \\
      p \otimes_{K} N \ar[r, "u'"] & M \otimes_{K} N \ar[r, "v'"] & ^{M}/_{p} \otimes_{K} N
    \end{tikzcd}
  \end{center}
  Since $^{M}/_{p}$ is flat, $\gamma$ is injective.
  Let $z \in (p \otimes_{R} N) \sqcap (M \otimes_{R} q) = \im \beta \sqcap \im u'$. By abusing notation, replace $z$ with some elements of $M \otimes_{K} q$ and continue with $\beta(z)\in\im \beta \sqcap \im u'$. Since $v'(\beta(z))=\gamma(v(z))$ and that $\beta(z)\in \im u'$, we conclude that $\gamma(v(z))=0$, that is $z\in\ker v=\im u$. We abuse notation again, let $z \in p \otimes_{K} q$, we need to show $\beta(u(z))\in\im \beta \sqcap \im u'$, but $\beta\circ u=u'\circ\alpha$, we finish the proof.
\end{proof}

\subsection{Centralizer and Center}
Let $R$ be a commutative ring and $A$, $B$ be two $R$-algebras. We denote centralizer of $S\subseteq A$ by $C_{A}S$ and centre of $A$ by $Z(A)$.

\begin{lemma}
  Let $S, T$ be two subalgebras of $A$, then $C_{A}(S\sqcup T)=C_{A}(S)\sqcap C_{A}(T)$.
  \leanok
  \lean{Subalgebra.centralizer_sup}
\end{lemma}
This lemma can be generalized to centralizers of arbitrary supremum of subalgebras.

\begin{lemma}
  If we assume $B$ is free as $R$-module, then $C_{A\otimes_{R}B}\left(\im\left(A\to A\otimes_{R}B\right)\right)$ is $Z(A) \otimes_{R} B$
\end{lemma}
A symmetric statement goes without saying.
\begin{proof}
  Let $w\in C_{A\otimes_{R}B}\left(\im\left(A\to A\otimes_{R}B\right)\right)$. Since $B$ is free, we choose an arbitrary basis $\mathcal{B}$; by~\cref{lem:expand-tensor-in-basis}, we write $w = \sum_{i}m_{i}\otimes_{K}\mathcal{B}_{i}$. It is sufficient to show that $m_{i}\in Z(A)$ for all $i$. Let $a \in A$, we need to show that $m_{i}\cdot a = a \cdot m_{i}$. Since $w$ is in the centralizer, $w \cdot (a\otimes 1) = (a\otimes 1)\cdot w$. Hence we have $\sum_{i}(a\cdot m_{i})\otimes\mathcal{B}_{i}=\sum_{i}(m_{i}\cdot a)\otimes\mathcal{B}_{i}$. By the uniqueness of~\cref{lem:expand-tensor-in-basis}, we conclude $a\cdot m_{i}=m_{i}\cdot a$.
\end{proof}

\begin{remark}
  Since $\im\left(R\otimes_{R}B\to A\otimes_{R}B\right)=\im\left(A\to A\otimes_{R}B\right)$, we conclude its centralizer in $A\otimes_{R}B$ is $Z(A)\otimes_{R}B$ as well.
\end{remark}

\begin{lemma}
  \label{lem:center-tensor}
  Assume $R$ is a field. The centre of $A\otimes_{R} B$ is $Z\left(A\right)\otimes_{R}Z\left(B\right)$.
  \leanok
  \lean{IsCentralSimple.center_tensorProduct}
\end{lemma}
\begin{proof}
 From previous lemmas, we know that $Z\left(A\otimes_{R}B\right)$ is equal to $Z\left(A\right)\otimes_{R}B \sqcap A\otimes_{R}Z\left(B\right)$ which is $Z\left(A\right)\otimes_{R}Z\left(B\right)$
\end{proof}

%%% Local Variables:
%%% mode: LaTeX
%%% TeX-master: "../print"
%%% End:

\chapter{Wedderburn-Artin Theorem for Simple Rings}

\section{Classification of Simple Rings}

\begin{lemma}[minimal ideal of simple rings]\label{lemma:min-ideal-simple-ring}
  Let $A$ be a ring and $I$ a non-trivial minimal left ideal of $A$, then $I$ is a simple $A$-module.
  \leanok
  \lean{minimal_ideal_isSimpleModule}
\end{lemma}
\begin{proof}
Let $J\le I$ be an $A$-submodule of $I$, suppose $J$ is non-trivial, we prove that $J=I$. Then the image $J'$ of $J$ under $I \hookrightarrow A$ is a non-trivial left ideal of $A$. Since $I \hookrightarrow A$ is injective, it is sufficient to prove that $J' = I$. This is because $J'\le I$ and $J' \not< J$.
\end{proof}

\begin{lemma}
  Let $A$ be a simple ring and $I$ a non-trivial left ideal. One can write $1 \in A$ as $\sum_{i=0}^{n}x_{i}y_{i}$ for some $x_{i}\in I$ and $y_{i} \in A$.
\end{lemma}

\begin{proof}
  Let $I'$ be the two-sided ideal spanned by $I$. Then since $A$ is a simple ring, $I' = A$. Thus $1 \in I'$. One can write $1 \in A$ as $\sum_{i}a_{i}x_{i}b_{i}$ for some $x_{i}\in I$ and $a_{i},b_{i}\in A$, since $I$ is a left ideal $a_{i}x_{i}\in I$ as well.
\end{proof}

Now, we can find the smallest $n$ such that $1\in A$ can be written as $\sum_{i=0}^{n}x_{i}y_{i}$ for some $x_{i}\in I$ and $y_{i}\in A$. Let us fix the notations $n$, $x_{i}$ and $y_{i}$

\begin{lemma} The $n$, $x_{i}$ and $y_{i}$ are all non-zero.
  \leanok
  \lean{Wedderburn_Artin.aux.nxi_ne_zero}
\end{lemma}

\begin{proof}
  If $n$ is $0$, then $1 = 0$ in $A$, but all simple rings are non-trivial.
  We argue by contradiction to prove that all $x_{i}$ and $y_{i}$ are non-zero. Assume there exists a $j$ such that $y_{j}\ne0$ implies $x_{j} = 0$. Without loss of generality, we assume $j = 0$. Then $1 = \sum_{i=0}^{n}x_{i}y_{i}=\sum_{i=1}^{n}x_{i}y_{i}$. This contradicts the minimality of $n$.
\end{proof}

\begin{theorem}[Wedderburn]\label{thm:wed}
  Let $A$ be a simple ring and $I$ a non-trivial minimal left ideal. Then there exists a non-zero $n \in \mathbb{N}$ such that $A \cong I^{n}$ as $A$-modules.
  \leanok
  \lean{Wedderburn_Artin.aux.equivIdeal}
  \uses{lemma:min-ideal-simple-ring}
\end{theorem}

\begin{proof}
  We continue to write $1 = \sum_{i=0}^{n}x_{i}y_{i}$ in the shortest possible manner. Then we can define an $A$-linear map $g : I^{n}\to A$ by $(v_{i})\mapsto \sum v_{i}y_{i}$. Then $g$ is surjective: if $a \in A$, then $(ax_{i})$ is mapped to $a$ under $g$. $g$ is injective as well: support $g(v_{i})=0=\sum_{i}v_{i}y_{i}$ with $(v_{i})$ not all zero. Without loss of generality, we assume $v_{0} \ne 0$, then the ideal $\langle v_{0}\rangle$ is equal to $I$ (since $I$ is simple\cref{lemma:min-ideal-simple-ring}). Thus $x_{0}\in I = \langle v_{0}\rangle$; implying that $x_{0}=r\cdot v_{0}$ for some $r\in A$. Thus $1=1 - r\cdot 0 = \sum_{i=0}^{n}x_{i}y_{i}-\sum_{i=0}^{n}r\cdot v_{i}y_{i}$. In this way, we cancelled the term at $i=0$, contradicting the minimality of $n$. Hence $g$ is an isomorphism.
\end{proof}

\begin{theorem}[Wedderburn-Artin (Ideal)]
  Let $A$ be an Artinian simple ring. There exists a non-zero $n$ and an ideal $I \subseteq A$ such that $I$ is simple as an $A$-module and $A \cong I^{n}$ as $A$-module.
  \label{thm:wed-artin-ideal}
  \leanok
  \lean{Wedderburn_Artin_ideal_version}
  \uses{thm:wed}
\end{theorem}

\begin{proof}
  By~\cref{thm:wed}, we only need a minimal left ideal. Since $A$ is Artinian, such ideal exists.
\end{proof}

\begin{theorem}[Wedderburn-Artin (Algebra)]
  Let $K$ be a field and $B$ an finite dimensional simple algebra over $K$. There exists a non-zero $n\in\mathbb{N}$ and a division $K$-algebra $S$ such that $B \cong \Mat_{n}(S)$.
  \leanok
  \lean{Wedderburn_Artin_algebra_version}
\end{theorem}

\begin{proof}
  By~\cref{thm:wed-artin-ideal}, we can find a $n$ and a minimal left ideal $I$ such $A \cong I^{n}$ as $A$-modules. Note that ${\left(\End_{B}I\right)}^{\opp}$ is a division ring. Then since $B^{\opp}\cong\End_{B}B\cong{\End_{B}(I^{n})}\cong\Mat_{n}\left(\End_{B}I\right)$ as rings, we have $e : B\cong \Mat_{n}\left(\End_{B}I\right)^{\opp}$ as rings. We also have a $K$-algebra structure on ${\left(\End_{B} I\right)}^{\opp}$ given by $(a \cdot f)(x)=f(a\cdot x)$, and this algebra structure promotes the ring isomorphism $e$ to a $k$-algebra isomorphism.
\end{proof}

\section{Uniqueness of the classification}
In the previous section, we know that finite dimensional simple $K$-algebra $B$ over is in fact a matrix algebras of a division $K$-algebra $S$. In this section, we prove that the division algebra $S$ is essentially unique.

%%% Local Variables:
%%% mode: LaTeX
%%% End:

\chapter{Skolem-Noether Theorem}

Let $K$ be a field, $A$, $B$ be $K$-algebras where $A$ is finite $K$-dimensional. Let $M$ be a simple $A$-module.

\begin{construction}
  For any $K$-algebra homomorphism $f : B \to A$, we give $M$ a $B \otimes_{K}\End_{A}M$-module structure by defining $(b\otimes l)\cdot m$ to be $f(b)\cdot l(m)$. To emphasis $f$, we denote $M$ with the $B\otimes_{K}\End_{A}M$-module structure by $M^{f}$.
  \leanok
  \lean{IsMod}
\end{construction}

\begin{lemma}
  Let $f : B \to A$ be a $K$-algebra homomorphism, $M^{f}$ is finitely generated as a $B\otimes_{K}\End_{A}M$-module.
  \leanok
  \lean{module_inst_findim}
\end{lemma}
\begin{proof}
  Since $M$ is a finite $A$-module and $A$ a finite dimensional $K$-vector space, $M$ is a finite dimensional $K$-vector space as well. Suppose $S \subseteq M$ generates $M$ as $K$-module, the claim is that $S$ generates $M^{f}$ as well. Let $x \in M^{f}$, we write $x = \sum_{i}\lambda_{i}\cdot s_{i}$ with $\lambda_{i}\in K$ and $s_{i}\in S$.
  Note that $\lambda_{i}\cdot s_{i} = (\rho(\lambda_{i})\otimes\mathbf{1}_{M})$ in $M^{f}$ where $\rho : K \to B$ is the map giving $B$ its $K$-algebra structure. Hence $x$ is in the span of $S$ in $M^{f}$ as well.
\end{proof}

%%% Local Variables:
%%% mode: LaTeX
%%% TeX-master: "../print"
%%% End:

\section{Double Centralizer Theorem}\label{sec:double-centralizer}

In this section let $F$ be a field and $A$ an $F$-algebra. Define
$\McL_{A} \subseteq \End_{F}A$ to be
\[\left\{f : A \to A|f(x)=ax~\text{for some}~a\in A\right\},\]
i.e. $F$-linear maps defined by left multiplication; similarly define
$\McR_{A}$. Note that $\McL_{A}$ and $\McR_A$ are $F$-subalgebras of
$\End_{F}A$. When we need to stree the underlying field is $F$, we also write
$\McL_{A}^{F}$ and $\McR_{A}^{F}$. We assume $A$ to be a finite dimensional
central simple $F$-algebra.
\begin{lemma}
  \label{lem:centralizer-mul-left-le}
  The centralizer of $\McL_{A}$ in $\End_{F}A$ is smaller than or equal to
  $\McR_{A}$:
  \[
    C_{\End_{F}A}\left(\McL_{A}\right) \le \McR_{A}.
  \]
  \leanok \lean{centralizer_mulLeft_le_of_isCentralSimple}
\end{lemma}

\begin{proof}
  Indeed, let $x \in C_{\End_{F}A}\left(\McL_{A}\right)$. Recall
  from~\cref{con:self-tensor-opp-iso-end} that
  $e : A\otimes_{F}A^{\opp}\cong \End_{F}A$ as $F$-algebras. Then $e^{-1}(x)$ is
  in
  $C_{A\otimes_{F}A^{\opp}}\left(\im\left(A\to A\otimes_{F}A^{\opp}\right)\right)$
  (for $e$ sends $a\otimes 1$ to the $F$-linear map $(a\cdot\bullet)$). Since
  $C_{A\otimes_{F}A^{\opp}}\left(\im\left(A\to A\otimes_{F}A^{\opp}\right)\right) = Z(A)\otimes_{F}A^{\opp}=F\otimes_{F}A^{\opp}=\im\left(A^{\opp}\to A\otimes_{F}A^{\opp}\right)$,
  we find some $y\in A^{\opp}$ such that $1 \otimes y = e^{-1}(x)$. Therefore
  $e\left(1\otimes y\right) = x$; but $e\left(1\otimes y\right)$ is in
  $\McR_{A}$ for it is the linear map $(\bullet\cdot y)$.
\end{proof}

\begin{remark}\label{rem:mul-left-center-linear}
  For any $F$-algebra $A$, every element in $C_{\End_{F}A}\left(\McL_{A}\right)$
  is in fact $Z(A)$-linear. Let $x\in C_{\End_{F}A}\left(\McL_{A}\right)$,
  $z\in Z(A)$ and $a \in A$, we have $x(z\cdot a) = z\cdot x(a)$ because $x$
  commutes with the linear map $\left(z\cdot\bullet\right)$.
\end{remark}

\begin{remark}
  $A$ is a $Z(A)$-algebra whose algebra structure is given by
  $Z(A)\hookrightarrow A$. By~\cref{lem:center-simple-ring}, $Z(A)$ is a field.
  $A$ is finite dimensional as a $Z(A)$-module because of the tower $A/Z(A)/F$.
\end{remark}

 \begin{lemma}
   As $F$-algebras, we have $\McR_{A}\cong A^{\mop}$. \leanok
   \lean{Module.End.rightMulEquiv}
 \end{lemma}
 \begin{proof}
   We prove the map $A \to \McR_{A}$ is bijective. It is injective because if
   $\left(\bullet \cdot a\right) = \left(\bullet\cdot b\right)$, then
   $a = 1 \cdot a = 1 \cdot b = b$. The map is surjective by the definition of
   $\McR_{A}$.
 \end{proof}
 \begin{lemma}\label{lem:centralizer-mul-left-eq-mul-right}
   Let $B$ be any simple $F$-algebra ({\em not\/} necessarily central). The
   centralizer of $\McL_{B}$ in $\End_{F}A$ is equal to $\McR_{B}$. \leanok
   \lean{centralizer_mulLeft}
 \end{lemma}
 \begin{proof}
   It is straightforward to show
   $\McR_{B}^{F}\le C_{\End_{F}A}\left(\McL_{B}^{F}\right)$. So we only need to
   prove $C_{\End_{F}A}\left(\McL_{B}^{F}\right)\le \McR_{B}^{F}$.
   By~\cref{lem:centralizer-mul-left-le}, since $B$ is a central simple finite
   dimensional $Z(B)$-algebra, we have that
   \[
     C_{\End_{Z(B)}B}\left(\McL_{B}^{Z(B)}\right) \le \McR_{B}^{Z(B)}.
   \]
   Suppose $f\in\End_{F}B$ is in $C_{\End_{F}B}B$, by
   ~\cref{rem:mul-left-center-linear}, $f$ is $Z(B)$-linear as well. Then $f$ is
   in $\McR_{B}^{Z(B)}$; that is $f$ is equal to $\left(\bullet\cdot b\right)$
   for some $b \in B$ as $Z(B)$-linear maps. Then $f$ is also equal to
   $\left(\bullet\cdot b\right)$ as $F$-linear maps.
 \end{proof}

 \begin{construction}

 \end{construction}
 
%%% Local Variables:
%%% mode: LaTeX
%%% TeX-master: "../print"
%%% End:

\chapter{Brauer Group}\label{cha:brauer-group}

\section{Construction of Brauer Group}
Let $K$ be a field. We denote the class of finite dimensional central simple $K$-algebras as $\CSA_{K}$. When $K$ is clear, we drop the subscript.

\begin{remark}
  By~\cref{lem:tensor-central} and~\cref{thm:tensor-csa}, $\CSA$ is closed under tensor product, that is if $A, B\in\CSA$, we have $A\otimes_{K} B\in\CSA$ as well.
\end{remark}

\begin{definition}[Brauer Equivalence]
  For any two $A, B\in\CSA$, we say $A$ and $B$ are Brauer equivalent, when there exists $m, n \in \mathbb{N}_{\ge0}$ such that $\Mat_{m}(A)\cong \Mat_{n}B$ as $K$-algebras. We denote this relation as $A\sim_{\Br_{K}} B$, when $K$ is clear, we drop the subscript.
\end{definition}

\begin{remark}
  Isomorphic $K$-algebras are Brauer equivalent.
\end{remark}

\begin{lemma}
  $\sim_{\Br}$ is reflexive.
  \leanok
  \lean{IsBrauerEquivalent.refl}
\end{lemma}
\begin{proof}
  Indeed, $A \cong \Mat_{1}(A)$ as $K$-algerbas.
\end{proof}

\begin{lemma}
  $\sim_{\Br}$ is symmetric.
  \leanok
  \lean{IsBrauerEquivalent.symm}
\end{lemma}
\begin{proof}
  Indeed, just exchange $m$ and $n$.
\end{proof}

\begin{lemma}
  $\sim_{\Br}$ is transitive.
  \leanok
  \lean{IsBrauerEquivalent.trans}
\end{lemma}
\begin{proof}
  Let $A\sim_{\Br}B$ and $B\sim_{\Br}C$; that is for some $m,n,p, q\in\mathbb{N}_{\ge0}$ we have $\Mat_{n}(A)\cong\Mat_{m}(B)$ and $\Mat_{p}(B)\cong \Mat_{q}(C)$ as $K$-algebras. Hence, from~\cref{con:matrix-matrix}, we have the following:
  \[
    \begin{aligned}
      \Mat_{np}(A)&\cong \Mat_{p}\left(\Mat_{n}(A)\right)\cong\Mat_{p}\left(\Mat_{m}(B)\right)\\
                  &\cong\Mat_{mp}(B)\cong\Mat_{m}\left(\Mat_{p}(B)\right)\\
      &\cong\Mat_{m}\left(\Mat_{q}(C)\right)\cong\Mat_{mq}(C).
    \end{aligned}
  \]
  In another word, $A\sim_{\Br}C$.
\end{proof}

Hence $\sim_{\Br}$ is really an equivalence relation, we denote the quotient ${}^{\CSA}/_{\sim_{\Br}}$ as $\Br(K)$.

\begin{lemma}\label{lem:br-mul-wd}
  $(\bullet\otimes_{K}\bullet):\CSA\times\CSA \to \CSA$ descends to a function on $\Br(K)$.
  \leanok
  \lean{BrauerGroup.eqv_tensor_eqv}
\end{lemma}
\begin{proof}
  We need to prove that for all $A, B, C, D \in \CSA$ such that $A\sim_{\Br} B$ and $C\sim_{\Br}D$, $A\otimes_{R}C \sim_{\Br} B\otimes_{R}D$ as well.
  Suppose $\Mat_{m}(A)\cong \Mat_{n}(B)$ as $K$-algebras and $\Mat_{p}(C)\cong\Mat_{q}(D)$, by~\cref{con:matrix-tensor-matrix}, we have
  \[
    \begin{aligned}
      \Mat_{mp}\left(A\otimes_{R} C\right)&\cong \Mat_{m}(A)\otimes_{R}\Mat_{p}(C) \\
                                          &\cong \Mat_{n}(B)\otimes_{R}\Mat_{q}(D)\\
      &\cong \Mat_{nq}\left(B\otimes_{R}D\right).
    \end{aligned}
  \]

\end{proof}

\begin{construction}[Brauer Group]\label{con:br}
  $\Br(K)$ forms a group under $[A]_{\sim_{\Br}}\cdot[B]_{\sim_{\Br}}=[A\otimes_{K}B]_{\sim_{\Br}}$ with neutral element $[K]_{\sim_{\Br}}$ where $A, B\in\CSA$ and $[A]_{\sim_{Br}}^{-1}=[A^{\opp}]_{\sim_{\Br}}$. We need to prove the following properties:
  \begin{enumerate}
    \item associativity: for all $A, B, C\in\CSA$, $[A]_{\sim_{\Br}}\cdot\left([B]_{\sim_{\Br}}\cdot[C]_{\sim_{\Br}}\right)=\left([A]_{\sim_{\Br}}\cdot[B]_{\sim_{\Br}}\right)\cdot [C]_{\sim_{\Br}}$ because $A\ox_{R}\left(B\ox_{R}C\right)\cong \left(A\ox_{R}B\right)\ox_{R}C$ as $K$-algebras.
    \item neutral element: for all $A\in \CSA$, $[K]_{\sim_{\Br}}\cdot[A]_{\sim_{\Br}}=[A]_{\sim_{\Br}}=[A]_{\sim_{\Br}}\cdot [K]_{\sim_{\Br}}$. Since $[K]_{\sim_{\Br}}\cdot[A]_{\sim_{\Br}}=[K\ox_{K} A]_{\sim_{\Br}}$, in~\cref{con:matrix-tensor-matrix}, we see that $\Mat_{n}(A)\cong A\ox_{K}\Mat_{n}(K)$, by~\cref{lem:br-mul-wd}, $A\ox_{K}\Mat_{n}(K)$ is Brauer equivalent to $A\ox_{K} K$ since $K\sim_{\Br}\Mat_{n}(K)$.
    \item cancellation: for all $A \in \CSA$, we need $[A]_{\sim_{\Br}}\cdot[A^{\opp}]_{\sim_{\Br}}$, that is we want $A\ox_{K}A^{\opp}\sim_{\Br} K$. By~\cref{con:self-tensor-opp-iso-end}, we have $A\ox_{K}A^{\opp}\cong \End_{K}A$ which is isomorphic to $\Mat_{\dim_{K}A}(K)$ as $K$-algebras.
  \end{enumerate}
  \leanok
  \lean{BrauerGroup.Bruaer_Group}
\end{construction}

\begin{theorem}
  \label{thm:br-triv-alg-closed}
  If $K$ is algebraically closed, $\Br(K)$ is trivial; in particular $\br_{n}(\mathbb{C})$ is trivial.
  \leanok
  \lean{BrauerGroup.Alg_closed_eq_one}
\end{theorem}
\begin{proof}
  We need to show that every $A\in\CSA$ is isomorphic to $\Mat_{n}(K)$ for some $K$ when $K$ is algebraically closed. Indeed, by~\cref{thm:wed-artin-algebra}, $A \cong \Mat_{n}(D)$ for some division algebra $D$ and $n\in\nat_{\ge0}$. Since $K$ is algebraically closed and $D$ is an integral domain and finite dimensional, the structure morphism $\rho: K\to D$ is a isomorphism; therefore $A\cong \Mat_{n}(K)$.
\end{proof}

\begin{lemma}
  \label{lem:common-div-alg}
  \leanok
  \lean{IsBrauerEquivalent.exists_common_division_algebra}
  Let $A, B \in \csa_{K}$. There exists a division $K$-algebra $D$ and non-zero $m,n\in\mathbb{N}$ such that $A\cong\mat_{m}(D)$ and $B\cong\mat_{n}(D)$ as $K$-algebras.
\end{lemma}

\begin{proof}
  By~\cref{thm:wed-artin-algebra}, we can find division algebras $S_{A}, S_{B}$ and non-zero $m, n\in\mathbb{N}$ such that $A\cong\mat_{n}\left(S_{A}\right)$ and $B\cong\mat_{m}\left(S_{B}\right)$ as $K$-algebras. Hence $B\sim_{\br}A\sim_{\br}\mat_{n}\left(S_{A}\right)\sim_{\br}S_{A}$, in another word, for some non-zero $a, a'\in\mathbb{N}$, we have $\mat_{a}(B)\cong\mat_{a'}\left(S_{A}\right)$ as $K$-algebras. Hence, by~\cref{thm:wed-artin-uniq}, we have that $S_{A}\cong S_{B}$ as $K$-algebras and the lemma is proved.
\end{proof}

\section{Base Change}
In this section let $E/K$ be a field extension. We have seen in~\cref{cor:csa-basechange} that if $A\in\CSA_{K}$ then $E\otimes_{K}A\in\CSA_{E}$; therefore we have a set-theoretic function $\CSA_{K}\to\CSA_{E}$. In this section we prove that this descends to a group homomorphism $\Br(K)\to\Br(E)$. For brevity, if $A\in \CSA_{K}$, we dentoe $E\ox_{K}A$ as $A_{E}$ when this causes no confusion.
\begin{construction}
  \label{con:br-base-change}
  We will construct a series of isomorphisms (either over $K$ or $E$) to arrive at the conclusion that $A\sim_{\Br_{K}}B$ implies $A_{E}\sim_{\Br_{E}}B_{E}$. Assume $m,n\in\nat_{\ge0}$ are such that $\Mat_{m}(A)\cong\Mat_{n}(B)$ are $K$-algebras. Then we do the following calculation: as $E$-algebras
  \[
    \begin{aligned}
      \Mat_{m}\left(A_{E}\right)
      &\cong A_{E}\otimes_{E}\Mat_{m}(E) &\text{see~\cref{con:matrix-tensor-matrix}}\\
      &\cong A_{E}\otimes_{E}\left(E\ox_{K} \Mat_{m}(K)\right) & \text{see}~\dagger\\
      &\cong E \ox_{K}\left(A\ox_{K}\Mat_{m}(K)\right) &\text{see}~\ddagger\\
      &\cong E\ox_{K}\Mat_{m}(A) &\text{see~\cref{con:matrix-tensor-matrix} and}~\dagger\!\!\dagger \\
      \Mat_{n}\left(B_{E}\right) &\cong E\ox_{K}\Mat_{n}(B) &\text{same as the case of}~A\\
      Mat_{m}\left(A_{E}\right)&\cong \Mat_{n}\left(B_{E}\right) &\text{see}~\dagger\!\!\dagger
    \end{aligned}.
  \]

   \noindent$\dagger$: Wee need to check $\Mat_{m}(E)\cong E\otimes_{K}\Mat_{m}K$ as $E$-algebras since~\cref{con:matrix-tensor-matrix} only gives a $K$-algebra isomorphism. If $e \in E$, then its image in $E\ox_{K}\Mat_{m}(K)$ is $e\otimes 1$ and its image in $\Mat_{m}(E)$ is $\diag(e)$ which under the $K$-algebra isomorphism is mapped to $\sum_{ij}\diag(e)_{ij}\cdot \delta_{ij}=e\ox1$.

   \noindent$\ddagger$: This is defined by combining two $E$-algebra homomorphisms
   \[
     A_{E}\to A_{E}\ox_{K}\Mat_{m}(K)\to E\ox_{K}\left(A\ox_{K}\Mat_{m}(K)\right)\]
     and
     \[
       E\ox_{K}\Mat_{m}(K)\to \left(E\ox_{K}\Mat_{m}(K)\right)\ox_{K}A \to E\ox_{K}(A\ox_{K}\Mat_{m}(K)).
    \]
   Since $\left(E\ox_{K}A\right)\ox_{E}\left(E\ox_{K}\Mat_{m}(K)\right)$ is a simple ring, this morphism is automatically injective. It is surjective as well: let $x\in E\ox_{K}\left(A\ox_{K}\Mat_{m}(K)\right)$, without loss of generality, assume $x=e\ox (a\ox\delta_{ij})$ for some $e\in E$, $a\in A$. Then precisely $\left(e\ox a\right)\ox\left(1\ox\delta_{ij}\right)$ is mapped to $x$.

   \noindent$\dagger\!\dagger$: a $K$-algebra isomorphism $A\cong B$ gives an $E$-algebra isomorphism $E\ox_{K}A \cong E\ox_{K} B$.

   Thus we have a well defined function $\Br(K)\to\Br(E)$. We now check that this is a group homomorphism. $[K]_{\sim_{\Br_{K}}}$ is mapped to $[E\ox_{K}K]_{\sim_{\Br_{E}}}$ but $E\ox_{K}K\cong E$ as $E$-algebra. For $A, B\in\CSA_{K}$, we have that $[AB]_{\sim_{\Br_{K}}}$ is mapped to $\left(A\ox_{K}B\right)_{E}\cong A_{E}\ox_{E} B_{E}$ as $E$-algebras; hence $[AB]_{\sim_{\Br_{K}}}$ and $[A]_{\sim_{\Br_{K}}}\cdot[B]_{\sim_{\Br_{K}}}$ have the same image under base change.
   \leanok
   \lean{BrauerGroupHom.BaseChange}
 \end{construction}
 Denote the base change morphism in~\cref{con:br-base-change} as $\Br_{K}^{E}$.
 \begin{lemma}
   $\Br_{K}^{K}$ is identity.
   \leanok
 \end{lemma}
 \begin{proof}
   If $A\in\CSA$, then $A\sim_{\Br}K\ox_{K}A$.
 \end{proof}

 \begin{lemma}\label{lem:br-base-change-self}
   Consider the tower of field extension $E/F/K$,
   \[
     \Br_{K}^{E} = \Br_{E}^{F}\circ\Br_{K}^{E}.
   \]
 \end{lemma}

 \begin{proof}
   If $A\in\CSA_{K}$, then $E\ox_{F}\left(F\ox_{K}A\right)$ is isomorphic to $E \ox_{F} A$ as $E$-algebras.
   \leanok
   \lean{BrauerGroupHom.baseChange_idem}
 \end{proof}

 \begin{corollary}\label{lem:br-base-change-tower}
   $\Br$ forms a functor from category of field to category of abelian groups.
   \leanok
   \lean{BrauerGroupHom.Br}
 \end{corollary}
 \begin{proof}
   This is the categorical version of~\cref{lem:br-base-change-self} and~\cref{lem:br-base-change-tower}.
 \end{proof}

 \begin{definition}[Relative Brauer Group]
   Let $E/K$ be a field extension, we define the relative Brauer group $\Br(E/K)$ to be the kernel of the base change morphism $\Br_{K}^{E}$.
 \end{definition}

 \begin{remark}
   Unpacking the definition of the relative Brauer group, we see that for any $A \in \CSA_{K}$, if $E\ox_{K}A\cong\mat_{n}(E)$ as $E$-algebras, then $\br^{E}_{K}\left([A]_{\sim_{\br}}\right)=1$.
 \end{remark}

 \begin{definition}[Splitting Field]
   For any field extension $E/K$ and any $K$-algebra $A$, we say $E$ is a splitting field of $A$ if and only if $E\ox_{K}A \cong \mat_{n}(E)$ as $E$-algebras for some non-zero $n$. We also say $E$ splits $A$ or $A$ is splited by $E$
   \leanok
   \lean{isSplit}
 \end{definition}

  \begin{theorem}
   \label{thm:split-iff-mem-relative}
   Let $E/K$ be a field extension and $A\in\csa_{K}$, $E$ splits $A$ if and only if $[A]_{\sim_{\br}}\in\br(E/K)$.
   \leanok
   \lean{BrauerGroup.split_iff}
 \end{theorem}

 \begin{proof}
   The ``only if'' part is by definition. For the other direction, we know by definition that $\mat_{n}(E\ox_{K}A)\cong\mat_{m}(E)$ as $E$-algebras for some non-zero $m, n$. By~\cref{thm:wed-artin-algebra}, we find some division algebra $D$ and non-zero natural number $p$ such that $E\ox_{K}A\cong \mat_{p}(D)$ as $E$-algebras. Thus $\mat_{pm}(E)\cong\mat_{pn}\left(E\ox_{K}A\right)\cong\mat_{p^{2}n}(D)$ as $E$-algebras. By~\cref{thm:wed-artin-uniq}, we conclude that $E\cong D$ as $E$-algebras. Hence $E\ox_{K}A\cong \mat_{p}(E)$, in another word, $E$ splits $A$.
 \end{proof}

 \begin{remark}
   In light of~\cref{lem:br-base-change-tower}, if $K$ is algebraic closed then $K$ splits any $K$-algebra $A$. Indeed, $K$ splits $A$ if and only if $[A]_{\sim\br}$ but $[A]_{\sim\br}$ is equal to $1$.
 \end{remark}

 \begin{remark}
   If two $\csa_{K}$ are Brauer equivalent, in another word, $A \sim_{\br_{K}} B$, then $E$ splits $A$ if and only if $E$ splits $B$. Indeed, if $A$ and $B$ are equivalent, then $[A]_{\sim\br} \in \br(E/K)$ if and only if $[B]_{\sim\br}\in\br(E/K)$.
 \end{remark}


 \section{Good Representative Lemma}
 In this section, let $K/F$ be a finite dimensional field extension.

 \begin{lemma}
   \label{lem:good-rep-inv}
   Let $A\in\csa_{F}$ splitted by $K$. There exists a $B\in\csa_{F}$ such that
   \begin{itemize}
     \item $[A]_{\sim_{\br}}[B]_{\sim_{\br}}=1$
     \item there exists $F$-algebra map $K\hookrightarrow B$
     \item ${\left(\dim_{F}K\right)}^{2}=\dim_{F}B$.
   \end{itemize}
   \leanok
   \lean{exists_embedding_of_isSplit}
 \end{lemma}

 \begin{proof}
   Since $K$ splits $A$, we find a non-zero natural number $n$ such that $K\ox_{F}A\cong\mat_{n}K\cong\End_{K}\left(K^{n}\right)$ as $K$-algebras.
   We define an $F$-algebra map $\iota: A\to\End_{F}\left(K^{n}\right)$ by
   \begin{center}
     \begin{tikzcd}
       A \arrow{r} & K \ox_{F} A \arrow{r}{\cong} & \End_K\left(K^n\right) \arrow{r}{|_{F}} & \End_F\left(K^n\right)
     \end{tikzcd},
   \end{center}
   where $|_{F}$ is restriction of scalars. Since $A$ is simple, $\iota$ is injective, therefore $A \cong \iota(A)$ as $F$-algebras. Define $B$ as $C_{\End_{F}\left(K^{n}\right)}(\iota(A))$, the centralizer of the range of $\iota$ in $\End_{F}\left(K^{n}\right)$.
   We construct an embedding $K \hookrightarrow B$ by $r \mapsto (r \cdot \bullet)$

   $B$ is a central $F$-algebra: if $x \in Z(B)$, then $x \in \iota(A)$ because by~\cref{thm:double-centralizer}, it is sufficient to prove that $x$ is in $C_{\End_{F}\left(K^{n}\right)}\left(B\right)$ which follows from the fact that $x \in Z(B)$. In fact, $x \in Z(\iota(A))$: suppose $a \in A$, we need to check $x \cdot \iota(a) = \iota(a)\cdot x$, this is the case because $B$ is defined as the centralizer of $\iota(A)$. Since $\iota(A) \cong A$ as $F$-algebras, $\iota(A)$ is $F$-central, hence $x \in F$.

   $B$ is a simple ring: by~\cref{lem:simple-centralizer}, it is sufficient to prove that $\iota(A)$ is a simple ring which comes from $A \cong \iota(A)$ as $F$-algebras.

   By~\cref{cor:self-tensor-centralizer}, we have $F$-algebra isomorphism $\End_{F}\left(K^{n}\right)\cong \iota(A)\ox_{F}B \cong A \ox_{F}B$. Since $\End_{F}\left(K^{n}\right)\cong\mat_{\dim_{F}\left(K^{n}\right)}\left(F\right)$ as $F$-algebras, we see that $[A]_{\sim_{\br}}$ and $[B]_{\sim_{\br}}$ are inverses.

   By~\cref{lem:dim-centralizer}, $\dim_{F}B\cdot \dim_{F}\iota(A) = \dim_{F}B\cdot\dim_{F}A = \dim_{F}\End_{F}\left(K^{n}\right) = {\left(\dim_{F}\left(K^{n}\right)\right)}^{2}={\left(\dim_{F}K\cdot \dim_{K}\left(K^{n}\right)\right)}^{2}=n^{2}\cdot{\left(\dim_{F}K\right)}^{2}$. On the other hand, since $K\ox_{F}A\cong\mat_{n}K$, we have $\dim_{F}K\ox_{F}A=\dim_{F}K\cdot\dim_{F}A=\dim_{F}\mat_{n}K=\dim_{F}K\dim_{K}\mat_{n}K=n^{2}\dim_{F}K$. Since $\dim_{F}K\ne 0$ , we conclude $\dim_{F}A=n^{2}$. Since $n\ne 0$ and $\dim_{F}(B)\cdot \dim_{F}(A)=n^{2}\dim_{F}(B)=n^{2}{\left(\dim_{F}K\right)}^{2}$, we get the desired result.

 \end{proof}

 \begin{corollary}\label{cor:good-rep}
   Let $A\in\csa_{F}$ splitted by $K$. There exists a $B \in \csa_{F}$ such that
   \begin{itemize}
     \item $[B]_{\sim_{\br}}=[A]_{\sim_{\br}}$
     \item there exists an $F$-algebra map $K\hookrightarrow B$
     \item  ${\left(\dim_{F}K\right)}^{2}=\dim_{F}B$.
  \end{itemize}
 \end{corollary}

 \begin{proof}
   Let $B$ and $\iota : K \hookrightarrow B$ be as in~\cref{lem:good-rep-inv}. Consider $B^{\opp}$ and $K \hookrightarrow B \to B^{\opp}$. This works.
 \end{proof}

 \begin{theorem}\label{thm:good-rep-iff-split}
   Let $A\in\csa_{F}$. $K$ splits $A$ if and only if there exists a $B\in\csa_{F}$ such that
   \begin{itemize}
     \item $[B]_{\sim_{\br}}=[A]_{\sim_{\br}}$
     \item there exists an $F$-algebra map $K \hookrightarrow B$
     \item ${\left(\dim_{F}K\right)}^{2}=\dim_{F}B$.
   \end{itemize}
   \leanok
   \lean{isSplit_iff_dimension}
 \end{theorem}
 \begin{proof}
   The ``if'' direction is~\cref{cor:good-rep}. For the ``only if'' direction, let $B \in\csa_{F}$ and $\iota : K \hookrightarrow B$ be given. We give $B$ a $K$-module structure by right multiplication, that is for any $a \in K$ and $b \in B$, we define $a \cdot b := b \cdot \iota(a)$.
   Since $B$ is a finite dimensional $F$-vector space and $K/F$ is a finite dimensional field extension, $B$ is a finite dimensional $K$-vector space as well. Since $[B]_{\sim_{\br}}=[A]_{\sim_{\br}}$, it is sufficient to show that $K$ splits $B$.
   We define an $F$-bilinear map $\mu : K \to B \to \End_{K}B$ by $(c, a) \mapsto (c \cdot a \cdot \bullet)$ which induce an $F$-linear map $\mu' : K \ox_{F} B \to \End_{K}B$. Since for any $r, c\in K$ and $a \in B$, we have $\mu'\left(r \cdot c\ox a\right)(a') = a a' \iota(rc)=aa'\iota(c)\iota(r)=r\cdot \mu'(c\ox a)$, that is $\mu'$ is $K$-linear as well. Note that \[\mu'(1)=\mu'(1\ox 1) = (1\cdot 1\cdot \bullet) = 1\] and that
   \[
     \begin{aligned}
       \mu'(c\ox a \cdot c'\ox a')(a'') &= \mu'(cc'\ox aa')(a'') \\
                                        &= cc' \cdot aa' \cdot a'' \\
                                        &= aa' a'' \iota(c c') \\
                                        &= a(a' a'' \iota(c'))\iota(c) \\
                                        &=\mu'(c \ox a)(a' a'' \iota(c'))\\
                                        &=\mu'(c\ox a)\left(\mu'(c'\ox a')(a'')\right) \\
                                        &= \left(\mu'(c\ox a)\circ\mu'(c'\ox a')\right)(a'')
     \end{aligned},
     \]
     that is, $\mu'$ is an $K$-algebra map.

   If we can show that $\mu'$ is a bijection, we will prove the result for $K\ox_{F} B \cong \End_{K} B \cong \mat_{\dim_{K} B}K$ as $K$-algebras. By~\cref{cor:alghom-bijective-of-dim-eq}, it is sufficient to show $\dim_{K}K\ox_{F}B = \dim_{K}\End_{K}B$. Let $n$ denote $\dim_{F}K$. Since, $\dim_{F}K\dim_{K}K\ox_{F}B =\dim_{F}K\ox_{F}B=\dim_{F}K \dim_{F}B$. we have $\dim_{K}K\ox_{F}B = \dim_{F}B = {\left(\dim_{F}K\right)}^{2}$. On the other hand, since ${\left(\dim_{F}K\right)}^{2}=\dim_{F} B = \dim_{F}K\dim_{K}B$, we have $\dim_{K} B = \dim_{F}K$; thus $\dim_{K}\End_{K} B = {\left(\dim_{K}B\right)}^{2}={\left(\dim_{F}K\right)}^{2}$ and the result is proved.
 \end{proof}

 In light of~\cref{thm:good-rep-iff-split}, we isolate the following useful definition:
 \begin{definition}[Good Representation]\label{def:good-rep}
   For any $X \in \br(F)$, a good representation of $X$ is an $A \in \csa_{F}$ and an $F$-algebra map $K \hookrightarrow A$ which we often denote as $\iota$ or $\iota_{A}$ such that $[A]_{\sim_{\br}} = X$ and $\dim_{F}A={\left(\dim_{F}K\right)}^{2}$.
   \leanok
   \lean{GoodRep}
 \end{definition}

 \begin{corollary}
   \label{cor:mem-relative-br-iff-good-rep}
   For any $X\in\br(F)$, $X\in\br(K/F)$ if and only if $X$ admits a good representation.
   \leanok
   \lean{mem_relativeBrGroup_iff_exists_goodRep}
 \end{corollary}
 \begin{proof}
   Rephrase of~\cref{thm:good-rep-iff-split} and~\cref{thm:split-iff-mem-relative}.
 \end{proof}

We observe the following easy result about good representations. Let $X \in \br(F)$ and $A$ be a good representation of $X$.

\begin{lemma}
  The range $\iota_{A}(A)$ is a simple ring.
  \leanok
  \lean{GoodRep.ιRange}
\end{lemma}

\begin{proof}
  Because $K$ is a simple ring, $\iota_{A}$ is injective therefore $\iota_{A}(A)\cong K$.
\end{proof}

\begin{lemma}\label{lem:centralizer-range-good-rep}
  $C_{A}\left(\iota_{A}(A)\right) = \iota_{A}(A)$.
  \leanok
  \lean{GoodRep.centralizerιRange}
\end{lemma}
\begin{proof}
  In the language of~\cref{sec:subfield}, $\iota_{A}(A)$ is a subfield of $A$, hence by~\cref{lem:tfae-subfield}, we only need to show $\dim_{F}A = {\left(\dim_{F}\iota_{A}(A)\right)}^{2}$. But $\dim_{F}A={\left(\dim_{F}K\right)}^{2}$ and $\iota(A)\cong K$.
\end{proof}

\begin{construction}
  We give $A$ a $K$-module structure by {\em left} multiplication, that is for any $c \in K$ and $a \in A$, we define $c\cdot a$ to be $\iota_{A}(c)a$. Then $A$ is a finite dimensional $K$-vector space and $\dim_{K}A=\dim_{F}K$: indeed $\dim_{F}K\cdot\dim_{K}A = \dim_{F}K\cdot \dim_{F}K = \dim_{F}A$.
  \leanok
  \lean{GoodRep.dim_eq'}
\end{construction}

\section{The Second Galois Cohomology}

In this section, we construct a group isomorphism between $\br(K/F) \cong \HH^{2}\left(\gal(K/F), K^{\star}\right)$ where $K/F$ is a finite dimensional Galois extension. To keep alignment of the Brauer group, let us use the multiplicative notation for group cohomology. Recall:

\begin{definition}[the Second Group Cohomology]
  Let $G$ be a group and $M$ an abelian group (written multiplicatively) with a $G$-action.

  A function $f : G \times G \to M$ is a {\em 2-cocycle\/} if for all $g,h,j\in G$,
  \[
    f(gh, j)f(g, h) = \left(g\cdot f(h, j)\right) f(g, hj).
  \]

  A function $f : G\times G \to M$ is a {\em 2-coboundary\/} if there exists an $x : G \to M$ such that for all $g, h \in G$
  \[
    \frac{g\cdot x(h)}{x(gh)} x(g) = f(g, h).
  \]
  The second group cohomology $\hh^{2}\left(G, M\right)$ is defined to be the quotient group of 2-cocycles modulo 2-coboundaries.
  \leanok
  \lean{groupCohomology.H2, groupCohomology.IsMulTwoCoboundary, groupCohomology.IsMulTwoCocycle}
\end{definition}

In the following sections of this chapter, we assume that $X \in\br(F)$ and $A$ is a good representation of $X$. We use $\rho, \sigma, \tau$ to denote elements of $\gal(K/F)$. To improve typographic aesthetics of our proofs, we sometimes use subscript to mean function application.


\subsection{From $\br(K/F)$ to $\hh^{2}\left(\gal(K/F),K^{\star}\right)$}

\begin{definition}[Conjugation Factor]
  \label{def:conj-factor}
  With respect to $A$, a conjugation factor of $\sigma$ is a unit $x_{\sigma} \in A^{\star}$ such that for all $c \in K$,
  \[
    x_{\sigma} \iota_{A}(c)x_{\sigma}^{-1} = \iota_{A}(\sigma(c)).
  \]

  A conjugation sequence is a sequence $x:\gal(K/F) \to A^{\star}$ such that for all $\sigma\in\gal(K/F)$, $x_{\sigma}$ is a conjugation factor of $\sigma$. When we want to stress $A$, we say $A$-conjugation factor and $A$-conjugation sequence.
  \leanok
  \lean{GoodRep.conjFactor}
\end{definition}

\begin{remark}\label{rem:conj-factor-alternative-eq}
  When $x_{\sigma}$ is a conjugation factor of $\sigma$, the equalities $x_{\sigma}\iota_{A}(c) = x_{\sigma}\iota_{A}(\sigma(c))$ and $\iota_{A}(c)x_{\sigma}^{-1}=x_{\sigma}^{-1}\iota_{A}(\sigma(c))$ are also useful.
\end{remark}

\begin{construction}
  $A$ has a conjugation sequence:
  let $\sigma \in \gal(K/F)$, we have two $F$-algebra homomorphisms $K \to A$ given by $\iota_{A}$ and $\iota_{A}\circ \sigma$. Applying~\cref{thm:skolem-noether} to $\iota_{A}$ and $\iota_{A}\circ \sigma$ gives us the desired conjugation factor.
  \leanok
  \lean{GoodRep.aribitaryConjFactor}
\end{construction}

\begin{construction}
  If $x$ is a conjugation factor of $\sigma$ and $y$ of $\tau$, then $xy$ is a conjugation factor of $\sigma\tau$. For any $c \in K$
  \[
    \iota_{A}(\sigma(\tau(c))) = x\iota_{A}(\tau(c))x^{-1} = xy\iota_{A}(c)y^{-1}x^{-1}=\left(xy\right)\iota_{A}{\left(xy\right)}^{-1}.
  \]
  \leanok
  \lean{GoodRep.mul'}
\end{construction}

\begin{lemma}[Twisting Conjugation Factors]
  If $x$ and $y$ are two conjugation factors of $\sigma$, then there exists a unique $c \in K$ such that $x = y\iota_{A}(c)$.
  \leanok
  \lean{GoodRep.conjFactor_rel, GoodRep.conjFactorTwistCoeff}
\end{lemma}

\begin{proof}
  The uniqueness is clear: suppose $x = y\iota_{A}(c)=y\iota_{A}(c')$, then $c = c'$ because $x, y$ are units and $\iota_{A}$ is injective. We first observe that $y^{-1}x\in C_{A}(\iota(A))$: for any $z \in K$, $y^{-1}x\iota_{A}(z)=y^{-1}\iota_{A}(\sigma(z))x=\iota_{A}(z)y^{-1}x$ (by~\cref{rem:conj-factor-alternative-eq}). By~\cref{lem:centralizer-range-good-rep}, $y^{-1}x\in \iota(A)$, that is for some $z \in K$, we have that $y^{-1}x = \iota_{A}(z)$ and the claim is proved.
\end{proof}

We denote such $c$ by $\twist^{\sigma}(x, y)$ or $\twist^{\sigma}_{x, y}$, when $\sigma$ is clear from context, we often omit the superscript. With this notation, $x = y\iota_{A}(\twist_{x,y})$.

\begin{remark}
  $\twist(x, x)$ is equal to $1$ by uniqueness.
\end{remark}

\begin{remark}
  In fact, $\twist(x, y)$ is in $K^{\star}$ and $\twist(x, y)^{-1}=\twist(y, x)$.
\end{remark}

\begin{lemma}
  If $x$ and $y$ are conjugation factors for $\sigma$, $x=\iota_{A}(\sigma(\twist_{x, y}))y$.
\end{lemma}


%%% Local Variables:
%%% mode: LaTeX
%%% TeX-master: "../print"
%%% End:

\subsection{$\HH^{2} \circ \cross$ and $\cross \circ \HH^{2}$}

For a finite dimensional Galois extension of field $K/F$, we have constructed two functions $\HH^{2}$ and $\cross$ between the second cohomology group $\HH^{2}\left(\gal(K/F), K^{\star}\right)$ and the relative Brauer group $\br(K/F)$. In this section, we prove that they are mutual inverse to one another,

\begin{lemma}\label{lem:relative-br-snd-inverse-1}
  The composition of $\cross$ and $\HH^{2}$ is the identity:
  \begin{center}
    \begin{tikzcd}
      \displaystyle \HH^2\left(\gal(K/F), K^\star\right) \arrow{r}{\cross} \arrow[bend right = 15, swap]{rr}{\id} &
      \br(K/F) \arrow{r}{\HH^2} & \displaystyle \HH^2\left(\gal(K/F), K^\star\right).
    \end{tikzcd}
  \end{center}
  \leanok
  \lean{RelativeBrGroup.toSnd_fromSnd}
\end{lemma}
\begin{proof}
  Let $\mfa$ be any 2-cocycle, by~\cref{lem:cross-product-basis-conj}, we notice that $x : \sigma \mapsto \Delta_{\sigma, 1}$ is a conjugation sequence for $\cross_{\mfa}$. Hence by~\cref{con:good-rep-to-2-cocycles}, ~\cref{cor:to-2-cocylce-wd} and~\cref{thm:snd-coh-to-relative-br}, we evaluate the composition at $\mfa$ as:
  \begin{center}
    \begin{tikzcd}
      {[\mfa]} \arrow[mapsto]{r} & {\left[\cross_\mfa\right]_{\sim_{\br}}} \arrow[mapsto]{r}& {\left[(\sigma,\tau)\mapsto\comp^{x}_{\Delta_{\sigma, 1}, \Delta_{\tau, 1}, \Delta_{\sigma\tau, 1}}\right]}
    \end{tikzcd}.
  \end{center}
  That is, we need to show that $\mfa$ and $(\sigma, \tau) \mapsto \comp_{\Delta_{\sigma, 1}, \Delta_{\tau, 1}, \Delta_{\sigma\tau, 1}}$ are 2-cohomologous. In fact, they are equal.
  By\cref{con:compare-conj-factors}~, we have that $\iota_{\cross_{\mfa}}\left(\comp^{x}_{\Delta_{\sigma, 1}, \Delta_{\tau, 1}, \Delta_{\sigma\tau, 1}}\right) = \Delta_{\sigma, 1}\Delta_{\tau, 1}\Delta_{\sigma\tau, 1}^{-1} = \mfa(\sigma,\tau)\cdot \Delta_{\sigma\tau, 1}\Delta_{\sigma\tau, 1}^{-1} = \mfa(\sigma,\tau)\cdot 1 = \Delta_{\id,\mfa(\sigma,\tau)}$ which is precisely $\iota_{\cross_{\mfa}}(\mfa(\sigma,\tau))$.
\end{proof}

\begin{lemma}\label{lem:relative-br-snd-inverse-2}
  The composition of $\HH^{2}$ and $\cross$ is the identity:
  \begin{center}
    \begin{tikzcd}
      \br(K/F) \arrow{r}{\HH^2} \arrow[bend right = 15, swap]{rr}{\id} &
      \HH^2\left(\gal(K/F), K^\star\right) \arrow{r}{\cross} &
      \br(K/F)
    \end{tikzcd}.
  \end{center}
  \leanok
  \lean{RelativeBrGroup.fromSnd_toSnd}
\end{lemma}
\begin{proof}
  Let $X \in \br(K/F)$, $A$ be an arbitrary good representation of $X$ and $x$ be an arbitrary $A$-conjugation sequence which exists by~\cref{cor:mem-relative-br-iff-good-rep} and~\cref{con:exists-conj-seq}. By~\cref{def:good-rep}, $X = \left[A\right]_{\sim_{\br}}$.
  Hence by~\cref{cor:to-2-cocylce-wd} and~\cref{thm:snd-coh-to-relative-br}, we evaluate the composition at $X$ as:
  \begin{center}
    \begin{tikzcd}
      {[A]_{\sim_{\br}}} \arrow[mapsto]{r} &
      {\left[\mathcal{B}^2_x\right]} \arrow[mapsto]{r} &
      {\left[\cross_{\mathcal{B}^2_x}\right]}
    \end{tikzcd}.
  \end{center}
  Hence we need to prove that $A$ and $\cross_{\mathcal{B}^{2}_{x}}$ are Brauer equivalent. We will show that they are isomorphic as $F$-algebras. since $\{x_{\sigma}|\sigma\in\gal(K/F)\}$ is a $K$-basis for $A$ and $\{\Delta_{\sigma, 1}|\sigma \in \gal(K/F)\}$ is a $K$-basis for $\cross_{\mathcal{B}^{2}_{x}}$, they are certainly isomorphic as $K$-modules. Let $\phi : \cross_{\mathcal{B}^{2}_{x}} \cong A$ be the $K$-linear isomorphism defined by $\Delta_{\sigma, 1} \mapsto x_{\sigma}$, since the $K$-action on $A$ and the $F$-action on $A$ are compatible (\cref{con:good-rep-mod}), $\phi$ is also an $F$-linear isomorphism. Like in~\cref{thm:snd-coh-to-relative-br}, we check that $\phi(1) = 1$ and $\phi(xy)=\phi(x)\phi(y)$ for all $x,y \in A$:
  \begin{enumerate}
    \item preservation of one: by~\cref{con:compare-conj-factors}, we have
          \[
          \begin{aligned}
            \phi(1)
            &= \phi\left(\Delta_{\id,\mathcal{B}^{2}_{x}(\id,\id)^{-1}}\right) \\
            &= \mathcal{B}^{2}_{x}(\id,\id)^{-1} \phi\left(\Delta_{\id, 1}\right) \\
            &= \mathcal{B}^{2}_{x}(\id,\id)^{-1} x_{\id} \\
            &= \comp_{x_{\id},x_{\id},x_{\id}}^{-1} x_{\id} \\
            &= \comp_{x_{\id},x_{\id},x_{\id}} x_{\id} x_{\id} x_{\id}^{-1} \\
            &= x_{\id} x_{\id}^{-1} \\
            &= 1.
          \end{aligned}
          \]
    \item preservation of multiplication: let $\sigma,\tau\in\gal(K/F)$ and $c,d \in K$, by~\cref{con:compare-conj-factors} and~\cref{def:conj-factor}, we have
          \[
          \begin{aligned}
            \phi\left(\Delta_{\sigma,c}\Delta_{\tau,d}\right)
            &= \phi\left(\Delta_{\sigma\tau, c\sigma(d)\mathcal{B}^{2}_{x}(\sigma,\tau)}\right) \\
            &= c\sigma(d)\mathcal{B}^{2}_{x}(\sigma,\tau) \,\cdot\, \phi\left(\Delta_{\sigma\tau, 1}\right) \\
            &= c\sigma(d)\mathcal{B}^{2}_{x}(\sigma,\tau) \,\cdot\, x_{\sigma\tau} \\
            &= c\sigma(d)\comp_{x_{\sigma},x_{\tau},x_{\sigma\tau}} \,\cdot\, x_{\sigma\tau} \\
            &= c\sigma(d)\,\cdot\, \iota_{A}\left(\comp_{x_{\sigma},x_{\tau},x_{\sigma\tau}}\right)x_{\sigma\tau} \\
            &= c\sigma(d) \,\cdot\, x_{\sigma}x_{\tau}\\
            %%%%
            %%%%
            \phi\left(\Delta_{\sigma,c}\right)\phi\left(\Delta_{\tau,d}\right)
            &=\left(c \cdot \phi\left(\Delta_{\sigma, 1}\right)\right)
              \left(d \cdot \phi\left(\Delta_{\tau, 1}\right)\right) \\
            &= \left(c \cdot x_{\sigma}\right) \left(d \cdot x_{\tau}\right)\\
            &= c \,\cdot\, x_{\sigma}\iota_{A}(d) x_{\tau} \\
            &= c\sigma_{d} \,\cdot\, x_{\sigma}x_{\tau}.
          \end{aligned}
          \]
  \end{enumerate}
\end{proof}

\begin{corollary}
  For a finite dimensional and Galois extension of field $K/F$, the relative Brauer group $K/F$ bijects to the second cohomology group $\HH^{2}\left(\gal(K/F), K^{\star}\right)$ by the following commutative diagram
  \begin{center}
  \begin{tikzcd}
    \br(K/F) \arrow{r}{\HH^2} \ar[equal]{d} & \HH^2\left(\gal(K/F), K^\star\right) \arrow[equal]{d} \\

    \br(K/ F) & \HH^2\left(\gal(K/F), K^{\star}\right) \arrow{l}{\cross}
  \end{tikzcd}.
\end{center}
\leanok
\lean{RelativeBrGroup.equivSnd}
\end{corollary}

\begin{proof}
  Exactly~\cref{lem:relative-br-snd-inverse-1} and~\cref{lem:relative-br-snd-inverse-2}.
\end{proof}

%%% Local Variables:
%%% mode: LaTeX
%%% TeX-master: "../print"
%%% End:


%%% Local Variables:
%%% mode: LaTeX
%%% TeX-master: "print"
%%% End:


\end{document}
